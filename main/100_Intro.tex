\label{intro:chapter}
Scheduling an event for a group of agents is a challenging problem. 
It tends to be tedious and time-consuming for everyone involved.
In practice, procrastination and strategic behavior of agents are often a problem, but even if we assume that agents are truthful, prompt, and indifferent among the possible outcomes, the group scheduling problem exhibits an interesting algorithmic problem. 
We formally model this as an optimization problem in which we are allowed to query availability of agents in order to learn about their availability (which is unknown a priori) with the assumption that agents are prompt and truthful in responding to such queries. We wish to minimize a certain cost function such as the total number of queries. We provide an intuitive, efficient algorithm that solves this optimization problem for a fairly broad class of settings. We define the underlying problem as a general optimization problem, how it can be applied to other real world settings than group scheduling problems.

Another setting that is relevant to group scheduling comes from group assignment problems. Consider a setting where the organizer wishes to assign agents to social activities. Agents may have preferences over activities as well as the number of participants in the activity they are assigned to (for now, suppose agents have anonymous preferences in that they do not care who the participants are, but only the number of participants). For instance, agents may prefer more participants in social networking receptions, but wish to have not too many participants in a table tennis tournament. Naturally the organizer wishes to assign as many agents to activities as possible, but at the same time he wishes to ensure that every individual is satisfied with the assignment. We study the solution concept of individual rationality and stability which require that there is no single-agent deviation from an assignment. This problem is known to be NP-hard in general even if certain restrictions of preferences of agents are assumed. In this work we first show that the problem is still computationally difficult even if we seek a small, fixed-size solution in general. However, we also show that for a relaxed version of the stability requirement, the problem admits an efficient algorithm for finding a fixed-size, small solution. 

We then futher generalize the group assignment problem by considering the cases where agents have friend and/or enemy relationships which introduce additional constraints or preferences. Because most of the technical results we obtain for the anonymous case are already hardness results, we focus on a special case where there is only one activity, which is known to be easy (i.e., polynomial-time solvable) when agents have anonymous preferences. We show if the number of friends or enemies of an agent is large (namely, two or more), then the problem of finding a stable solution is NP-hard. Furthermore, finding a fixed-size, small solution also depends on the cardinality of the largest friend-set or enemy-set, which naturally categorizes the underlying problem into different complexity classes. We also show that the problem becomes computationally easier if the friends and enemies relationship is symmetric.

Lastly we consider strategic agents in this setting where agents may report false information to the event organizer. We show that in general finding a stable solution and strategy-proofness are incompatible in the activity selection problem, but we also provide a socially optimal, computationally efficient, and strategy-proof mechanism in the special case where there is only one activity and the preferences of agents align with the goal of the designer (so as to maximize the number of participants).


\section{Organization of the Thesis}
Chapter~\ref{bdoodle:chapter} introduces the Batched Doodle Problem which is motivated by a well-known tool for group scheduling, called ``Doodle''~\footnote{http://www.doodle.com/}. Through the Batched Doodle Problem, we propose an algorithmic approach to group scheduling by optimizing the duration of scheduling process and inconvenience caused by it. Chapter~\ref{matrix:chapter} generalizes the Batched Doodle Problem further to the Probabilistic Matrix Inspection Problem, which has many applications other than just group scheduling. In both problems, we focus on designing efficient algorithms that find a feasible solution at minimum (expected) cost. 
We then turn our attention to group assignment problems in Chapter~\ref{GASP:chapter} by considering the Group Activity Selection Problem which was originally introduced by Darmann et al.~\cite{GASP12WINE} In the Group Activity Selection Problem, agents are assumed to have anonymous preferences over activities in that not only do agents care about which activity they are assigned to, but they also care about the number of participants. The objective is to assign agents to activities subject to individual rationlity or Nash stability conditions. The problem is known to be NP-hard even under some natrual restrictions on preferences of agents. In this work, we further investigate complexity of this problem by parameterizing the size of a solution, and show that the problem is still computationally difficult to solve. In Chapter~\ref{SIP:chapter}, we define the Stable Invitations Problem which is motivated by the Group Activity Selection Problem. In the Stable Invitations Problem, there is only one activity, rather than many, but agents are no longer assumed to have anonymous preferences. Instead, they have friend and enemy relationships. We investigate both classical and parameterized complexity of this problem, by considering different restrictions on preferences and friend/enemy relationships. Lastly, in Chapter~\ref{GT:chapter} we introduce strategic agents. When agents act strategically, we show that it is in general impossible to obtain a strategy-proof mechanism that can find a desirable solution in the Group Activity Selection Problem or the Stable Invitations Problem. However, in a special case when all agents prefer more participants than fewer participants and there is only one activity, we obtain a strategy-proof, optimal, and computationally efficient mechanism. 
Chapter~\ref{discussion:chapter} discusses several open problems motivated by this thesis. 
