\label{discussion:chapter}

In Chapter~\ref{bdoodle:chapter}, we modeled group scheduling problems as an optimization problem called the Batched Doodle Problem by considering time and inconvenience as two important meausres of efficiency of the scheduling processes. We prsented an efficient algorithm that can find an optimal partition of date/time options into batches, which can substantially improve the overall efficiency of the scheduling process compared to Doodle, under various cost functions. In particular, when agents are relatively less busy, the number of agents is small, and the number of options is large, our proposed Batched Doodle approach can shine. It will be an interesting driection for future work to apply our theoretic results to real-world settings.

In Chapter~\ref{matrix:chapter}, we further generalized the Batched Doodle Problem by relaxing the condition that an event organizer must poll all agents at the same time. Instead, the organizer can query an individual for a specific option (or a candidate), which can model many other settings than just group scheduling as we disucssed. We presented an efficient algorithm which is motivated by an intuitive greedy approach. Both in the Batched Doodle Problem and the Probabilistic Matrix Inspection Problem, a crucial assumption is made: The event organizer has prior estimates on the likelihood of each agent being available for (approving of) a certain option (candidate). It is a challenging task to gather accurate estimates, which may come from different sources. 

In Chapter~\ref{GASP:chapter}, we considered the Group Activity Selection Problem in which an event organizer is to assign agents to activities given their preferences over activities as well as the number of attendees. Motivated by game-theoretic solution concepts in which the organizer wishes to avoid different levels of instability in a resulting assignment. Finding such solutions was known to be NP-hard in most cases, even if preferences of agents are restricted. However, we analyze parameterized complexity of this problem which revealed that different solution concepts and restrictions on inputs led to different complexity classes in the W-hierarchy. 

In Chapter~\ref{SIP:chapter}, we generalized a special case of the Group Activity Selection Problem, and proposed the Stable Invitations Problem in which there is only one activity, but agents now have friends and enemies. We consider various special cases of the problem by assuming that the number of friends and/or enemies is bounded above by a constant. We showed that in most cases (when agents can have more than two friends and enemies combined), the problem of finding an individually rational or stable solution is NP-hard. However, when we consider parameterized complexity, we observed that two solution concepts differ substantially as the former 