\label{discussion:chapter}

In Chapter~\ref{bdoodle:chapter}, we modeled group scheduling problems as an optimization problem called the Batched Doodle Problem by considering time and inconvenience as two important meausres of efficiency of the scheduling processes. We prsented an efficient algorithm that can find an optimal partition of date/time options into batches, which can substantially improve the overall efficiency of the scheduling process compared to Doodle, under various cost functions. In particular, when agents are relatively less busy, the number of agents is small, and the number of options is large, our proposed Batched Doodle approach can shine. It will be an interesting driection for future work to apply our theoretic results to real-world settings.

In Chapter~\ref{matrix:chapter}, we further generalized the Batched Doodle Problem by relaxing the condition that an event organizer must poll all agents at the same time. Instead, the organizer can query an individual for a specific option (or a candidate), which can model many other settings than just group scheduling as we disucssed. We presented an efficient algorithm which is motivated by an intuitive greedy approach. 