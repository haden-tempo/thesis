\documentclass[11pt]{report}
\usepackage{suthesis-2e}

% \usepackage{hyperref}
\usepackage{amsmath,amsthm,amsfonts,amssymb}
\usepackage{multirow}
\usepackage{thmtools}
\usepackage{enumitem}
\usepackage{graphicx}
\usepackage{mathptmx}
\DeclareMathAlphabet{\mathcal}{OMS}{cmsy}{m}{n}
% \usepackage{showframe}

\usepackage{algorithm}
\usepackage{algpseudocode}
\input{code}


% \newcommand\Chapter[2]{
%   \chapter[#1: {\itshape#2}]{#1\\[2ex]\Large\itshape#2}
% }
% ====================================================================
% \newcommand{\Pr}{\mathbf{Pr}}
\newcommand{\Exp}{\mathbb{E}}

% ====================================================================
\newcommand{\BDP}{{\normalfont \texttt{BDP}}}
\newcommand{\BDPs}{{\normalfont \texttt{BDP}}~}
\newcommand{\PMIP}{{\normalfont \texttt{PMIP}}}
\newcommand{\PMIPs}{{\normalfont \texttt{PMIP}}~}

\newcommand{\Time}{{\normalfont \textsf{Time}}}
\newcommand{\Inconvenience}{{\normalfont \textsf{Inconvenience}}}
\newcommand{\Times}{{\normalfont \textsf{Time}}~}
\newcommand{\Inconveniences}{{\normalfont \textsf{Inconvenience}}~}

\newcommand{\feasible}{\emph{feasible}}
% ====================================================================
\newcommand{\GASP}{{\normalfont \texttt{GASP}}} 
\newcommand{\GASPs}{{\normalfont \texttt{GASP}}~} 

\newcommand{\AOIP}{{\normalfont \texttt{AOIP}}} 
\newcommand{\AOIPs}{{\normalfont \texttt{AOIP}}~} 

\newcommand{\SIP}{{\normalfont \texttt{SIP}}} 
\newcommand{\SIPs}{{\normalfont \texttt{SIP}}~} 

\newcommand{\void}{a_{\emptyset}}

\newcommand{\true}{\normalfont \textsf{true}}
\newcommand{\false}{\normalfont \textsf{false}}

\newcommand{\INC}{{\normalfont \textsf{INC}}} 
\newcommand{\DEC}{{\normalfont \textsf{DEC}}} 
% ====================================================================


% ====================================================================
\theoremstyle{plain}

\newtheorem{theorem}{Theorem}
\newtheorem*{theorem*}{Theorem}
\newtheorem{lemma}{Lemma}
\newtheorem{corollary}{Corollary}

\theoremstyle{definition}
\newtheorem{definition}{Definition}
\newtheorem{example}{Example}
% ==================================================================

\dept{Computer Science}

\setcounter{tocdepth}{1}

\begin{document}
\title{GROUP SCHEDULING AND ASSIGNMENT: \\
COMPLEXITY AND ALGORITHMS}
\author{Hooyeon Lee}
\submitdate{December 2016}

\principaladviser{Virginia V. Williams}
\firstreader{Ashish Goel}
\secondreader{Yoav Shoham}
 
\beforepreface
\prefacesection{Abstract}

Scheduling an event for a group of agents is a challenging problem. 
It tends to be tedious and time-consuming for everyone involved. 
In practice, procrastination and strategic behavior of agents are often a problem, but even if we assume that agents are truthful, prompt, and indifferent among the possible outcomes, the group scheduling problem exhibits an interesting algorithmic problem. Assuming truthful and prompt behavior of agents, we consider settings where the event organizer has probabilistic estimates on availability of agents and is allowed to query agents for their actual availability. Naturally, it is desirable to minimize the (expected) number of queries so as to optimize the time and inconvenience caused by the scheduling process. We consider two models that are motivated by an existing tool that is widely used in practice today, and offer intuitive and computationally efficient algorithms that are applicable to group scheduling settings. Furthermore, we also discuss how our algorithms can be used in different domains than group scheduling. 

Another setting that is relevant to group scheduling comes from group assignment problems. Imagine that an event organizer wishes to assign agents to social activities. Agents may have preferences over activities as well as the number of participants in the activity they are assigned to. The organizer would like to assign as many agents to activities as possible, but not at the cost of upsetting some agents by disregarding their preferences. Finding a Nash stable solution is an interesting algorithmic question of its own in this setting, and it has been studied in the literature that many variants of the problem are computationally difficult (i.e., NP-hard). In this thesis, we take an extra step to investigate some of these problems by analyzing their parameterized complexity when the size of a solution is parameterized. Parameterized complexity offers a finer scale than classical complexity, and we classify many variants of the group assignment problem into different complexity classes in the W-hierarchy. Motivated by this problem, we also consider another class of group assignment problems in which agents have friends and enemies. Friend and enemy relationships introduce combinatoric complications into the setting, and we show that the number of friends and/or enemies each agent has plays an important role in complexity of these problems. In addition, we also show that symmetric relationship reduces complexity of these problems to a great extent.

Lastly, we consider strategic agents in the group assignment problems we mentioned earlier. We show that in general it is impossible to obtain a strategy-proof mechanism that can also find a Nash stable solution in these settings, even if we restrict preferences of agents. The only positive result we show is 


Lastly we consider strategic agents in this setting where agents may report false information to the event organizer. We show that in general finding a stable solution and strategy-proofness are incompatible in the activity selection problem, but we also provide a socially optimal, computationally efficient, and strategy-proof mechanism in the special case where there is only one activity and the preferences of agents align with the goal of the designer (so as to maximize the number of participants).


\prefacesection{Acknowledgments}
It has been quite a journey to pursue a doctoral degree in computer science, and finally put an end to it. My journey began two decades ago when I first used a personal computer to play games. Computers and programs were a fascinating thing, and problem solving became my hobby and specialty as I participated in various programming contests -- mostly, algorithm design and problem solving. 

Studying computer science as my major in undergrad was a no-brainer for me -- it was the only thing I knew how to do and the only thing I wanted to do better. Just a couple of months before my graduation, Johannes asked me what I planned to do after graduation; at the time, I accepted a full-time job offer from a trading company in Chicago at which I interned before. I told him, and he suggested whether I would give a try to work as a research assistant at his database lab, and possibly apply for graduate school a year later. I thought about it, and I took the offer. 


% My journey of studying computer science started nearly two decaeds ago when I first used a personal computer to play games. Just like anyone else around me, I wanted to create a fun game of my own, and decided to learn how to write a program. I first learned QBasic, and it was fun to write a program that prints out a bunch of characters on an old CRT monitor. At the time playing games was more fun to me than writing programs -- partly because I was not as motivated -- and I was more interested in studying for math olympiads than programming.
%
% Then, in year 2000, I randomly stumbled upon a programming contest by a recommendation from my middle school teacher -- not because I knew how to write a program, but because I was good at using some of the PC tools for homework. I thought it would be fun to participate, so I did. It turned out that I was taking a written exam on basic programming questions, such as ``what will be the output of the following program?'' I was given thirty multiple-choice questions, and I was not able to answer any of them with confidence  -- after all, I knew nothing about programming.
%
% For some reason, this bothered me to a great extent so that I asked my parents to let me study computer programming more seriously at a private institute. The institute happened to be the most successful institute in South Korea at the time in the sense that every year they would send three to four kids to the International Olympiad in Informatics (out of four kids per country). Without knowing this, I took an entrance exam for the institute, and failed; I was asked to write a program in any language of my choice which could output the following:
%
% \begin{verbatim}
%  1  2  3  4  5
% 16 17 18 19  6
% 15 24 25 20  7
% 14 23 22 21  8
% 13 12 11 10  9
% \end{verbatim}
%
% To my surprise, I did not know how to do this even though I had written hundreds of for-loops by then, and I was ashamed; to their surprise, I knew nothing, and I wanted to attend this institute. After all, the institute was to train kids who already know how to write programs but need to practice more on problem solving skills for the contests. As my father recalls, he saw something special in my eyes that day, and he was sure I wanted to, and needed to, try it anyway. So, I did enroll for the most basic level class where I would start from the beginning -- by printing out ``hello, world.''
%
% A year later, I participated in the same contest again, and I barely passed the preliminary contest as the very last perosn to move on to the next round, a minucipal contest. Again, I barely passed the minicipal contest to move on to a national contest as the second-to-last person. In the national contest, I barely made it to a sliver medalist, which meant that I would be able to participate in a training camp for the International Olympiad in Informatics. Three years later, I represented South Korea to participate in the Internatinoaly Olympiad in Informatics. It was the moment when I first felt that I had accomplished something great that I had been longing for.

% After all, I think I was lucky enough to find the right resources at the right time.  I have enjoyed studying computer science, learning how research is done at top universities, and solving problems that are exciting to me.
%
%

\afterpreface

% TODO: Move Introduction to Abstract?
\chapter{Introduction}
\label{intro:chapter}
Scheduling an event for a group of agents is a challenging problem. 
It tends to be tedious and time-consuming for everyone involved.
In practice, procrastination and strategic behavior of agents are often a problem, but even if we assume that agents are truthful, prompt, and indifferent among the possible outcomes, the group scheduling problem exhibits an interesting algorithmic problem. 
We formally model this as an optimization problem in which we are allowed to query availability of agents in order to learn about their availability (which is unknown a priori) with the assumption that agents are prompt and truthful in responding to such queries. We wish to minimize a certain cost function such as the total number of queries. We provide an intuitive, efficient algorithm that solves this optimization problem for a fairly broad class of settings. We define the underlying problem as a general optimization problem, how it can be applied to other real world settings than group scheduling problems.

Another setting that is relevant to group scheduling comes from group assignment problems. Consider a setting where the organizer wishes to assign agents to social activities. Agents may have preferences over activities as well as the number of participants in the activity they are assigned to (for now, suppose agents have anonymous preferences in that they do not care who the participants are, but only the number of participants). For instance, agents may prefer more participants in social networking receptions, but wish to have not too many participants in a table tennis tournament. Naturally the organizer wishes to assign as many agents to activities as possible, but at the same time he wishes to ensure that every individual is satisfied with the assignment. We study the solution concept of individual rationality and stability which require that there is no single-agent deviation from an assignment. This problem is known to be NP-hard in general even if certain restrictions of preferences of agents are assumed. In this work we first show that the problem is still computationally difficult even if we seek a small, fixed-size solution in general. However, we also show that for a relaxed version of the stability requirement, the problem admits an efficient algorithm for finding a fixed-size, small solution. 

We then futher generalize the group assignment problem by considering the cases where agents have friend and/or enemy relationships which introduce additional constraints or preferences. Because most of the technical results we obtain for the anonymous case are already hardness results, we focus on a special case where there is only one activity, which is known to be easy (i.e., polynomial-time solvable) when agents have anonymous preferences. We show if the number of friends or enemies of an agent is large (namely, two or more), then the problem of finding a stable solution is NP-hard. Furthermore, finding a fixed-size, small solution also depends on the cardinality of the largest friend-set or enemy-set, which naturally categorizes the underlying problem into different complexity classes. We also show that the problem becomes computationally easier if the friends and enemies relationship is symmetric.

Lastly we consider strategic agents in this setting where agents may report false information to the event organizer. We show that in general finding a stable solution and strategy-proofness are incompatible in the activity selection problem, but we also provide a socially optimal, computationally efficient, and strategy-proof mechanism in the special case where there is only one activity and the preferences of agents align with the goal of the designer (so as to maximize the number of participants).


\section{Organization of the Thesis}
Chapter~\ref{bdoodle:chapter} introduces the Batched Doodle Problem which is motivated by a well-known tool for group scheduling, called ``Doodle''~\footnote{http://www.doodle.com/}. Through the Batched Doodle Problem, we propose an algorithmic approach to group scheduling by optimizing the duration of scheduling process and inconvenience caused by it. Chapter~\ref{matrix:chapter} generalizes the Batched Doodle Problem further to the Probabilistic Matrix Inspection Problem, which has many applications other than just group scheduling. In both problems, we focus on designing efficient algorithms that find a feasible solution at minimum (expected) cost. 
We then turn our attention to group assignment problems in Chapter~\ref{GASP:chapter} by considering the Group Activity Selection Problem which was originally introduced by Darmann et al.~\cite{GASP12WINE} In the Group Activity Selection Problem, agents are assumed to have anonymous preferences over activities in that not only do agents care about which activity they are assigned to, but they also care about the number of participants. The objective is to assign agents to activities subject to individual rationlity or Nash stability conditions. The problem is known to be NP-hard even under some natrual restrictions on preferences of agents. In this work, we further investigate complexity of this problem by parameterizing the size of a solution, and show that the problem is still computationally difficult to solve. In Chapter~\ref{SIP:chapter}, we define the Stable Invitations Problem which is motivated by the Group Activity Selection Problem. In the Stable Invitations Problem, there is only one activity, rather than many, but agents are no longer assumed to have anonymous preferences. Instead, they have friend and enemy relationships. We investigate both classical and parameterized complexity of this problem, by considering different restrictions on preferences and friend/enemy relationships. Lastly, in Chapter~\ref{GT:chapter} we introduce strategic agents. When agents act strategically, we show that it is in general impossible to obtain a strategy-proof mechanism that can find a desirable solution in the Group Activity Selection Problem or the Stable Invitations Problem. However, in a special case when all agents prefer more participants than fewer participants and there is only one activity, we obtain a strategy-proof, optimal, and computationally efficient mechanism. 
Chapter~\ref{discussion:chapter} discusses several open problems motivated by this thesis. 


\chapter{Batched Doodle Problem}
%TODO: 'm' refers to the numbebr of batches and 's' refers to the number of options. Must not mix up!
%TODO: Everyone should be 'she'?
\label{bdoodle:chapter}

Scheduling an event for a group of invitees is a frustrating task; it tends to be tedious and time consuming.
A typical scheduling process can be described as iterative approval voting.
First, an event organizer selects a candidate set of date/time options, and asks invitees to respond with their availability. 
Given the responses, the organizer then chooses an agreeable option and announces it, or she may repeat the process by proposing another set of options if no feasible option is found. 
Naturally the organizer and her invitees wish to reach an agreement within a small number of iterations and proposed options -- the more iterations and proposed options there are, the more laborious a scheduling process becomes. 

There exist several software tools that are designed to help an event organizer handle a scheduling process more efficiently -- one of the most well-known tools is Doodle~\footnote{http://www.doodle.com}. 
In Doodle, an organizer can simply list as many date/time options as she likes, 
and each invitee is asked to respond with her availability. 
Essentially, invitees participate in approval voting on all options proposed by the organizer.
Hence if too many options are proposed, then invitees are given the burden of answering them all. 
This often leads to undesired behaviors of invitees such as herding or procrastination, instead of honest, quick responses~\cite{zou2015strategic}. 
On the other hand, if the organizer proposes only few options, there may not exist an agreeable outcome after all, which may result in another iteration of proposals and responses. 
In fact, surveys find that the most challenging part of group scheduling is due to ``chasing people who do not answer'' and ``finding a suitable time.''~\footnote{http://en.blog.doodle.com/2012/07/26/new-findings-a-small-number-of-initiators-organize-most-of-the-meetings/}

Doodle has many practical advantages. 
One of them is its simplicity -- invitees simply need to approve a subset of options based on their availability.
Another is a short duration of scheduling process -- the duration it takes until
the last invitee responds and the meeting time is settled. Each invitee
needs to respond only once, which limits the degree to which the
process is hijacked by invitees' delayed responses. But Doodle has many
drawbacks, even in this idealized setting. In particular, Doodle
forces the invitees to examine many potential date/time options. This can be
inconvenient not so much due to the effort involved (though that is a
factor), but mostly because invitees need to block their available slots
off until a option is announced -- in a way, their `free time' is being hostaged until the final schedule is set.
To quantify this inconvenience, we use the expected number of time
slots floated by the organizer as a proxy.

In order to avoid incurring much inconvenience, 
the organizer can select just a handful number of options,
and poll the invitees about those. If a feasible option is not found
among them, then repeat with another batch of date/time options. 
We call this broad class of polling mechanisms \emph{B-Doodle} (for ``Batched Doodle").
Clearly, Doodle is a special case with one batch consisting of all
options. Another extreme case is the OAAT (one-at-a-time)
mechanism, in which the organizer tests a single option at each
iteration. Doodle minimizes the total number of iterations, whereas 
OAAT minimizes (expected) inconvenience caused by the scheduling process.
In-between lie many other mechanisms, with different
batching schemes, that trade off time against inconvenience
differently.

Figure~\ref{bdoodle:fig:Pareto_scatter} illustrates this via a simple example.
In this scenario there are six options, four invitees, each of whom
is available at each option independently with probability $.8$.
The figure depicts a scatter plot of all 32 different B-Doodle mechanisms,
including Doodle and OAAT. The vertical axis depicts the expected
number of rounds to determine the option, and the horizontal axis
the expected inconvenience.

\begin{figure}[h!] \small
\centering
\includegraphics[scale=0.48]{plots/pareto_scatter_fin.eps}
\caption{A scatter plot of B-Doodle Mechanisms given four invitees and six options.}
\label{bdoodle:fig:Pareto_scatter}
\end{figure}

We can clearly observe a time-inconvenience Pareto frontier in Figure~\ref{bdoodle:fig:Pareto_scatter}. 
In additino, if there is an overall cost function combining time
and inconvenience, one can identify an optimal B-Doodle mechanism
along this frontier as illustrated in
Figure~\ref{bdoodle:fig:Pareto_objective}. 
If we assume that the overall cost is defined as 3$\cdot$\Time+\Inconveniences (which is a linear combination of the two), the optimal mechanism happens to be the ``Half-n-half" mechanism; this mechanism
sends out $3$ options in the first batch, and if no feasible time
slot is found then sends out the remaining $3$ options in the next
batch.

\begin{figure}[h!] \small
\centering
\includegraphics[scale=0.48]{plots/pareto_objective_fin.eps}
\caption{Pareto-frontier and objective function.}
\label{bdoodle:fig:Pareto_objective}
\end{figure}

In this chapter we will investigate the difficulty of finding an optimal B-Doodle mechanism, and to what degree it improves on the simple Doodle in realistic scenarios.
Under the assumption that the set of events in which an invitee is available for a particular option is mutually independent, 
we provide an efficient recursive algorithm for computing an optimal polling mechanism for a broad class of objective functions.

In addition we also  assume that the event organizer is given probability estimates on availability of agents, but it is not clear how one can obtain such probabilities. 
 Probability estimation is an interesting and challenging research question on its own, and we do not attempt to solve the question in this work. However, we provide several plausible methods for estimating the probabilities in the context of group scheduling, which can enable our model and algorithm to be deployed as a real-world application in the future.
 In the psychology literature, Mann et al. found that cultural differences between the Western, individualistic countries (such as the United States) and the Eastern, collectivistic countries (such as China and Japan) lead to different behaviors of respondents when it comes to a group-decision making process~\cite{mann1998cross}.
 More recently, Reinecke et al. analyzed more than 1.5 million Doodle date/time polls from 211 countries, and confirmed similar findings regarding time perception and group's behavior~\cite{reinecke2013doodle}. Among others, they found that ``in comparison to predominantly individualist societies, poll participants from collectivist countries respond earlier, agree to fewer options but find more consensus,'' which agrees with the findings of Mann et al.
 Besides the cultural differences, Doodle's own surveys on event scheduling found that people tend to respond to the scheduling surveys on Mondays, while Monday is the least popular day for having a meeting.~\footnote{http://en.blog.doodle.com/2012/05/23/mondays-for-planning-busy-weekends/}
 We believe that these studies and findings can be used to design a reasonable estimator for availability of agents, by utilizing the features that are known to be crucial -- such as demographics of the group and purpose of the event being scheduled.
 Recent work by Zou et al. analyzed over 340,000 Doodle polls data to study behavioral patterns of the users, and they were able to identify response functions that match the response patterns observed in the real data~\cite{zou2015strategic}. We believe that a similar approach can be taken to tackle the problem of probability estimation in the context of group scheduling.


\section{Notation and Definitions}
We mentioned two dimensions of optimality in the scheduling process: \Times and \Inconvenience.
\Times captures the duration of the scheduling process and \Inconveniences measures how much inconvenience is caused for each invitee during the scheduling process.
In this section we formally define the class of B-Doodle mechanisms and the Batched Doodle Problem (\BDP).

We consider a setting where there are $n$ invitees and $s$ date/time options from which the organizer can choose to propose. The organizer wishes to find a {\em feasible} date/time option that works for at least $\lceil f \cdot n \rceil$ invitees (we call $f\in [0,1]$ the {\em feasibility threshold}) or determines that there is no feasible option.
We model the uncertainty of availability of invitees as a random matrix as follows.
Consider a matrix of mutually independent Bernoulli random variables where rows represent invitees and columns represent a set of date/time options. 
Given probability of success for each Bernoulli random variable, the organizer can ``inspect'' a batch of columns at once to know of the realization of those entries. This corresponds to polling invitees' availability for a batch of date/time options. The organizer wishes to determine with certainty whether the matrix contains a feasible column or it does not. Using the random matrix model, we first investigate interesting properties of an optimal solution, and then discuss how it performs better than the classic Doodle under various settings.
By modeling the setting as a more abstract problem using random matrices, we can apply our model and algorithms to different settings, which we discuss in Section~\ref{bdoodle:sec:discussion}.

 \begin{definition}[Feasibility]
 Let $A$ be an $n$ by $s$ matrix whose entries are from $\{0,1\}$, and refer to the entry of $A$ at row $r$ and column $c$ as $a_{r,c}$.
 Rows represent the invitees, columns represent the options, and an entry being $1$ means the invitee is available for the option. 
 Let $f \in [0,1]$ be the {\em feasibility threshold}.
 We say that a column $c$ of $A$ is {\em $f$-feasible} if at least a fraction $f$ of the entries of the column are 1's.
 We say that $A$ is {\em $f$-feasible} if it contains at least one feasible column.
 We often simply write feasible by dropping $f$.
 \end{definition}
 
In our setting, the organizer iteratively sends out a batch of options until a feasible option is found. 
\Times spent by the scheduling process is measured by the number of iterations and \Inconveniences caused is measured by the number of options that have been sent out by the organizer.

Let us define a class of B-Doodle mechanisms that describe how the organizer sends out options in each iteration.
\begin{definition}[B-Doodle Mechanism]\label{bdoodle:def:BDoodleMechanism}
Let $S = \{1, 2, \dots, s\}$ be a set of $s$ options.
We define a B-Doodle mechanism for $S$ as an ordered partition of $S$.
Let $B = \langle S_1, S_2, \dots, S_m \rangle$ be a partition of $S$ into $m$ subsets such that $S_j \neq\emptyset$ for all $j$, $S_j \cap S_k = \emptyset$ for all $j \neq k$, and $\cup_{l=1}^{m} S_l = S$ where $1 \leq j,k \leq m$.
We define $b_j = |S_j|$ for all $j \leq m$ and call $b_j$ the size of the $j$-th batch.
We write $B_m$ to emphasize that $B$ has $m$ batches.
\end{definition}


We interpret a B-Doodle mechanism $B$ as follows: the organizer proposes the options contained in $S_1$ during the first iteration. If a feasible option is found, the process ends. Otherwise she sends out the next batch, $S_2$, and so on.
Note that there exist exponentially many B-Doodle mechanisms for any a given $S$ (exponential in cardinality of $S$).

The objective in the Batched Doodle Problem (\BDP) is to find an optimal B-Doodle mechanism with minimum expeted cost.
Earlier we considered a simple cost function that is a linear combination of the two.
While this cost function fits in many realistic situations, we want to explore a larger class of cost functions.
We define a class of cost functions that depend only on the number of iterations and batch sizes.
\begin{definition}[Cost function] \label{bdoodle:def:CostFunction}
	A cost function $c$ takes two integers $j$ and $b$ as arguments, and $c(j, b) > 0$ describes the aggregate cost of \Times and \Inconveniences that is incurred by sending out a batch containing $b$ options during the $j$-th iteration. 
	We assume that cost is additive so that the overall cost of executing the first $j$ iterations of ${B}_m = \langle S_1, S_2, \dots, S_m \rangle$ is simply the sum, $\sum_{k=1}^{j} c(k, b_k)$, which we denote by $C(j, {B}_m)$; recall that $b_k = |S_k|$.
A cost function $c$ is said to be $\theta$-\emph{simple} if there exists some constant $\theta>0$ such that $c(j, b) = \theta \cdot c(j-1, b)$ for all $j > 1$ and for all $b \geq 1$.
We often drop $\theta$ and just state that a cost function is \emph{simple}, for brevity.
\end{definition}
It is reasonable to assume that cost is additive (with respect to iterations) because an additional iteration adds the time spent and inconvenience caused for the invitees.
$\theta$-simple cost functions are not as restrictive as they appear.
For instance, when $\theta = 1$, it means that the cost of sending out a batch of a certain size is the same regardless of during which iteration it is being sent. When $\theta > 1$, it means that a later iteration costs more than an earlier iteration if the size of a batch is the same. Lastly, when $\theta \in (0, 1)$, it is the opposite in that an earlier iteration costs more than a later iteration given the same batch size -- this last case may not occur in real life, but our definition naturally allows for such scenarios. 

Note that $C(j, B_{m})$ is strictly increasing in $j$ for any fixed $B_{m}$ because $c(j, b) > 0$ for any $j$ and $b$. While the class of simple cost functions may seem too restrictive, we present a few natural choices of cost functions that belong to the class of simple cost functions. In this work we assume that the underlying cost function is simple; further we assume that $c(j, b)$ for all $1 \leq j, b \leq s$ can be computed in polynomial time with respect to $s$.
\begin{example}[Three simple cost functions] \label{bdoodle:exmp:CostFunctions}
	
	The simplest choice of a cost function is a linear combination of \Times and \Inconvenience, parameterized by some constant $\alpha > 0$; we call this a \emph{linear} cost function. This cost function is $1$-simple. Notice that the first term captures \Times and the second captures \Inconveniences in Equation~\ref{bdoodle:eqn:linear_cost_function}.
	\begin{equation} \label{bdoodle:eqn:linear_cost_function}
		c_{\alpha}(j, b_j) = \alpha + b_j,
		~~	C_{\alpha}(j, {B}_m) = \left(\alpha \cdot j\right) + \left(\sum_{k=1}^{j} b_k\right).
	\end{equation}
	In some cases the organizer may want to penalize the mechanism for executing too many iterations by making later iterations to cost more than earlier iterations. The following describes such cost function, parameterized by some constant $\beta > 1$; we call this a \emph{time-averse} cost function. Notice that the term $\beta^j$ is strictly increasing as $j$ increases (recall $\beta > 1$). This cost function is $\beta$-simple.
	\begin{equation} \label{bdoodle:eqn:time_averse_cost_function}
	c_{\beta}(j, b_j) = \beta^j \cdot b_j,
	~~ C_{\beta}(j, {B}_m) = \left(\sum_{k=1}^{j} \beta^k \cdot b_k \right).
	\end{equation}
	In contrast if the organizer is interested in reducing the size of each batch because it may cause too much inconvenience, then the following cost function could be used, which is parameterized by some constant $\gamma > 1$; we call this an \emph{inconvenience-averse} cost function which is $1$-simple.
	\begin{equation} \label{bdoodle:eqn:inconvenience_averse_cost_function}
		c_{\gamma}(j, b_j) = \gamma^{b_j},
	~~ C_{\gamma}(j, {B}_m) = \left(\sum_{k=1}^{j} \gamma^{b_k} \right).
	\end{equation}
\end{example}

Let us now formally define the Batched Doodle Problem (\BDP).
\begin{definition}[Batched Doodle Problem] \label{bdoodle:def:Problem}
	An instance of the Batched Doodle Problem (\BDP) is a tuple $(A, P_A, f, c)$
	where $A$ is a matrix of Bernoulli random variables,
	$P_A$ is the probability matrix associated with it,
	$f$ is the feasibility threshold with $f \in [0,1]$,
	and $c$ is a cost function.
	An entry $p_{r,c}$ of $P_A$ represents the probability that $a_{r,c}$ is equal to $1$ (we assume $p_{r,c}\in (0,1)$).
	The objective in \BDPs is to find an optimal B-Doodle mechanism $B^*$ such that $B^*$ minimizes the expected cost of the scheduling process (expectation with respect to $P_A$), given $(A, P_A, f, c)$.
\end{definition}

Since there are exponentially many B-Doodle mechanisms, it is impractical to enumerate each B-Doodle and compute its expected cost in order to find an optimal one. In what follows we make some simplifying assumptions that are assumed throughout this paper, and we present an efficient algorithm to solve \BDPs in next section.

% TODO: To change 'invitees' to 'rows' and 'options' to 'columns' and 'availability' to 'entries in A'

\section{Efficient Algorithm for Finding Optimal B-Doodle}
In this section, we discuss a series of useful lemmas, which eventually lead to an efficient algorithm for finding an optimal B-Doodle mechanism in many settings. 

\subsection{Preliminaries}
	Intuitively, if the organizer knows the probability distribution of availability of invitees, then he should inspect the columns that are more likely to be feasible first. We will prove this claim using the following lemma.
	\begin{lemma} \label{bdoodle:lemma:q_t_polytime}
	Given $(A, P_A, f, c)$, let $q_t$ for each column $t$ be the probability that column $t$ is $f$-feasible.
	We can compute $q_t$ in polynomial time.
	\end{lemma}
	\begin{proof}
	For any fixed $t$, let us define $V_t(i, z)$ to be the probability that among the column $t$ of $A$, exactly $z$ entries out of the first $i$ entries (in their row indices) are 1's. $V_t(i, z)$ is well-defined where $1 \leq i \leq n$ and $0 \leq z \leq i$. We further define $V_t(i, z)$ for the following degenerate cases:
	\begin{equation*}
	V_t(i, z) =
	\begin{cases}
		1 & \mbox{if~} i = z = 0 \\
		0 & \mbox{if~} z = -1 \mbox{~or~} (i = 0 \land z > 0)
	\end{cases}
	\end{equation*}
	Then we can compute $V_t(i, z)$ using a simple dynamic programming algorithm according to the following recurrence relation where $1 \leq i \leq n$ and $0 \leq z \leq i$:
	\begin{equation} \label{bdoodle:eqn:vt_precompute}
	V_t(i, z) = V_t(i-1, z-1) \cdot p_{i,t} + V_t(i-1, z) \cdot (1- p_{i,t})
	\end{equation}
	It is easy to verify that the recurrence relation is correct.
	For the degenerate cases (or base cases) when $i = z = 0$, no entries (out of $0$ entries) in column $t$ are equal to 1, and thus $V_t(i, z) = 1$ in this case. When $(z = -1)$ or $(i=0 \land z>0)$, $V_t(i, z) = 0$ because it is impossible to have $z$ entries out of $i$ entries.

	For the main recurrence relation, the entry $a_{r, t}$ is either 1 or 0, and we compute $V_t(i, z)$ by considering both cases. For each fixed $t$, there are $O(n^2)$ entries of $V_t$ to be computed, and computing each entry takes $O(1)$; therefore it takes $O(sn^2)$ time to compute $V_t(i, z)$ for all $t, i, z$.

	Finally we can compute $q_t$ after we compute $V_t(i, z)$ for all $i,z$, as follows:
	
	$$q_t = \sum_{z = \lceil f \cdot n \rceil}^{n} V_t(n, z).$$
	\end{proof}

	Lemma~\ref{bdoodle:lemma:q_t_polytime} allows us to describe a B-Doodle mechanism in a simpler form, along with the following theorem because an optimal B-Doodle mechanism must inspect the columns in non-increasing order of $q_t$.
	\begin{theorem} \label{bdoodle:thm:exchange_argument}
	Consider a B-Doodle mechanism $B$ described by $\langle S_1, S_2, \dots, S_m \rangle$. Suppose that there is some $t \in S_l$ and $t'\in S_{l'}$ with $q_t < q_{t'}$ and $l < l'$. Then $B$ is not an optimal in the sense that we can find another mechanism $B^*$ whose expected cost is strictly less than the expected cost of $B$.
	\end{theorem}
	\begin{proof}
	Given $B$, let us construct $B^* = \langle S^*_1, S^*_2, \dots, S^*_m \rangle$ that is the same as $B$ except that we swap $t$ and $t'$. That is, $S^*_i = S_i$ for all $i \neq l$ and $i\neq l'$, and $S^*_{l} = (S_l \cup t') \setminus t$ and $S^*_{l'} = (S_{l'} \cup t) \setminus t'$. Note that the two mechanisms have the same batch sizes ($b_j = b^*_j$ for all $j$). 

	Let $w_j$ be the probability that the scheduling process ends after the $j$-th iteration if we use $B$, and $w_j^*$ if we use $B^*$. Then $w_j = w_j^*$ for all $j < l$ because $S_j = S^*_j$ for all $j$ with $1 \leq j < l$. Clearly it holds that $w_l < w_l^*$ because of the swap of $t$ and $t'$. For $j$ with $l < j \leq l'$, it holds that $w_j > w_j^*$; this is not obvious, but the intuition is that if $w_l^*$ increases then the subsequent $w_j^*$'s with $l < j \leq l'$ must decrease as they depend on $1-w_l^*$, until this effect is canceled out in the $l'$-th batch. Since the probability of finding no feasible columns during $l'$ iterations is the same in both cases (because $\cup_{i=1}^{l'} S_i = \cup_{i=1}^{l'}S^*_i$) it holds that $w_j = w_j^*$ for all $j > l'$. 

	Let $E_B$ be the expected cost of $B$ and $E_{B^*}$ of $B^*$. Since $b_j = b^*_j$ for all $j$, it holds that $C(j, B) = C(j, B^*)$ for all $j$, and we have:
	\begin{eqnarray}
		E_B - E_{B^*}
		&=& (w_l - w_l^*)C(l, B) + \sum_{j=l+1}^{l'} (w_j - w_j^*)C(j, B) \\
		&=& \sum_{j=l+1}^{l'} (w_j - w_j^*)(C(j, B) - C(l, B))
	\end{eqnarray}
	The first inequality holds by definition of the expected cost and canceling out some terms; the second inequality holds because $(w_l - w_l^*) + \sum_{j=l+1}^{l'} (w_j - w_j^*) = 0$ (as probabilities must add up to one). Since $w_j > w_j^*$ and $C(j, B) > C(l, B)$ for all $j$ (because $C(j, B)$ is an increasing function in $j$ for fixed $B$), we conclude that $E_{B} > E_{B^*}$.
	\end{proof}
	Theorem~\ref{bdoodle:thm:exchange_argument} implies that an optimal B-Doodle that minimizes the expected cost must have the following property: for any two batches $S_j$ and $S_k$ with $j < k$, it must hold that $q_t \geq q_{t'}$ for all $t\in S_j$ and $t' \in S_k$ (otherwise we can apply the theorem and swap the two options to obtain a mechanism with smaller expected cost).
	Therefore we can limit our attention to the sub-class of B-Doodle mechanisms that only describe batch sizes, but not explicitly which options in each batch. Due to Lemma~\ref{bdoodle:lemma:q_t_polytime} we can compute $q_t$ for each $t$ in polynomial time, and thus we can simply sort options by $q_t$ in non-increasing order as a pre-processing step.

	Therefore we can focus on the following sub-class of \emph{simplified} B-Doodle mechanisms when we seek an optimal B-Doodle mechanism.
	\begin{definition}[Simplified B-Doodle Mechanism]
	Let $S = \{1, 2, \dots, s\}$ be a set of $s$ options.
	A simplified B-Doodle mechanism $B$ for $S$ is a vector of integers, described as $B = \langle b_1, b_2, \dots, b_m\rangle$ where $(b_j \geq 1$ for all $j \leq m)$ and $(\sum_{k=1}^{m} b_k = s)$.
	We write $B_m$ to emphasize that $B$ has $m$ batches. It is assumed that $B_m$ partitions $S$ into $m$ batches according to $b_j$'s where options (or columns) are sorted by $q_t$ in non-increasing order.
	\end{definition}
	From now on we use the definition of a simplified B-Doodle, and simply write a B-Doodle mechanism to mean a simplified B-Doodle mechanism but this should cause no confusion.


\subsection{Optimal Algorithm} \label{bdoodle:sec:Algorithm}

	In this section we present an algorithm that finds an optimal B-Doodle mechanism that minimizes the expected cost, given an instance of \BDP.
	Let us first describe how one can express the expected cost of a B-Doodle mechanism.

	Given an instance $(A, P_A, f, c)$, consider some B-Doodle mechanism $B_{m} = \langle b_1, b_2, \dots, b_m \rangle$ with $\sum_{k=1}^{m} b_k = |S|$. Let $\Pr(j)$ be the probability that the scheduling process ends after $j$-th iteration. Let us denote the expected cost of $B_{m}$ by $\Exp_c[B_{m}]$, which can be expressed in terms of $\Pr(\cdot)$ and the cost function $c$:
	\begin{equation} \label{bdoodle:eqn:expCost}
		\Exp_c[B_{m}] = \sum_{j=1}^{m} \Pr(j) \cdot C(j, B_{m})  =  \sum_{j=1}^{m} \Pr(j) \cdot \left(\sum_{k=1}^{j} c(k, b_k)\right)
	\end{equation}%

	While one can compute $\Pr(j)$ in polynomial time, it is not necessary for our algorithm. Instead we present an important lemma that leads us to an efficient algorithm. The following lemma states that if $c$ is simple, then we can compute $\Exp_c[B_{m}]$ in a recursive manner.
	\begin{lemma} \label{bdoodle:lemma:recurrence}
		Consider any mechanism $B_m = \langle b_1, b_2, \dots, b_m \rangle$. Let us denote another mechanism that is obtained after removing the first batch ($b_1$) from $B_m$, as $\hat{B}_{m-1}$ (i.e. $\hat{B}_{m-1} = \langle b_2, b_3, \dots, b_m \rangle$). For each option $t \in S$, let $q_t$ be the probability that $t$ is $f$-feasible.

	If $c$ is a $\theta$-simple cost function with with $\theta > 0$, the following equality holds:
	\begin{equation} \label{bdoodle:eqn:recurrence}
	\Exp_c[B_{m}] = c(1, b_1) +  \left(\prod_{t=1}^{b_1} \left(1 - q_t\right)\right) \cdot \theta \cdot \Exp_c[\hat{B}_{m-1}]
	\end{equation}
	\end{lemma}
	\begin{proof}
	Let us first provide an intuitive way to understand the recurrence relation given by Equation~\ref{bdoodle:eqn:recurrence}. The first term $c(1, b_1)$ captures the cost of sending out the first batch; regardless of when the scheduling process ends, this cost incurs with probability of $1$. If it turns out that the first batch does not contain a feasible option with probability of $(1 - \Pr(1))$, then the organizer must send out the remaining batches, which is precisely $\hat{B}_{m-1}$. It is easy to verify that the product term in the equation is equal to $(1 - \Pr(1))$. The expected cost of $\hat{B}_{m-1}$ is adjusted by a factor of $\theta$ in the equation because the cost function is $\theta$-simple. 

	We now formally prove the claim. Given $B_{m}$ as described above, for each batch $j \in \{1, 2, \dots, m\}$, let $r_j$ be the probability that the $j$-th batch has at least one $f$-feasible option. Let us define $v_0 = 0$ and $v_j = \sum_{k=1}^{j} b_k$ (i.e. $v_j$ is the number of options in batches $1$ through $j$).
	We can express $r_j$ as:
	\begin{equation} \label{bdoodle:eqn:r_j}
	r_j = 1 - \prod_{t=v_{j-1} + 1}^{v_j} (1 - q_t)
	\end{equation}
	$\Pr(j)$ is the probability that the scheduling process ends after $j$-th iteration (with $\Pr(m) = 1 - \sum_{k=1}^{m-1} \Pr(k)$). For $1 \leq j < m$, $\Pr(j)$ is given by:
	\begin{equation} \label{bdoodle:eqn:Pr_j}
	\Pr(j) =  r_j  \prod_{k=1}^{j-1} (1 - r_k)
	\end{equation}
	It is understood that $\prod_{k=x}^{y} (\cdot)$ is equal to $1$ when $x > y$. 

	Recall that $\hat{B}_{m-1} = \langle b_2, b_3, \dots, b_{m} \rangle$.
	Not to be confused, let us express it as $\hat{B}_{m-1} = \langle \hat{b}_1, \hat{b}_2, \dots, \hat{b}_{m-1}\rangle$ with $\hat{b}_j = b_{j+1}$ for all $j\geq 1$. Let $\hat{r}_j$ be the probability that a feasible option exists in $j$-th batch of $\hat{B}_{m-1}$, which is equal to $r_{j+1}$ for all $j \geq 1$. Let $\hat{Pr}(j)$ be the probability that scheduling ends after the $j$-th iteration of $\hat{B}_{m-1}$:
	\begin{equation}
		\hat{Pr}(j)
		= \hat{r}_j \prod_{k=1}^{j-1} (1 - \hat{r}_k)
		= r_{j+1} \prod_{k=2}^{j} (1 - r_{k})  \label{bdoodle:eqn:pr_hat_b}
	\end{equation}
	We can express $\Exp_c[\hat{B}_{m-1}]$ as follows:
	\begin{eqnarray*}
	\Exp_c[\hat{B}_{m-1}]
	&=& \sum_{j=1}^{m-1} \hat{Pr}(j) \left( \sum_{k=1}^{j} c(k, \hat{b}_k) \right) \\
	&=& \sum_{j=1}^{m-1} r_{j+1} \left(\prod_{k=2}^{j} (1 - r_{k})\right) \left(\sum_{k=2}^{j+1} c(k-1, b_{k+1})\right) \\
	&=& \sum_{j=2}^{m} r_{j} \left(\prod_{k=2}^{j-1} (1 - r_{k})\right) \left(\sum_{k=2}^{j} c(k-1, b_{k+1})\right)
	\end{eqnarray*}
	The first equality holds by definition. The second equality is due to Equation~\ref{bdoodle:eqn:pr_hat_b} and because $\hat{b}_k = b_{k+1}$. The third equality is by changing the range of $j$ in the summation. 

	If we multiply $\Exp_c[\hat{B}_{m-1}]$ by $(1 - r_1)\theta$, we get the following:
	\begin{equation} \label{bdoodle:eqn:ExpCost_Bhat}
	(1-r_1)\theta \Exp_c[\hat{B}_{m-1}] = \sum_{j=2}^{m} r_{j} \left(\prod_{k=1}^{j-1} (1 - r_{k})\right) \left(\sum_{k=2}^{j} c(k, b_{k+1})\right)
	\end{equation}
	Notice that the product term now runs from $k = 1$ to $j-1$ as we multiply by $(1-r_1)$ and the inner-most summation has $c(k, b_{k+1})$ as we multiply by $\theta$ (recall that $c$ is $\theta$-simple).

	Finally we can express $\Exp_c[B_m]$ in terms of $\Exp_c[\hat{B}_{m-1}]$ as follows (where $c_1 = c(1, b_1)$ for brevity):
	\begin{eqnarray*}
	\Exp_c[B_{m}]
	&=& \sum_{j=1}^{m} \Pr(j) \left(\sum_{k=1}^{j} c(k, b_k)\right) \\
	&=& c_1 + \left(\sum_{j=2}^{m} r_j \left(\prod_{k=1}^{j-1} (1 - r_k)\right) \left(\sum_{k=2}^{j} c(k, b_k)\right) \right) \\
	&=& c_1 + (1-r_1)\theta \Exp_c[\hat{B}_{m-1}] \\
	&=& c(1, b_1) + \left(\prod_{t=1}^{b_1} (1-q_t) \right)\cdot \theta \cdot \Exp_c[\hat{B}_{m-1}]
	\end{eqnarray*}
	The first equality holds by definition of $\Exp_c[B_{m}]$. The second equality is obtained by taking $c(1, b_1)$ out from the summation (and note that $\Pr(j)$ adds up to 1) first and then applying Equation~\ref{bdoodle:eqn:Pr_j}. The last two inequalities hold due to Equation~\ref{bdoodle:eqn:ExpCost_Bhat} and \ref{bdoodle:eqn:r_j}, respectively.
	The last expression exactly  matches Equation~\ref{bdoodle:eqn:recurrence} in the lemma.
	\end{proof}

	Using the recurrence relation in Lemma~\ref{bdoodle:lemma:recurrence} and the computing method in Lemma~\ref{bdoodle:lemma:q_t_polytime}, we can now design an efficient recursive algorithm finds the optimal B-Doodle mechanism. Our algorithm is presented in Algorithm~\ref{bdoodle:alg:recursive} as a recursive method $Rec(x)$. We assume that the values of $q_t$ have been computed as a pre-processing step prior to running our algorithm, using the method in Lemma~\ref{bdoodle:lemma:q_t_polytime}. We further assume that options are sorted in decreasing order of $q_t$ (i.e. $q_1 \geq q_2 \geq \dots \geq q_{s}$); therefore we simply refer to the option by its index (i.e. $t = 3$ refers to the third option in the sorted list of options).

	Given $1 \leq x \leq s$, $Rec(x)$ returns optimal B-Doodle ($B^*$) that consists of options $\{x, x+1, \dots, s\}$ and its expected cost ($EC^*$). To solve for a given instance of the problem, we simply call the method $Rec(x)$ with $x = 1$.

	\begin{algorithm}
	\caption{Recursive Algorithm: $Rec(x)$}
	\label{bdoodle:alg:recursive}
	\begin{algorithmic}[1]
		\State ${B}^* \gets \langle s-x+1 \rangle$, $EC^* \gets c(1, s-x+1)$
		\For{$b := 1, 2, \dots, s-x$}
			\State $({B}, EC) \gets Rec( x+b )$
			\State $EC_b \gets c(1, b) + \left(\prod_{t=x}^{x+b-1} (1 - q_t)\right)\cdot \theta \cdot EC$
			\If {$EC^* > EC_b$}
				\State $EC^* \gets EC_b$
				\State $B^* \gets \langle b, {B} \rangle$
			\EndIf
		\EndFor
		\State \textbf{return} $({B}^*, EC^*)$
	\end{algorithmic}
	\end{algorithm}

	\begin{theorem} \label{bdoodle:thm:recursive_algo}
	Algorithm~\ref{bdoodle:alg:recursive} runs in polynomial time, and returns an optimal B-Doodle mechanism that minimizes the expected cost, given an instance $(N, S, f, c, P)$ of \BDPs when $c$ is $\theta$-simple.
	\end{theorem}
	\begin{proof}
	Our recursive method is very simple: given options $\{x, x+1, \dots, s\}$, it iteratively considers the case of sending out $b$ options in the first iteration, and computes the expected cost $EC_b$ of doing so, for each $b$ with $1 \leq b \leq s-x+1$.  In line 1, our method checks for the trivial case when $b = s-x+1$ (i.e. all options are sent in a single batch); we store this mechanism in $B^*$ and its expected cost in $EC^*$. Then we iterate $b$ from $1$ to $s-x$ and compute the expected cost $EC_b$, and compare with the optimal expected cost found so far (lines 2-9). For each $b$, we first recursively compute the expected cost for the remaining options (namely, $\{x+b, x+b+1, \dots, s\}$) by calling our method $Rec(x+b)$, and store the expected cost in $EC$ (line 3). Using Lemma~\ref{bdoodle:lemma:recurrence} we can compute $EC_b$ as in line 4 (note that the first batch contains $b$ options from $x$ to $x+b-1$). In lines 5-8, we simply compare $EC_b$ with the current optimal, $EC^*$, and update if necessary; $\langle b, B \rangle$ should be interpreted as the concatenation of $b$ and $B$ into a single vector of integers. Finally in line 10, our method returns the optimal B-Doodle found, $B^*$ and its expected cost $EC^*$. 

	We prove correctness of the algorithm by induction on $x$ (from $x = s$ to $x = 1$). The base case is trivial: if $x = s$ the only B-Doodle mechanism is $\langle 1 \rangle$ and our method finds it in line $1$ and returns it in line $10$. Suppose our method correctly returns the optimal B-Doodle mechanism (and its expected cost) for all $x > k$ for some $k$ (the inductive hypothesis), and we prove for the case of $x = k$. When $x = k$, there are precisely $(s-k+1)$ options that are to be sent, and the first batch can contain any number of options between $1$ and $(s-k+1)$, inclusive. In lines 1-9 our method considers all such cases, and for each case it computes the expected cost correctly (in line 4) due to our inductive hypothesis and Lemma~\ref{bdoodle:lemma:recurrence}. Therefore our method finds the optimal B-Doodle for all $x$ with $1 \leq x \leq s$; in particular when $x = 1$, it returns the desired optimal B-Doodle mechanism for the given problem instance.

	Our recursive method runs in polynomial time when we use memoization on $Rec(x)$ with respect to $x$; that is, for each $x$ we cache what $Rec(x)$ returns for the first time, and for any subsequence calls to $Rec(x)$ we simply return the cached values. Therefore lines 1-9 are executed at most once for each $x$ with $1 \leq x \leq s$. Also note that $Rec(x)$ makes a call to $Rec(x + b)$ with $b \geq 1$ and $x+b \leq s$, which means there are no infinite loops. Once $Rec(x)$ is computed for all $x > k$, it takes $O(|S|^2)$ time to compute $Rec(k)$, and thus the overall running time of our algorithm is $O(|S|^3)$ when we start with $Rec(1)$. Pre-processing steps (of computing $q_t$) also run in polynomial time due to Lemma~\ref{bdoodle:lemma:q_t_polytime}.
	\end{proof}





\section{Experimental Results with Synthetic Data}

We showed that we can find an optimal B-Doodle mechanism that minimizes the expected cost. But a very important question remains: Is our optimal B-Doodle substantially better than Doodle in realistic settings? If the answer is no, we do not have much incentive to discard the simplest mechanism, Doodle, over using our sophisticated algorithm. Intuitively we expect Doodle to be inefficient if there are many options (i.e., $s$ is large) because it causes much inconvenience for invitees to examine the options. Reasoning further, we can also see that inefficiency of Doodle depends on the probability of each option being feasible (i.e. $q_t$ values), which in turn depends on $p_{i, t}$ values. If invitees are relatively free, then a small number of options would be sufficient for finding a feasible option, which makes Doodle inefficient. Last but not least, the underlying cost function plays an important role as well. These all sound plausible, and we validate our intuition with simulation results.

In our setting there are many experimental choices one can choose from. In what follows we look at a representative example in which each invitee is available for each option with the same probability of $p$; that is, $p_{i, t} = p$ for some constant $p$. (While in realistic situations the $p_{i, t}$'s may all differ, we  note that in our simulations  our results seem to carry over to these more general settings as well.) We present experimental results for each of the three cost functions discussed in Example~\ref{bdoodle:exmp:CostFunctions}: The linear cost function, the time-averse cost function, and the inconvenience-averse function.

% ===================== Linear Cost Function (Below)

\subsection{Case of Linear Cost Function} \label{bdoodle:sec:result_linear_cost}
Recall that the linear cost function is parameterized by some constant $\alpha>0$ and that $c_{\alpha}(j, b) = \alpha + b$. We used $\alpha = 2$ as an experimental choice, and we later discuss what we observed for different values of $\alpha$.

We reasoned that Doodle becomes suboptimal if the number of options, $s$, is large. Therefore it is interesting to know for what ranges of $s$, Doodle is suboptimal. For some fixed $(n, s, p)$, let $C^D(n, s, p)$ be the cost of Doodle and $C^*(n, s, p)$ be the expected cost of $B^*$. We want to find the smallest critical point $S^*(n, p)$ such that for any $s \geq S^*(n, p)$ we have $C^D(n, s, p) > C^*(n, s, p)$.

In Table~\ref{bdoodle:table:DoodleSuboptimal}, we show $S^*$ for various $(n, p)$ when $f = 1$ (i.e. the organizer requires everyone be available). For instance we find $S^*(6, .8) = 5$, which implies that Doodle is suboptimal for all $s \geq 5$ given that $n=6$ and $p=.8$. The smaller $S^*$ is, the less practical Doodle is for the corresponding $(n, p)$.
Across the table we can observe that $S^*$ is small when $n$ is small and/or when $p$ is high -- this agrees with our intuition.
We highlighted $8$ entries in boldface to emphasize that $S^* \leq 15$; for the corresponding $(n, p)$ values, Doodle is suboptimal if there are $15$ or more options being considered.
\begin{table}[h!]  % \small
\centering
\begin{tabular}{|c|c|c|c|c|c|}
	\hline
	$S^*$ & $n = 2$ & $n = 4$ & $n = 6$ & $ n = 10 $ & $n = 15$ \\ \hline
	$p = .8$ & \textbf{3} & \textbf{4} & \textbf{5} & \textbf{8} & \textbf{15} \\ \hline
	$p = .5$ & \textbf{5} & \textbf{11} & 22 & 90 & $>300$\\ \hline
	$p = .2$ & \textbf{14} & 70 & $>300$ & $>300$ & $>300$ \\ \hline
\end{tabular}
\caption{Critical point $S^*$ for various $(n, p)$ when $f = 1, \alpha = 2$.
Entries in boldface emphasize that $S^* \leq 15$.
} \label{bdoodle:table:DoodleSuboptimal}
\end{table}

Not only do we want to know when Doodle is suboptimal, but we also want to know how inefficient Doodle is in realistic situations. For some fixed $(n, s, p)$, we define the efficiency of Doodle, $e_{D}(n,s,p)$, as the ratio of the optimal expected cost to the cost of Doodle: $e_{D}(n,s,p) = C^*(n, s, p) / C^D(n, s, p)$.

In Table~\ref{bdoodle:table:DoodleEfficiency} we show $e_{D}$ for the same settings of $(n, p)$ we used in Table~\ref{bdoodle:table:DoodleSuboptimal}. For this experiment we used $s = 15$ and $f = 1$. If $e_{D} = 1$ Doodle is optimal, and if $e_{D}$ is close to zero then Doodle is very inefficient. We find that $e_{D}(2, .8) = .270$, which implies that Doodle is very inefficient in this case; on the other hand $e_{D}(10, .5) = 1$, in which case Doodle is optimal. Across the table we observe the same pattern we observed before -- for small $n$ and high $p$, Doodle is substantially inefficient.
\begin{table}[h]  % \small
\centering
\begin{tabular}{|c|c|c|c|c|c|c|}
	\hline
	$e_{D}$ & $n = 2$ & $n = 4$ & $n = 6$ & $ n = 10 $ & $n = 15$ \\ \hline
	$p = .8$ & \textbf{.270} & \textbf{.361} & \textbf{.486} & .777 & .986 \\ \hline
	$p = .5$ & \textbf{.502} & .904 & 1 & 1 & 1  \\ \hline
	$p = .2$ & .970 & 1 & 1 & 1 & 1\\ \hline
\end{tabular}
\caption{Efficiency of Doodle $e_{D}$ for various $(n, p)$ when $f = 1, \alpha=2, s = 15$.
Entries in boldface emphasize that $e_{D} < .750$.
} \label{bdoodle:table:DoodleEfficiency}
\end{table}

Inefficiency of Doodle is more pronounced when the organizer has a lower feasibility threshold such as $f = .7$; in such case, only a fraction of invitees need to be available.
We can clearly observe the worsened inefficiency of Doodle in Table~\ref{bdoodle:table:DoodleEfficiency-lower-attendance}.
Here we show $e_{D}$ values for the same set of $(n, p)$ as in Table~\ref{bdoodle:table:DoodleEfficiency} but with $f = .7$.
Previously we highlighted four entries with $e_{D} < .750$ when $f = 1$ in Table~~\ref{bdoodle:table:DoodleEfficiency}; when $f = .7$ the number of entires with $e_{D} < .750$ doubled to eight.
Note that the first column is identical between the two tables; this is because $f = .7$ still requires both invitees to be available when $n = 2$.
Also notice that $e_{D}$ is surprisingly low in the first row of Table~\ref{bdoodle:table:DoodleEfficiency-lower-attendance} across all columns. This shows that regardless of the number of invitees, Doodle is significantly inefficient when the invitees are highly available ($p \geq .8$) and the organizer has a relaxed feasibility threshold.
\begin{table}[h]  % \small
\centering
\begin{tabular}{|c|c|c|c|c|c|c|}
	\hline
	$e_{D}$ & $n = 2$ & $n = 4$ & $n = 6$ & $ n = 10 $ & $n = 15$ \\ \hline
	$p = .8$ & \textbf{.270} & \textbf{.215} & \textbf{.267} & \textbf{.201} & \textbf{.211} \\ \hline
	$p = .5$ & \textbf{.502} & \textbf{.434} & .772 & \textbf{.628} & .913  \\ \hline
	$p = .2$ & .970 & 1 & 1 & 1 & 1\\ \hline
\end{tabular}
\caption{Efficiency of Doodle $e_{D}$ for various $(n, p)$ when $f = .7, \alpha=2, s = 15$.
Entries in boldface emphasize that $e_{D} < .750$.
} \label{bdoodle:table:DoodleEfficiency-lower-attendance}
\end{table}

We ran experiments with different values of $(n, s, p, f, \alpha)$, and observed the same  trends that are presented here.


% ===================== Other Cost Functions (Below)

%\balancecolumns
\subsection{Case of Other Cost Functions}
We present some of the experimental results we obtained with different cost functions, namely the time-averse cost function and the inconvenience-averse cost function. As the names suggest, the former places more weight on optimizing \Time, while the latter on optimizing \Inconvenience. 

For the time-averse cost function (recall $c_{\beta}(j, b) = \beta^j \cdot b$), we chose $\beta = 2$ as an experimental choice. Notice that the cost increases exponentially for later iterations, which forces the organizer to send out a small number of batches (as in Doodle). We ran the same set of experiments as before to measure efficient of Doodle, $e_{D}$.

The result is summarized in Table~\ref{bdoodle:table:DoodleEfficiency-lower-attendance_time_averse}, which shows the same trends as we observed in previous experiments. Notice that in the first row ($p = .8$), as $n$ increases we expect $e_{D}$ to decrease, but $e_{D}$ fluctuates while decreasing in general. The fluctuation is due to the rounding of attendance requirement ($\lceil a \cdot n \rceil$). For instance when $n = 4$, $\lceil a \cdot n \rceil$ = 3 which effectively requires 75\% of attendees be available.
\begin{table}[h]  % \small
\centering
\begin{tabular}{|c|c|c|c|c|c|c|}
	\hline
	$e_{D}$ & $n = 2$ & $n = 4$ & $n = 6$ & $ n = 10 $ & $n = 15$ \\ \hline
	$p = .8$ & \textbf{.180} & \textbf{.104} & \textbf{.175} & \textbf{.088} & \textbf{.099} \\ \hline
	$p = .5$ & \textbf{.561} & \textbf{.457} & .876 & \textbf{.726} & .987  \\ \hline
	$p = .2$ & 1 & 1 & 1 & 1 & 1\\ \hline
\end{tabular}
\caption{Efficiency of Doodle $e_{D}$ for various $(n, p)$ when $f = .7, \beta=2, s = 15$.
Entries in boldface emphasize that $e_{D} < .750$.
} \label{bdoodle:table:DoodleEfficiency-lower-attendance_time_averse}
\end{table}

For the inconvenience-averse cost function (recall $c_{\gamma}(j, b) = \gamma^{b}$), we chose $\gamma = 1.1$ as an experimental choice. While $\gamma$ is fairly small, because the cost function is an exponential function of $b$, an optimal B-Doodle must send out small-size batches. This cost function optimizes \Inconveniences primarily. We ran the same set of experiments as before to measure efficiency of Doodle, $e_{D}$.

The result is summarized in Table~\ref{bdoodle:table:DoodleEfficiency-lower-attendance_inconvenience_averse}. Notice that $e_{D}$ is not equal to $1$ even when $p = .2$ in this setting, which we did not observe with the other cost functions previously. Due to integer-rounding, we again observe some fluctuations in $e_{D}$ across columns within the same row, but the general trend is that $e_{D}$ decreases as $n$ increases.
\begin{table}[h]  % \small
\centering
\begin{tabular}{|c|c|c|c|c|c|c|}
	\hline
	$e_{D}$ & $n = 2$ & $n = 4$ & $n = 6$ & $ n = 10 $ & $n = 15$ \\ \hline
	$p = .8$ & \textbf{.333} & \textbf{.299} & \textbf{.329} & \textbf{.294} & \textbf{.298} \\ \hline
	$p = .5$ & \textbf{.499} & \textbf{.450} & \textbf{.695} & \textbf{.592} & .799  \\ \hline
	$p = .2$ & .850 & .887 & .974 & .976 & .980 \\ \hline
\end{tabular}
\caption{Efficiency of Doodle $e_{D}$ for various $(n, p)$ when $f = .7, \gamma=1.1, s = 15$.
Entries in boldface emphasize that $e_{D} < .750$.
} \label{bdoodle:table:DoodleEfficiency-lower-attendance_inconvenience_averse}
\end{table}

%\balancecolumns

\subsection{Summary of Experiments}
While we only presented experimental results with specific values of $(n, s, p, f)$ and parameters $(\alpha, \beta, \gamma)$, we observed that Doodle is in general substantially inefficient, including (but not limited to) when one or more of the following conditions hold:
\begin{itemize}
	\item There is a relatively small number of invitees ($n \leq 10$).
	\item There is a large number of options ($s \geq 15$).
	\item invitees are relatively free ($p > .5$).
	\item Feasibility threshold is relaxed ($f < .8$).
	\item The cost function places more weight on \Inconveniences than \Times ($\alpha < 20$, $\beta < 5$, or $\gamma > 1.05$).
\end{itemize}
Intuitively the first four conditions affect $q_t$ (the probability that option $t$ is feasible) in the same way, and if $q_t$'s are high then Doodle is more likely to be inefficient because a few options may be sufficient for finding a feasible one. While the last condition is independent of $q_t$, the cost function determines what is being optimized, and Doodle becomes more inefficient if $c$ favors reducing \Inconveniences over reducing \Time.


\section{Discussion}
\label{bdoodle:sec:discussion}

In this chapter we identified two important dimensions of optimality in
group scheduling: \Times and \Inconvenience. We generalized the popular
Doodle mechanism to a class of B-Doodle mechanisms that partition date/time
options into batches. We showed an example of the Pareto-frontier of
B-Doodle mechanisms on the \Time-\Inconveniences dimensions. We then
described an efficient algorithm for finding an optimal B-Doodle
mechanisms, given a simple cost function that aggregates \Times and
\Inconvenience, assuming probabilistic independence among availability
of invitees. We showed in simulations that optimal B-Doodle mechanism
is superior to Doodle in realistic situations, sometimes greatly so.


% TODO: Move to next chapter?
% While useful in and of itself, this work leaves open many questions.
% Here we focused on a sub-class of the Single-proposer mechanisms in
% which there is a sole organizer who is trying to find an agreeable
% schedule. In realistic situations a group of invitees often negotiate
% among themselves or respond with counter-offers. It will be
% interesting to extend our B-Doodle mechanisms to the Multi-proposer
% mechanisms. Another obvious extension of our work is to relax the
% independence assumption and develop an adaptive mechanism that infers
% availability of invitees based on their responses. In addition, we
% implicitly assumed that invitees prefer all options equally (by
% allowing invitees to specify their availability but not preferences),
% but people often have specific preferences. Thus it will be important
% to study the trade-off between optimizing the cost of scheduling
% process and finding a `good' schedule.
% Finally, we assumed honesty, promptness, and trust of invitees, but
% these are strong assumptions. In particular, promptness assumes that
% invitees do not procrastinate and do not deliberately delay responses.
% It is arguably true that if the scheduling process is efficient then
% it will indeed reduce the procrastination of invitees. However in
% practice an invitee may have an incentive to delay her response or to
% lie about her availability or preferences. Practical mechanisms should
% not be vulnerable to such strategic behavior of invitees, and this is
% yet another direction for future work.
%


\chapter{Probabilistic Matrix Inspection Problem}
\label{matrix:chapter}

In Chapter~\ref{bdoodle:chapter}, we assumed that an event organizer is only allowed to inspect a group of columns (as a batch of date/time options) by polling all invitees at once.
In this chapter, we generalize the notion of ``inspection'' by the event organizer, and assume that the organizer can inspect each entry of a random matrix at unit cost (by querying a certain invitee about a certain date/time option at a unit cost).
Consequently, \Times and \Inconveniences are always equal in this setting, and the goal is to find an optimal inspection sequence which is a permutation of entries of a given random matrix.

In classic Doodle, an inspection is performed over the entire random matrix by querying all invitees (rows) about all date/time options (columns).
In the Batched Doodle Problem, we considered a class of B-Doodle mechanisms in which all invitees are queried about a subset of date/time options.
In this chapter, we study the Probabilistic Matrix Inspection Problem in which an individual invitee is queried over a single date/time option. In practice, an organizer may want to ask the most important (and/or busiest) invitee first about a couple of date/time options, to ensure that the person is available, before asking other invitees. For instance, when a doctoral graduate student is scheduling his/her university oral examination at Stanford University (which is usually the defense of dissertation in computer science), it is required that both the student's primary advisor and the exam chair must be physically present while other examiners may participate virtually. %TODO CITE as of some date.
In such cases, the student can ask the advisor and exam chair to rule out certain date/time options that do not work for them, and then ask other examiners about their availability. 

While group scheduling is a motivating example for the Probabilistic Matrix Inspection Problem, there are many other applications that can be modeled in this manner. For instance, suppose that some research lab is trying to hire a new researcher and they are considering a number of candidates.
After reading each candidate's curricular vitae and research statement,
each member of the lab can estimate the likelihood of him/her approving the candidate. It is still necessary to interview a candidate to be certain of his or her quality.
In this simplified setting, the problem of finding a candidate who is approved by every member of the lab is equivalent to the problem of finding a feasible date/time option. Clearly, assuming that the lab is happy to hire any of the candidates whom all of the lab members approve, it is rational to minimize the expected number of in-person interviews until someone is hired or every candidate is rejected. 
Another example can include a search quest for housing or childcare facility.
Each house or facility is a candidate, and there is a set of requirements for the candidate being considered. Making inquiries about and taking a tour of the property or facility involves much effort -- even though much information is available through brochures or booklets, people in this situation would wish to minimize the time and effort they need to spend until they declare a winner.

In this chapter, we study the Probabilistic Matrix Inspection Problem which models the abstract essence of all applications we described earlier.
In Section~\ref{matrix:sec:notation}, we introduce notation and formally define the problem. 
In Section~\ref{matrix:sec:results}, we present our technical results.
In Section~\ref{matrix:sec:discussion}, we discuss how our technical results can be used for the real-life scenarios we described in this introduction. 





\section{Notation and Definitions} \label{matrix:sec:notation}



  We use the same definition of feasibility as in previous chapter, which is copied below. 

 \begin{definition}[Feasibility]
 Let $A$ be a matrix whose entries are from $\{0,1\}$, and refer to the entry of $A$ at row $r$ and column $c$ as $a_{r,c}$.
 We say that a column $c$ of $A$ is {\em feasible} if it consists only of 1's (otherwise it is {\em infeasible}). 
 We say that $A$ is {\em feasible} if it contains at least one feasible column (otherwise it is {\em infeasible}).
 \end{definition}

 In the Probabilistic Matrix Inspection Problem, we do not know of the values of the entries of $A$, as they are Bernoulli random variables, but we know of probability distribution of each entry.
 An input to the problem is this probability distribution. 
 \begin{definition}[Input instance]
 An input instance of the Probabilistic Matrix Inspection problem is a pair of matrices $(A, P_A)$ of size $n$ by $m$
 where $A$ is a matrix of Bernoulli random variables 
 and $P_A$ is the probability matrix associated with it.
 We denote an entry of $A$ as $a_{r,c}$ and of $P_A$ as $p_{r,c}$.
 Each entry $a_{r,c}$ of $A$ is a Bernoulli random variable, and $p_{r,c}$ is the probability of success for $a_{r,c}$ (i.e., $p_{r,c} = \mathbf{P}[a_{r,c} = 1]$).
 We define $s_c = \prod_{r=1}^{n} p_{r,c}$ to denote the probability that column $c$ is feasible (probability of success for column $c$).
 \end{definition}
 Throughout this work we will assume that the set of random variables $\{a_{r,c}\}$ are mutually independent. This assumption is crucial to our technical results because it allows to compute (in polynomial time) the probability of a specific realization of $A$ conditioning on the event in which some entries of $A$ have already been realized. 
 Without this assumption, it is unclear how one can compute such probabilities without having an access to the joint probability distribution over all realizations of $A$ (whose size is exponential in the size of $A$).

 We define an ``inspection'' as an operation that can be performed on $A$. 
 One can inspect an arbitrary entry $a_{r,c}$ of $A$ at unit cost, so as to know of the realization of the random variable. In group scheduling, an inspection corresponds to querying an agent about her availability for a certain outcome. 
 The objective of the problem is to determine whether $A$ is feasible or not, with minimum (expected) number of inspections possible. The expectation is with respect to the probability distribution specified by $P_A$.

 Because we are interested in determining feasibility of $A$ with minimum number of inspections, there are certain ``unnecessary inspections'' that an optimal strategy must avoid.
 For instance, if a certain entry $a_{r,c}$ is found to be unsuccessful (i.e., $a_{r,c} = 0$ is realized), then there is no need to inspect any other entry from the same column because the column is already known to be infeasible. Similarly, if a certain column is found to be feasible (which implies $A$ is feasible) or if all columns are found to be infeasible (which implies $A$ is infeasible), then there is no need to inspect any other entries of the matrix. Lastly, if $p_{r,c} = 0$ or $p_{r,c} = 1$, then there is no need to inspect the entry $a_{r,c}$ because we already know its realization with probability $1$.
 Therefore, without loss of generality, we will assume that $p_{r,c} \in (0,1)$ (i.e., $p_{r,c}\neq 0,1$) in this work. 

 Let us define what constitutes a solution to the problem. 
 \begin{definition}[Inspection policy]
 	Given an input instance $(A,P_{A})$, a solution is any permutation of the entries of $A$, and we call it an ``inspection policy'' (or simply, a ``policy'').
 \end{definition}
 The interpretation of a permutation is as follows. The entries of $A$ will be inspected in order specified by the permutation. After each inspection, if $A$ is found to be feasible or infeasible, the inspection process ends. Otherwise, it continues inspecting the entries as specified, but it will not make any unnecessary inspections as mentioned earlier. 

 If $\pi$ is a permutation of the entries of $A$, we write $C(\pi)$ to denote the number of inspections performed by $\pi$. $C(\pi)$ is a random variable whose probability distribution is determined by $P_A$. We are interested in finding an optimal policy which minimizes the expected number of inspections, $\mathbf{E}[C(\pi)]$.
 Note that there are $(nm)!$ permutations of the entries of $A$, and therefore exhaustive search for an optimal permutation will not produce an efficient algorithm. 

 \subsection{Example} \label{matrix:sec:example}

 Consider a 2-by-2 matrix $A$ of Bernoulli random variables whose probability of success is given by $P_A$ as follows. In group scheduling, this corresponds to two agents and a set of two date/time options being considered.
 \begin{equation*}
 A = 
 	\begin{bmatrix}
 		a_{1,1}  & a_{1, 2} \\
 		a_{2,1}  & a_{2, 2}
 	\end{bmatrix},
 P_A = 
 	\begin{bmatrix}
 		0.6  & 0.7  \\
 		0.9  & 0.8 
 	\end{bmatrix}
 \end{equation*}
 Let us consider an inspection policy $\pi$ which inspects the entries of $A$ column-by-column while inspecting them from top to bottom within a column:
 \begin{equation*}
 	\pi = 
 	\begin{pmatrix} 
 	1 & 2 & 3 & 4  \\
 	a_{1,1} & a_{2,1} & a_{1,2} & a_{2,2} 
 	\end{pmatrix}.
 \end{equation*}	
 Suppose that the realization of $A$ happens to be the identity matrix of size $2$ (i.e., $a_{1,1} = a_{2,2} = 1$ and $a_{2,1} = a_{1,2} = 0$). If we use $\pi$, it will first inspect $a_{1,1}$ and learn its realization. Since $a_{1,1} = 1$, it will inspect $a_{2,1}$ next only to find that column $1$ is infeasible after all. It will then inspect $a_{1,2}$ and learn that column $2$ is also infeasible, which implies that $A$ is infeasible. At this point, the inspection process terminates without inspecting $a_{2,2}$.
 This specific realization of $A$ happens with probability $p_{1,1}(1 - p_{2,1})(1-p_{1,2})p_{2,2}$, and yields $C(\pi) = 3$ because 3 inspections would occur. 
 If the realization of $A$ happens to be the null matrix (i.e., all entries are $0$'s), then $\pi$ would only inspect $a_{1,1}$ and $a_{1,2}$ but skip $a_{2,1}$ and $a_{2,2}$.  In this manner one can consider all $2^{2\cdot 2} = 16$ possible realizations of $A$, and compute $\mathbf{E}[C(\pi)]$ in this example. 

 Another way to compute $\mathbf{E}[C(\pi)]$ is by de-coupling $C(\pi)$ into two random variables $N(\pi, c)$ with $c\in\{1,2\}$ where $N(\pi,c)$ denotes the number of inspections (on column $c$) performed by $\pi$ conditioning on the event that (at least one element of) column $c$ is inspected.
 We can efficiently compute these: $\mathbf{E}[N(\pi, c)] = 1 \cdot \mathbf{P}[a_{1,c} = 0] + 2 \cdot \mathbf{P}[a_{1,c} = 1]$ for $c\in \{1, 2\}$.
 To express $\mathbf{E}[C(\pi)]$ in $N(\pi,c)$'s, we need to take conditional probability into account, as column 2 is inspected only if column 1 is infeasible: 
 $\mathbf{E}[C(\pi)] = \mathbf{E}[N(\pi,1)] + \mathbf{E}[N(\pi,2)] (1-s_1) = 2.382$. Recall that $s_c$ is the probability of success for column $c$.


\section{Technical Results and Algorithm} \label{matrix:sec:results}


 We first consider two special cases (1-row or 1-column matrices) of the Probabilistic Matrix Inspection problem, which admit intuitive, greedy algorithms. We then discuss a couple of interesting properties of an optimal inspection policy, which leads to our main result and algorithm.

 \subsection{1-Row Matrix and 1-Column Matrix}
 Let us first consider the case where an input matrix $A$ has only one row (i.e., $n = 1$).
 In this case it is natural to inspect entries with largest probability first because we can stop as soon as we find an entry whose value is $1$ -- which makes its column and $A$ feasible. This intuition is exactly what an optimal policy should do in the single row case.

 \begin{lemma}[1-Row Matrix]\label{matrix:lemma:single_row_opt}
 When $n = 1$, an inspection policy $\pi$ is optimal if and only if it inspects the entries in non-increasing order of their associated probabilities.
 \end{lemma}
 \begin{proof}
 	Without loss of generality let us assume that a policy $\pi$ inspects the entries in increasing order of their column index; that is, $\pi(i) = a_{1,i}$. 
 	Recall that $C(\pi)$ is a random variable that denotes the number of inspections $\pi$ incurs. 
 	We can express the expectation of $C(\pi)$ in terms of $p_{1,c}$'s as follows:
 	\small
 	\begin{equation} \label{matrix:eqn:exp_cost_pi_1row}
 		\mathbf{E}\left[C(\pi)\right] = 
 		m \left(\prod_{k=1}^{m-1}(1 - p_{1,k})\right) + 
 		\sum_{j=1}^{m-1} j \left(p_j \prod_{k=1}^{j-1} (1-p_{1,k}) \right).
 	\end{equation}
 	\normalsize
	
 	Suppose that there exists some $c^*$ such that $p_{1,c^*} < p_{1,c^*+1}$ (if no such $c^*$ exists, then $\pi$ is an inspection policy that inspects the entries in non-increasing order of probabilities).
 	Let $\pi'$ be the same policy as $\pi$ except we swap the order of $a_{1,c^*}$ and $a_{1,c^*+1}$. 
 	That is, $\pi'$ is defined as follows.
 	\begin{equation*}
 		\pi'(j) = 
 		\begin{cases}
 			\pi'(j) = \pi(c^*+1)&  \mbox{if~} j = c^* \\
 			\pi'(j) = \pi(c^*)  &  \mbox{if~} j = c^*+1 \\
 			\pi'(j) = \pi(j)  &  \mbox{if~} j \neq c^* \land j \neq c^*+1
 		\end{cases}
 	\end{equation*}	
 	After expressing $\mathbf{E}[C(\pi)]$ and $\mathbf{E}[C(\pi')]$ as in Equation~\ref{matrix:eqn:exp_cost_pi_1row}, one can re-arrange the terms to obtain the following:
 	\begin{equation} \label{matrix:eqn:proof_pi_1row}
 	\mathbf{E}\left[C(\pi)\right] - \mathbf{E}\left[C(\pi')\right] = 
 	\left(\prod_{j=1}^{c^*-1} (1-p_{1,j})\right) (p_{1,c^*+1} - p_{1,c^*}).
 	\end{equation}
 	This quantity is positive if $p_{1,c^*+1} > p_{1,c^*}$ (recall that $p_{1,j} \in (0,1)$ for all $j$ as mentioned in Section~\ref{matrix:sec:notation}). 
	
 	This proves the lemma because any policy that inspects an entry with smaller probability before another entry with higher probability is suboptimal, and therefore an optimal policy must inspect entries in non-increasing order of their associated probabilities. 
 \end{proof}
 Although the proof of Lemma~\ref{matrix:lemma:single_row_opt} is simple, it confirms correctness of our intuition. Equation~\ref{matrix:eqn:proof_pi_1row} illustrates this intuition; conditioning on the event that the first $c^*-1$ inspections fail (whose probability is the product term in Equation~\ref{matrix:eqn:proof_pi_1row}), the difference $\mathbf{E}[C(\pi)] - \mathbf{E}[C(\pi')]$ depends on the difference in the probabilities of success between the next-entry-to-be-inspected by $\pi$ and $\pi'$.

 We can also consider the case where an input matrix $A$ has only one column.
 Intuitively, if we wish to minimize the expected number of inspections, we must inspect entries with smallest probability first because we can stop as soon as we determine that $A$ is infeasible.
 Lemma~\ref{matrix:lemma:single_column_opt} formally states this intuition about optimal policy, and we omit a proof of it as it can be easily done by following the proof of Lemma~\ref{matrix:lemma:single_row_opt}. %TODO: do not omit?
 \begin{lemma}[1-Column Matrix] \label{matrix:lemma:single_column_opt}
 When $m = 1$, an inspection policy $\pi$ is optimal if and only if it inspects the entries in non-decreasing order of their associated probabilities.
 \end{lemma}


 \subsection{Inspection of Entire Column}
 Another interesting property of an optimal inspection policy is that once it inspects the first entry of a column, then it must commit to it and continue inspecting the remaining entries of the column until feasibility of the column is determined. Otherwise, if the policy switches to another column too soon, then it is not optimal. 
 \begin{theorem}[Optimality of inspecting entire column] \label{matrix:theorem:col_by_col_opt}
 Consider any inspection policy $\pi$.
 Without loss of generality, let us assume that for each column $c$, $\pi$ inspects $a_{n,c}$ the last among $n$ entries of the column. 
 Let $b_c$ be the index of $\pi$ such that $\pi(b_c) = a_{n,c}$. 
 Without loss of generality, assume $b_1 < b_2 < \cdots < b_m$ (we can do this by re-labeling the columns of $A$). 
 If there is some column $c^*$ such that $b_{c^*} > n\cdot c^*$, then $\pi$ is not optimal.
 \end{theorem}
 \begin{proof}
 First, note that $b_c \geq n\cdot c$ for all $c$ because we assumed $b_1 < b_2 < \cdots < b_m$, and therefore the entries of previous columns must appear before the last entry of each column.

 Let $\pi$ be an inspection policy being considered in the theorem for which there exists some $c$ with $b_{c} > n \cdot c$.
 Let us construct a different inspection policy $\pi'$.
 First, $\pi'$ inspects all entries of column $1$ in the same order $\pi$ does. 
 Then, $\pi'$ inspects all entries of column $2$ in the same order $\pi$ does, and so on. 
 In particular, $\pi'$ inspects all entries of a column before inspecting another column, while preserving the original ordering of the entries within each column that is given by $\pi$. 
 We will show that $\mathbf{E}[C(\pi')] < \mathbf{E}[C(\pi)]$.

 Let us define a set of new random variables which can be used to express $C(\cdot)$, as we did in Section~\ref{matrix:sec:example} when analyzing an example.
 Recall that $s_c = \prod_{r=1}^{n} a_{r,c}$ is the probability of success for column $c$.
 Let $N(\pi, c)$ ($N(\pi', c)$, respectively) be a random variable that denotes the number of entries of column $c$ that is inspected by $\pi$ (by $\pi'$, respectively), conditioning on the event that column $c$ is inspected (i.e., when the previous $c-1$ columns are infeasible).
 We can then express $\mathbf{E}[C(\pi)]$ and $\mathbf{E}[C(\pi')]$ as follows:
 \begin{equation} \label{matrix:eqn:exp_c_pi_decoupled}
 	\mathbf{E}[C(\pi)] = \sum_{c=1}^{m} \mathbf{E}[N(\pi, c)]\left( \prod_{k=1}^{c-1} (1 - s_c) \right)
 \end{equation}
 and
 \begin{equation} \label{matrix:eqn:exp_c_pi2_decoupled}
 	\mathbf{E}[C(\pi')] = \sum_{c=1}^{m} \mathbf{E}[N(\pi', c)]\left( \prod_{k=1}^{c-1} (1 - s_c) \right).
 \end{equation}


 To prove the theorem we will first show that for any realization of $A$, $N(\pi, c) \geq N(\pi', c)$ holds for all $c$; this immediately implies $\mathbf{E}[C(\pi)] \geq \mathbf{E}[C(\pi')]$.
 We will then show that there exists at least one realization of $A$ such that for some column $c'$ the strict inequality $N(\pi, c') > N(\pi', c')$ holds. These two statements together imply that $\mathbf{E}[C(\pi)] > \mathbf{E}[C(\pi')]$.


 Consider any realization of $A$ with the condition that the first $m-1$ columns are infeasible (recall that $m$ is the number of columns of $A$).
 Then $N(\pi, c) = N(\pi', c)$ for all $c$ regardless of feasibility of column $m$.
 To see why, both $\pi$ and $\pi'$ would inspect the same set of entries in each of the first $m-1$ columns in the same order until the column is determined to be infeasible, and therefore $N(\pi, c) = N(\pi', c)$ if $c<m$.
 If column $m$ is feasible, then both $\pi$ and $\pi'$ would inspect all $n$ entries of it, and thus we have $N(\pi, m) = N(\pi', m) = n$. Otherwise, if column $m$ is also infeasible (in which case $A$ is infeasible), then $\pi$ and $\pi'$ would inspect the same set of entries of column $m$ in the same order until the first infeasible entry of the column is found. Therefore if the first $m-1$ columns are infeasible we have $N(\pi, c) \geq N(\pi', c)$ for all $c$.

 Now consider any realization of $A$ with the condition that at least one of the first $m-1$ columns is feasible. Let $c'$ be the smallest index of feasible columns of $A$.
 Because the columns from $1$ to $c'-1$ are infeasible, $N(\pi, c) = N(\pi', c)$ for all $c < c'$ for the same reason we stated earlier for the other case. 
 Since $c'$ is feasible, $N(\pi, c') = N(\pi', c') = n$ as both policies would inspect all $n$ entries of $c'$. By our construction of $\pi'$ it is clear that $N(\pi', c) = 0$ for all $c > c'$; therefore we have $N(\pi, c) \geq N(\pi', c)$ for all $c>c'$. In summary $N(\pi, c) \geq N(\pi', c)$ holds for all $c$ in this case as well. 

 So far we proved the first claim we stated earlier: for all realizations of $A$, we have $N(\pi, c) \geq N(\pi', c)$ for all $c$.
 Let us now prove the second claim. Let $c^*$ be the smallest index $c$ of columns such that $b_c > nc$ (note that $c^* < m$ because $b_m = nm$ by definition). 
 Consider any realization of $A$ with the condition that the first $c^*-1$ columns are infeasible and column $c^*$ is feasible (feasibility of other columns do not matter). 
 Using the same arguments we used earlier, we can show that $N(\pi, c) = N(\pi', c)$ for all $c < c^*$, that $N(\pi, c^*) = N(\pi', c^*) = n$, and that $N(\pi', c) = 0$ for all $c> c^*$. 
 However, because $b_{c^*} > n \cdot c^*$, there is at least one entry $a_{r',c'}$ with $c' > c^*$ which  appears before $b_{n, c^*}$ in $\pi$ (otherwise, if no such entry exists, then $b_{c^*}$ would be equal to $n \cdot c^*$). This implies that there exists some $c'$ with $c' > c^*$ such that $N(\pi, c') > 0$. 
 This proves the second claim that for some realization of $A$, there is some column $c'$ for which $N(\pi, c') >  N(\pi', c')$, and together with the first claim we proved earlier, this implies that $\mathbf{E}[C(\pi)] > \mathbf{E}[C(\pi')]$. 

 This proves the theorem: Any policy that does not inspect all entries of a column consecutively is suboptimal.
 \end{proof}
 By Theorem~\ref{matrix:theorem:col_by_col_opt}, when seeking an optimal policy, it is sufficient to consider the set of policies that inspect an entire column before committing to another column. Lemma~\ref{matrix:lemma:single_column_opt} hints that one should inspect the entries of each column in increasing order of probabilities, and this is what we prove next.


 \subsection{Optimal Ordering within Column} 

 Lemma~\ref{matrix:lemma:single_column_opt} states that
 an optimal policy must inspect the entries in increasing order of their probability of success, if $A$ is a 1-column matrix.
 This argument can be generalized to the case where there is more than one column: If an optimal policy is to inspect an entry of some column $c$, it must inspect the entry with smallest probability of success first. 
 \begin{theorem}[Optimal ordering within column] \label{matrix:theorem:within_column_opt}
 	Consider any inspection policy $\pi$.
 	If there exist two entries $a_{r_1,c}$ and $a_{r_2,c}$ from the same column such that $a_{r_1,c}$ appears before $a_{r_2,c}$ in $\pi$ and $p_{r_1,c} > p_{r_2,c}$, then $\pi$ is not optimal. 
 	In other words, when restricted to each column, an optimal policy must inspect the entries of the column in non-decreasing order of probabilities.
 \end{theorem}
 \begin{proof}
 	Let $\pi$ be an inspection policy being considered in the theorem.
 	Because of Theorem~\ref{matrix:theorem:col_by_col_opt} we can assume, without loss of generality, that $\pi$ inspects all entries of column 1, followed by column 2, and so on.
 	Further let us assume that $\pi$ inspects the entries of each column in increasing order of their row index (we can do so by re-labeling the indices of entries). Precisely, $\pi(r + n (c-1)) = a_{r,c}$ defines $\pi$. Let $p_{r,c^*}$ and $p_{r+1,c^*}$ be the entries with $p_{r,c^*} > p_{r+1, c^*}$.
 	Let us consider a different inspection policy $\pi'$ that is the same as $\pi$ except that $\pi'$ inspects $p_{r+1,c^*}$ before $p_{r,c^*}$, by swapping the ordering of them.
 	\begin{equation*}
 		\pi'(j) = 
 		\begin{cases}
 			\pi'(j) = a_{r+1,c^*}  &  \mbox{if~} \pi(j) = a_{r,c^*} \\
 			\pi'(j) = a_{r,c^*}    &  \mbox{if~} \pi(j) = a_{r+1,c^*} \\
 			\pi'(j) = \pi(j)     &  \mbox{otherwise} 
 		\end{cases}
 	\end{equation*}	
	
 	We claim that $\mathbf{E}[C(\pi')] < \mathbf{E}[C(\pi)]$, which implies that $\pi$ is not optimal.
	
 	Let us define new random variables $N(\pi, c)$ and $N(\pi', c)$ as we did in our proof of Theorem~\ref{matrix:theorem:col_by_col_opt} (i.e., the number of inspections performed by the respective policy on column $c$, conditioning on the event that the column is inspected).
 	Then we can express $\mathbf{E}[C(\pi)]$ and $\mathbf{E}[C(\pi')]$ in terms of the new random variables and $s_c$'s as we did in Equations~\ref{matrix:eqn:exp_c_pi_decoupled} and \ref{matrix:eqn:exp_c_pi2_decoupled}.
	
 	Observe that $N(\pi, c) = N(\pi', c)$ for any realization of $A$ if $c \neq c^*$.
 	To see this, first note that column $c$ would not be inspected by $\pi$ or by $\pi'$ if any of the previous columns (that is, columns $1$ through $c-1$) is found to be feasible, in which case $N(\pi, c) = N(\pi', c) = 0$. Otherwise, if column $c$ is inspected, both policies would inspect the entries of $c$ in the very same order, so $N(\pi, c) = N(\pi', c)$ must hold. Therefore we conclude that $\mathbf{E}[N(\pi, c)] = \mathbf{E}[N(\pi', c)]$ when $c \neq c^*$. 
	
 	We will now show that $\mathbf{E}[N(\pi, c^*)] > \mathbf{E}[N(\pi', c^*)]$ holds. 
 	This immediately implies $\mathbf{E}[C(\pi)] > \mathbf{E}[C(\pi')]$ due to Equations~\ref{matrix:eqn:exp_c_pi_decoupled} and \ref{matrix:eqn:exp_c_pi2_decoupled}.
 	Let us express $\mathbf{E}[N(\pi, c^*)]$ in terms of $p_{r,c^*}$'s.
 	\begin{equation*}
 		\mathbf{E}[N(\pi, c^*)] = n\left( \prod_{k=1}^{n-1} p_{k,c^*}\right) + 
 		\sum_{j=1}^{n-1} j (1-p_{j,c^*}) \left( \prod_{k=1}^{j-1} p_{k,c^*} \right)
 	\end{equation*}
 	Note that the event $N(\pi, c^*) = j$ occurs if the first $j-1$ entries are feasible while the $j$-th entry is not feasible when $j < n$, and $N(\pi, c^*) = n$ occurs if the first $n-1$ entries are feasible (but the $n$-th entry's feasibility does not matter).
	
 	We can express $\mathbf{E}[N(\pi', c^*)]$ in a similar manner, and simplify  $\mathbf{E}[N(\pi, c^*)] - \mathbf{E}[N(\pi', c^*)]$ as follows:
 	\small
 	\begin{equation*}
 		\mathbf{E}[N(\pi, c^*)] - \mathbf{E}[N(\pi', c^*)] = 
 		r \left( \prod_{k=1}^{r-1} p_{k,c^*} \right) (p_{r,c^*} - p_{r+1,c^*}).
 	\end{equation*}
 	\normalsize
 	The quantity above is positive if $p_{r,c^*} > p_{r+1,c^*}$, which is the assumption we began with.
 	This proves the theorem.
 \end{proof}


 Theorems~\ref{matrix:theorem:col_by_col_opt} and \ref{matrix:theorem:within_column_opt} together tell us that 
 in order to find an optimal policy we only need to decide the ordering of the columns. 
 There are still $m!$ orderings of columns, and an exhaustive search algorithm would not be efficient.
 As we were able to generalize Lemma~\ref{matrix:lemma:single_column_opt} to Theorem~\ref{matrix:theorem:within_column_opt} by generalizing the optimal solution for 1-column case, it would be natural to consider generalizing Lemma~\ref{matrix:lemma:single_row_opt} in a similar manner.

 This idea leads to the following greedy algorithm: First we sort columns by their probability of success ($s_c = \prod_{r=1}^{n} p_{r,c}$) in decreasing order, and inspect the entries of each column in increasing order of their associated probabilities. However, as the following example shows, this algorithm is suboptimal. 
 \begin{equation*}
 A = 
 	\begin{bmatrix}
 		a_{1,1}  & a_{1, 2} \\
 		a_{2,1}  & a_{2, 2}
 	\end{bmatrix},
 	P_A =
 	\begin{bmatrix}
 		0.4459  & 0.2262 \\
 		0.4459  & 0.8114
 	\end{bmatrix}
 \end{equation*}
 Here we have $s_1 ~= 0.199$ and $s_2 ~= 0.184$, and the greedy algorithm would produce $\pi = \begin{pmatrix} a_{1,1} & a_{2,1} & a_{1,2} & a_{2,2} \end{pmatrix}$. Its expected cost, $\mathbf{E}[C(\pi)]$, is $2.428$, but if we inspect the second column first, then the expected cost is $2.407$ which is optimal in this example.
 One can consider another greedy algorithm which inspects the columns in increasing order of their expected number of inspections (within column), but this algorithm turns out to be suboptimal as well. 


 \subsection{Main Result and Algorithm}
 Let us present the main result that leads to an efficient algorithm for finding an optimal inspection policy. 

 \begin{theorem} \label{matrix:thm:main_result}
 	Let $s_c$ be the probability of success for column $c$ as before. 
 	Let $\mu_c$ be the expected number of inspections that column $c$ incurs if its entries are inspected in increasing order of their probability of success, conditioning on the event that column $c$ is inspected and infeasible.
 	An optimal policy must be a column-by-column policy (due to Theorem~\ref{matrix:theorem:col_by_col_opt}), must inspect the entries of each column in non-decreasing order of probabilities (due to Theorem~\ref{matrix:theorem:within_column_opt}), and must inspect the columns in non-decreasing order of $\mu_c(1 - s_c)/s_c$.
 \end{theorem}
 \begin{proof}
 	Consider a column-by-column inspection policy $\pi$ which inspects the column 1 through $m$ in increasing order of their index (we can assume this without loss of generality by re-labeling columns).
	
 	As before, let $N(\pi,c)$ be a random variable that denotes the number of inspections performed by $\pi$ on column $c$, conditioning on the event that column $c$ is inspected.
 	Then we can express $\mathbf{E}[N(\pi, c)]$ in terms of $s_c$ and $\mu_c$ as follows.
 	\begin{equation}\label{matrix:eqn:exp_cost_column_by_condition}
 		\mathbf{E}[N(\pi, c)] = s_c \cdot n  + (1-s_c) \cdot \mu_c
 	\end{equation}
 	This equation holds because if the column is feasible (with probability $s_c$), it would require $n$ inspections, but if it is not (with probability $1-s_c$), it would require $\mu_c$ inspections in expectation. The equation above simply considers these two events, and calculates the expected value of $N(\pi, c)$. 
	
 	Suppose that there is some column $c^*$ such that $\mu_{c^*}(1 - s_{c^*}) / s_{c^*} > \mu_{c^*+1} (1 - s_{c^*+1}) / s_{c^*+1}$. Because $\pi$ inspects column $c^*$ before column $c^*+1$, it would not be inspecting the columns in increasing order of $\mu_c(1-s_c)/s_c$.
 	Consider a different inspection policy $\pi'$ which inspects the columns in the same order as $\pi$ except that $\pi'$ inspects column $c^*+1$ before $c^*$ by swapping the inspection ordering of the two.
 	We can relate $N(\pi, \cdot)$ to $N(\pi', \cdot)$ as follows.
 	\begin{equation*}
 		N(\pi', c) = 
 		\begin{cases}
 			N(\pi, c^*+1) & \mbox{if~} c = c^* \\
 			N(\pi, c^*)   & \mbox{if~} c = c^* + 1 \\
 			N(\pi, c)     & \mbox{otherwise}
 		\end{cases}
 	\end{equation*}
	
 	As we did in proofs of Theorems~\ref{matrix:theorem:col_by_col_opt} and \ref{matrix:theorem:within_column_opt}, we can use Equations~\ref{matrix:eqn:exp_c_pi_decoupled} and \ref{matrix:eqn:exp_c_pi2_decoupled}, and simplify $\mathbf{E}[C(\pi)] - \mathbf{E}[C(\pi')]$ as follows.
	\small
 	\begin{equation} \label{matrix:eqn:main_result_diff}
 		\begin{aligned}
 		\mathbf{E}[C(\pi)] - \mathbf{E}[C(\pi')] ={} & \left(\frac{\mathbf{E}[N(\pi, c^*)]}{s_{c^*}}  - \frac{\mathbf{E}[N(\pi, c^*+1)]}{s_{c^*+1}} \right) \\
 		& \cdot \left( \prod_{j=1}^{c^*-1} (1 - s_j)\right) s_{c^*} s_{c^*+1}
 		 \end{aligned}
 	\end{equation}	
	\normalsize
 	The quantity in Equation~\ref{matrix:eqn:main_result_diff} is positive if the difference of the weighted expected values (in the first parentheses) are positive. Using Equation~\ref{matrix:eqn:exp_cost_column_by_condition} we obtain the following inequality.
 	\begin{equation*}
 		\begin{aligned}
 			{} &
 		\frac{\mathbf{E}[N(\pi, c^*)]}{s_{c^*}}  > \frac{\mathbf{E}[N(\pi, c^*+1)]}{s_{c^*+1}} \\
 		\Leftrightarrow &
 		\mu_{c^*} (1 - s_{c^*}) / s_{c^*} > \mu_{c^*+1}(1 - s_{c^*+1}) / s_{c^*+1} 
 		\end{aligned}
 	\end{equation*}
 	By definition of $c^*$, the second inequality above holds, which implies $\mathbf{E}[C(\pi)] > \mathbf{E}[C(\pi')]$.
 	Therefore, an optimal inspection policy must inspect the columns in non-decreasing order of $\mu_c(1 - s_c)/s_c$. 
 \end{proof}
 Because there is a unique ordering of columns if we sort them by $\mu_c(1-s_c)/s_c$ (up to ties), Theorem~\ref{matrix:thm:main_result} leads to the following algorithm: We inspect the columns in increasing order of $\mu_c(1-s_c)/s_c$, and in each column, we inspect the entries of it in increasing order of probabilities. 
 This algorithm can easily be implemented to run in polynomial time. 


\section{Discussion and Future Work} \label{matrix:sec:discussion}
In this work we defined the Probabilistic Matrix Inspection problem motivated by group scheduling and Doodle. 
We first considered two special cases, and discovered interesting properties of an optimal inspection policy which agree with our intuition. We then generalized our findings to design an efficient algorithm to solve the general case, and along the way we showed that two natural greedy algorithms fail to find an optimal solution. While we believe that our technical results make a great starting point for studying and optimizing a group scheduling process, there remain several open problems and future work to be done.

As we discussed in previous chapter, our model and algorithm rely on the assumption that probability estimates on availability of agents are available. We suggested several ideas motivated by previous work in the literature, but it will be important to deploy such ideas into a system, and integrate it with our algorithm. From a theoretical perspective, there remain several open problems. While we assumed that an inspection can be performed on a single entry at unit cost, one can generalize the cost model by allowing an inspection of any subset of entries whose cost depends on, for example, the number of entries being inspected. In the context of group scheduling, an inspection on many entries means querying multiple agents at the same time for one or many outcomes (but an inspection is not limited to the entries from the same column or row). This generalization is particularly useful when scheduling takes place in a hierarchical setting such as corporates. For instance the event organizer may feel that the cost of querying a supervisor is significantly different from that of querying a colleague.

Lastly, although we focused on relating our model to group scheduling, the Probabilistic Matrix Inspection problem has other applications. Finding the right childcare facility, for example, involves extensive inquiries as parents wish to gather more information about how they would handle certain situations, what benefits and environments they provide, and so on. Through advertisements or brochures parents may even be able to gauge the likelihood of a certain facility satisfying their needs. Yet they still need to inquire facilities for precise information, which can be modeled by our probabilistic matrix model where columns correspond to facilities and rows correspond to the needs of parents. 



\chapter{Group Activity Selection Problem}
\label{GASP:chapter}

In Chapters~\ref{bdoodle:chapter} and \ref{matrix:chapter}, we considered settings where an event organizer is trying to choose a date/time option at minimal cost given likelihood of availability of agents.
In this chapter we consider a different setting where the event organizer is trying to assign agents to different activities given constraints and preferences of agents. 
Imagine an event in which several activities are to take place concurrently. A group of agents are willing to participate and announce their activity preferences to the event organizer who is to assign agents to activities. In many settings, the agent preferences include not only which activities the agent is willing to participate in, but also the number(s) of participants in each activity that are acceptable to the agent.
For example, agents may wish to have enough participants in certain activities (such as group bus tour) to split the cost associated with it, whereas they may wish to have just few participants in activities with limited resources (such as a demo booth with a limited number of devices). 
We assume that an agent can be assigned to at most one activity (which naturally occurs when activities are run concurrently).

Given the preferences of agents, the organizer wishes to find a ``good'' assignment subject to certain rationality and/or stability conditions. 
The first condition is {\em individual rationality}: everyone who is assigned to some activity is willing to participate. In addition to individual rationality, the organizer may want to ensure that agents who are not assigned to any activity 
do not prefer to deviate from their assignment by joining an activity ({\em Nash stability}). Other concepts include {\em envy-freeness} that asserts that no unassigned agent would prefer to take the place of an assigned agent, and {\em perfection} in which all agents must be assigned.
To model this setting, Darmann et al.~\cite{GASP12WINE} proposed the Group Activity Selection Problem (\GASP) and defined three solution concepts (individual rationality, stability, and perfection). The authors also provided many computational complexity results, including NP-hardness results for \GASP, even under various restrictions on preferences of agents. 

These hardness results essentially argue that it is hard to find an assignment that maximizes the number of participants in activities. Suppose, however, that we are satisfied if we can assign a small number $k$ out of the $n$ participants to some (up to $k$) of the $p$ activities while satisfying our rationality criteria. There is a brute-force solution: try all possible $O(p^k)$ ordered choices of up to $k$ activities, all $O(k^k)$ choices for the number of participants in each activity, and all $O(n^k)$ ordered choices of $k$ participants; then check whether the desired criterion is satisfied by the induced assignment. This brute-force solution runs in $O((pnk)^k)$ time. While this running time is polynomial for any fixed $k$, it is not very desirable. A much better running time would be one of the form $O(f(k)\cdot (p+n))$ - such a runtime would be {\em linear}, regardless of the constant $k$, and the function $f$. More generally, on input size $N$, one would like an algorithm with runtime $f(k)\cdot N^c$, where $c$ is independent of $k$. Such an algorithm is known as fixed parameter tractable (FPT), and the problems that admit such algorithms are said to be in the class FPT. Developing FPT algorithms, especially linear time ones, greatly mitigates the NP-hardness of problems as it shows that these problems are actually quite tractable for many instances.

The field of parameterized complexity strives to classify NP-hard problems on a finer scale by analyzing time complexity in terms of both the input size and an additional parameter. The FPT problems are viewed as the most tractable problems in NP (of course, all polytime problems are naturally in FPT).
There is a hierarchy of complexity classes under parameterization (called the W-hierarchy), including FPT, W[1], W[2], etc. Hierarchy theorems show that FPT is contained in W[1] which is contained in W[2], and so on (see \cite{downey2012} for more details). It is believed that FPT $\neq$ W[1] and W[2] $\neq$ W[1], so that the NP-hard parameterized problems in W[2] are believed to be harder than those in W[1] that are themselves believed to be harder than the FPT problems. Furthermore, popular hardness assumptions such as the Exponential Time Hypothesis (ETH) of Impagliazzo and Paturi~\cite{impagl} can often be used to show that particular W[1]-hard problems cannot be solved in $N^{o(k)}$ time, giving concrete runtime lower bounds.

The goal of this work is to investigate how different solution concepts and different restrictions on agent preferences influence the difficulty of \GASP.
We focus on \GASPs with the size of the solution as our parameter, i.e. the number of assigned agents, or in the case of perfection, the number of used activities. 
We place different NP-hard versions of \GASPs under this parameterization into different parts of the W-hierarchy.
Our classification is nearly complete, as seen in Table~\ref{GASP:tbl:summary} (in Section~\ref{GASP:sec:results}). We show that \GASPs for individual rationality is in FPT, whereas for the other solution concepts the problem is W[1]-complete or W[2]-hard even for restricted types of agent preferences. However, if we restrict preference domains of agents, then more cases admit FPT algorithms. 

The restrictions on preferences that we consider are quite natural -- we consider the cases where agents have thresholds for each activity such that they are willing to participate if the number of participants is above or below the thresholds. In the case of {\em increasing} preferences, every agent $i$ approves an activity $a_j$ as long as at least $l_i(a_j)$ participants are assigned to it (hence the threshold is a lower-bound on the number of participants). In the case of {\em decreasing} preferences, agent $i$ approves an activity $a_j$ if at most $u_{i}(a_j)$ agents are assigned to it. Surprisingly to us, the case of decreasing preferences is actually easier than that of increasing preferences as it is FPT even when a stable assignment is to be found, whereas \GASPs for increasing preferences is W[1]-complete for all concepts except for individual rationality.

We also consider the special case of \GASPs in which all activities are equivalent. Here the preferences of the agents can vary but for each particular agent the preferences are the same for all copies of the activity. This special case is natural and captures many applications. We show that even though the problem is still NP-hard in this case~\cite{GASP12WINE}, all parameterized versions of it (except possibly perfection) are FPT. 

\paragraph{Related Work.}
Computational social choice is an interdisciplinary research area involving economics, social science, and computer science including artificial intelligence and multi-agent systems. Much work has been devoted to investigating both classical and parameterized complexity of social choice problems that range from winner determination~\cite{xia2014fixed,liu2016parameterized,betzler2010parameterized}, control problems in voting rule~\cite{erdelyi2010parameterized,hemaspaandra2013schulze,endriss2015parameterized}, coalition games~\cite{shrot2009easy,chitnis2011parameterized}, and more. This chapter studies parameterized complexity of a social choice problem under five different solution concepts and restrictions on inputs. 

Most closely related work is that of Darmann et al.~\cite{GASP12WINE}, in which the authors defined \GASP, and provided a number of classical complexity results for individual rationality, stability, and perfection.
In this chapter, we adopt their definitions, but we also consider the new solution concept of envy-freeness. It is worth noting that \GASPs is closely related to Hedonic Games (see Section 2.2 of \cite{GASP12WINE} and Section 2 of \cite{LEE15AAAI} for more details); in fact, \GASPs can be viewed as a class of hedonic coalition games with concise representation of preferences of agents.
Ballester~\cite{NP_hedonic} provides a number of computational complexity 
results (in fact, hardness results) for finding a core-stable, Nash-stable, or 
individually rational outcome in hedonic games and anonymous hedonic games, but these results do not apply to \GASPs because of the concise representation of an input to \GASP. Others have studied various solution concepts in hedonic coalition games such as stability and Pareto-optimality~\cite{BogomolnaiaJackson,DrezeGreenberg,AzizBrandl}.
Recently, Darmann~\cite{DARMANN15ADT} considered a different setting of \GASPs where agents are assumed to have strict, ordinal preferences over the outcomes, whereas both our work and \cite{GASP12WINE} assume that agents are indifferent among all outcomes that they approve of. It is an interesting future problem to consider how our results in this chapter can be extended to the ordinal setting that Darmann~\cite{DARMANN15ADT} considered. 
Lastly, we only consider truthful agents in this chapter, but we discuss in Chapter~\ref{GT:chapter} how strategic agents can affect the problem of finding solutions.
 
 
 
 
\section{Definitions and Known Results} \label{GASP:sec:prelim}

To make this work self-contained, we begin by introducing the formal definitions proposed by Darmann et al. in their work~\cite{GASP12WINE}, yet we make slight modifications to notation for readability and consistency in this chapter.


\begin{definition}
	In the Group Activity Selection Problem (\GASP), we are given a set of agents $N = \{1, 2, \dots, n\}$, a set of non-void activities $A^* =  \{a_1, a_2, \dots, a_p\}$, and the {\em void activity} ($\void$) which refers to the case when an agent does not participate in any of the activities in $A^*$.
An {\em outcome} is a pair $(a_j, x) \in A^* \times [n]$ which is interpreted as $x$ agents participating in non-void activity $a_j$. 
For each agent $i$ we are given a set $S_i$ of outcomes (called {\em approval set}) such that the outcomes in $S_i$ are equally liked and strictly preferred to $\void$, where $S_i \subseteq A^* \times [1,n]$.
We write $S_i(a_j) = \{x: (a_j, x)\in S_i\}$ to refer to the set of sizes which agent $i$ approves for activity $a_j$. 
\end{definition}

Similarly to the work of \cite{GASP12WINE}, in this work we assume that each agent is indifferent among the outcomes in $S_i$; that is, the void-activity ($\void$) draws the line between which outcomes are approved and which ones are not by the agent. While this is a simplifying assumption, note that hardness results immediately imply the same hardness for the general case without this assumption. By abusing notation, we also use $\void$ to refer to the outcome in which an agent is not assigned to any non-void activity. 

\begin{example} \label{GASP:eg:notation}
	Consider $N = \{1, 2, 3\}$ and $A^* = \{a_1, a_2\}$. There are six outcomes (besides $\void$) in the set $A^* \times [3]$. 
	Suppose $S_1 = \{(a_1, 1), (a_1, 2), (a_1, 3)\}$, $S_2 = \{(a_1,2), (a_2,2), (a_2, 3)\}$, and $S_3 = \{(a_1,1), (a_2, 1), (a_2, 2)\}$. In particular, agent $1$ approves $a_1$ for any size (i.e., unconditional approval) while does not approve $a_2$ for any size (i.e., unconditional refusal). Using our notation, $S_1(a_1) = \{1,2,3\}$ and $S_1(a_1) = \emptyset$. If we assign all agents to $a_1$, then $(a_1,3)$ is the outcome realized by all agents -- notice that only agent 1 approves it (and thus is willing to participate) while agents 2 and 3 do not (and thus are unwilling to participate). Naturally, this assignment induces instability. 
\end{example}


Let us define what constitutes a solution to \GASP.

\begin{definition}
	Let $A = A^* \cup \{\void\}$.
An assignment in \GASPs is a mapping $\pi: N \rightarrow A$ where $\pi(i) = \void$ means that agent $i$ is not assigned to any non-void activity. 
Each assignment naturally partitions the agents into at most $|A|$ groups. We define $\pi^0 = \{i : \pi(i) = \void \}$ and $\pi^j = \{i : \pi(i) = a_j\}$ for $j = 1, \dots p$, so that $|\pi^j|$ refers to the number of agents assigned by $\pi$ to a specific activity (including the void activity).
Let us define the size of an assignment, denoted by $|\pi|$, as the number of agents that are assigned to non-void activities; that is, $|\pi| = \sum_{j=1}^{p} |\pi^j|$.
\end{definition}

Note that $\pi$ induces an outcome for each agent: If $\pi(i) = \void$, then agent $i$ does not participate in any non-void activity (and thus outcome $\void$ is induced), and if $\pi(i) = a_j \in A^*$, then $(a_j, |\pi^j|)$ is the induced outcome for agent $i$.

The objective in \GASPs is to find a ``good'' assignment of maximum size thereby assigning as many agents to activities as possible. We define solution concepts which require different levels of rationality/stability in an assignment.

\begin{definition}
	Let $\pi$ be any assignment in \GASP.

	$\pi$ is {\em individually rational} (IR) if $\forall j \in [p]$ and $\forall i \in \pi^j$, it holds that $(a_j, |\pi^j|) \in S_i$.
	
$\pi$ is {\em (Nash) stable} if it is IR, and $\forall i \in N$ such that $\pi(i) = \void$ and $\forall a_j \in A^*$ it holds that $(a_j, |\pi^j| + 1) \not\in S_i$. 

$\pi$ is {\em envy-free} (EF) if it is IR, and $\forall i \in N$ such that $\pi(i) = \void$ and $\forall i' \in N$ such that $\pi(i') = a_j\in A^*$, it holds that $(\pi(i'), |\pi^j|) \not\in S_i$.

$\pi$ is {\em stable-EF} if it is both stable and envy-free.

$\pi$ is {\em perfect} if it is IR and $\pi(i) \neq \void$ for all $i\in N$.
\end{definition}
IR requires every agent assigned to an activity be unwilling to deviate.
Stability further requires that every unassigned agent be unwilling to deviate (unilaterally, without permission of other agents).
EF requires that every unassigned agent be not envious of someone else assigned to an activity.
Stability and EF together define a stronger solution concept than the two.
Lastly, a perfect assignment is the strongest solution concept which implies all others.


As Darmann et al. showed in their work~\cite{GASP12WINE}, finding a solution in \GASPs is NP-hard even under various restrictions on inputs. Yet we shall see that some of such restrictions make the problem less complex under parameterization. First we define two natural restricted domains of preferences of agents, called {\em increasing} and {\em decreasing} preferences. 
\begin{definition}
	We say that agent $i$ has an {\em increasing} preference for activity $a_j$ if there exists a threshold $l_i(a_j) \in \{1, 2, \dots, n+1\}$ such that $S_i(a_j) = [l_i(a_j), n]$ (where $[n+1,n] = \emptyset$).
	Similarly, we say that agent $i$ has a {\em decreasing} preference for activity $a_j$ if there exists a threshold $u_i(a_j) \in \{0, 1, \dots, n\}$ such that $S_i(a_j) = [1, u_i(a_j)]$ (where $[1,0] = \emptyset$).
\end{definition}
A natural example that accounts for an increasing preference is when an activity is associated with a cost that is to be split by the participants (e.g., group bus tourism in the city). For decreasing preferences, participants in some activity may need to share limited resources (e.g., a trial-demo of new wearable devices).  

Next, we consider a restriction on the activities when there may exist multiple ``copies'' of the same activity (e.g., chess matches with multiple chessboards). We define equivalence of activities, and consider a special case of the problem in which all activities are equivalent. 
\begin{definition}
	We say that two activities $a_j$ and $a_{j'}$ are {\em equivalent} if $\forall i \in N$, $S_i(a_j) = S_i(a_{j'})$. 
\end{definition}

Let us re-visit the problem instance from Example~\ref{GASP:eg:notation}, and relate it to various definitions and concepts we have defined in this section. 
\begin{example}
In Example~\ref{GASP:eg:notation}, note that agent $2$ has an increasing preference for $a_2$ with $l_2(a_2) = 2$ while agent $3$ has a decreasing preference for $a_2$ with $u_3(a_2) = 2$. Agent $1$ has (degenerate) increasing/decreasing preferences for both $a_1$ and $a_2$ with $l_1(a_1)=1, u_1(a_2) = 3$ and $l_1(a_2)=4,u_1(a_2)=0$.

Consider an assignment $\pi$ with $\pi(1) = \pi(2) = a_1$ and $\pi(3) = a_2$; under the assignment $\pi$, agents $1,2$ realize the outcome $(a_1, 2)$ and agent $3$ realizes $(a_2, 1)$. It is easy to check that $\pi$ is prefect (and thus IR, stable, and EF). Consider another assignment $\pi'$ with $\pi'(1) = a_1$ and $\pi'(2)=\pi'(3)=\void$; under $\pi'$, agent $1$ realizes the outcome $(a_1, 1)$ and agents $2,3$ realize $\void$. $\pi'$ is individually rational (as $(a_1,1)\in S_1$), but it is not stable (as $(a_1,2)\in S_2$) or envy-free (as $(a_1,1) \in S_3$).
\end{example}

Darmann et al.~\cite{GASP12WINE} proved many hardness results of the Group Activity Selection Problem, even under restrictions on preferences of agents.
Here we mention the most relevant results of theirs to this work.

\begin{theorem} \label{GASP:thm:nphard}
	Finding a perfect assignment is NP-hard.
	It remains to be NP-hard even if all agents have increasing preferences for all activities,
	even if all agents have decreasing preferences for all activities,
	or even if all activities are equivalent (Theorems 4.1, 4.2, 4.3, and 4.4 of \cite{GASP12WINE}).
\end{theorem}
As corollaries, the following problems are also NP-hard under any of the three restrictions mentioned in Theorem~\ref{GASP:thm:nphard}. Note that the first three problems parameterize the size of an assignment while the last problem parameterizes the number of used activities in a perfect assignment.
\begin{itemize}
	\item $k$-IR-GASP: Does there exist an individually rational assignment of size $k$?
	\item $k$-Stable-GASP: Does there exist a stable assignment of size $k$?
	\item $k$-EF-GASP: Does there exist an envy-free assignment of size $k$?
	\item $k$-Perfect-GASP: Does there exist a perfect assignment using $k$ non-void activities?
\end{itemize}
In what follows, we show that the parameterized complexity of $k$-GASP varies with different solution concepts and under different restrictions on inputs to \GASP.





\section{Parameterized Complexity} \label{GASP:sec:results}

In this section, we provide parameterized complexity of the problems mentioned in previous section: $k$-IR-GASP, $k$-Stable-GASP, $k$-EF-GASP, and $k$-Perfect-GASP.
Our main contributions are summarized in Table~\ref{GASP:tbl:summary}.
Note that all problems considered in this work are known to be NP-hard (and NP-complete) due to Darmann et al.~\cite{GASP12WINE}.
The ``general case'' refers to the case where approval set of each agent can be any set of outcomes (i.e., no restrictions on preferences of agents). 
``Increasing (decreasing) preferences'' refer to the case where all agents have increasing (decreasing) preferences for all activities. 
``Equivalent activities'' refer to the case where all activities are (pairwise) equivalent (i.e., they are copies of one kind of an activity). 

\begin{table}[h!]
	\centering
\begin{tabular}{|*{5}{c|}}\hline
	 					& $k$-IR-GASP & $k$-Stable-GASP & $k$-EF-GASP & $k$-Perfect-GASP \\ \hline
General case 			& FPT 		  & $W[1]$-hard 	& $W[1]$-complete & $W[2]$-hard \\ \hline
Increasing preferences 	& FPT 		  & $W[1]$-complete & $W[1]$-complete & $W[2]$-hard \\ \hline
Decreasing 	preferences	& FPT		  & FPT				& $W[1]$-complete & $W[2]$-hard \\ \hline
Equivalent activities 	& FPT		  & FPT				& FPT 		  & Unknown \\ \hline
\end{tabular}
\caption{Summary of results on parameterized complexity of \GASP.} 
\label{GASP:tbl:summary}
\end{table}

\subsection{$k$-IR-GASP: Finding Individually Rational Assignments in GASP}

Recall that $k$-IR-GASP is the problem of finding an IR assignment of size $k$. 
We show that $k$-IR-GASP is in FPT. As seen below, our algorithm runs in $\exp{k} np\log n$ time, where $n$ is the number of agents and $p$ the number of activities. The input size to \GASPs is $\Theta(n^2p)$ as the
number of possible outcomes is $np$ and each of the $n$ agents needs to specify a subset of approved ones. As we only care about solutions of size $k$, we are only interested in those preferences of the agents for numbers of activity participants that are at most $k$. We can hence prune the input to size $\Theta(nkp)$ (in about that much time, assuming random access to preferences). In general, we cannot prune the input more, and for any constant $k$, our FPT algorithm below runs in near linear time in the input size!

\begin{theorem} \label{GASP:thm:k_IR_GASP_FPT}
$k$-IR-GASP is in FPT, and can be solved in time $2^{O(k)}(np \log n)$ where $n= |N|$ and $p = |A^*|$.
%\footnote{Note that the size of an input is $O(n^2p)$ as there are $n$ agents and $O(np)$ outcomes of which each agent approves a subset, and thus this algorithm runs in sub-linear time in size of input for every fixed $k$.}
\end{theorem}
\begin{proof}
	We use ``Color Coding'' to design a randomized (Monte Carlo) algorithm, which can easily be de-randomized using a family of $k$-perfect hash functions as shown in the work~\cite{ColorCoding}.

	Recall that $N = \{1, 2, \dots, n\}$ is the set of agents, $A^* = \{a_1, a_2, \dots, a_p\}$ is the set of non-void activities, and $S_i$ is the set of approved outcomes for agent $i$. We first color each agent using $k$ colors independently and uniformly at random. We seek to assign exactly one agent of each color to some activity such that the resulting assignment is IR and of size $k$. 
	
	Let $c(i)$ denote the color of agent $i$ where $c(i) \in [k]$. 
	For each activity $a_j \in A^*$ and every subset $C$ of colors (i.e., $C \subseteq [k]$), we will first determine whether it is possible to assign to activity $a_j$ exactly $|C|$ agents with distinct colors specified by $C$ while satisfying the IR constraint for each agent; we refer to this subproblem by $T(C, j)$ for every set of colors $C$ and activity $a_j$. 
	For any fixed $a_j$ and $C$, we can check for every color $d\in C$ whether there exists an agent $i$ with $c(i) = d$ and $(a_j, |C|) \in S_i$ in time $O(n)$ by iterating over the set of agents and look up her approval set, $S_i$. If the test is affirmative, we conclude that we can assign exactly $|C|$ agents with distinct colors specified by $C$ to activity $a_j$. We perform this sub-routine for every activity $a_j\in A^*$ and every subset of colors, which can be done in time $O(n\cdot p \cdot 2^k)$ overall.
	
	Next, we solve another type of sub-problems (which we call $R(C, j)$) to check if it is possible to assign $|C|$ agents of distinct colors in $C$ to activities in $A_j = \{a_1, a_2, \dots, a_j\}$ for every $j\leq p$ and $C\subseteq [k]$. When $j = 1$, the sub-problems $R(C, j)$ and $T(R, j)$ are equivalent, and thus we simply use the result of $T(R, j)$ to solve $R(C, j)$. To solve $R(C, j)$ when $j>1$, we enumerate over every subset $C' \subseteq C$, and solve $R(C', j-1)$ and lookup the result of $T(C\setminus C', j)$. If both subproblems $R(C', j-1)$ and $T(C \setminus C', j)$ are affirmative for some $C' \subset C$, then we conclude that $R(C, j)$ is also affirmative.
	
	If $R([k], p)$ is affirmative, then we can find an IR assignment of size $k$ where those $k$ agents are distinctly colored. 
		There are at most $O(2^k \cdot p)$ subproblems in the form of $R(C, j)$ to be solved, and each subproblem can be solved in time $O(2^k)$ (as we enumerate over all subsets of $C$).
		
		Therefore the overall running time of this algorithm is bounded by $O(4^k \cdot p + 2^k \cdot (np)) = 2^{O(k)}(np)$; in particular, it is exponential only in $k$. It is easy to confirm that this algorithm is a Monte Carlo algorithm (i.e., if it finds a solution, it is guaranteed to be IR).  On the other hand, if there exists an IR assignment of size $k$ in the original instance, there is a chance that this algorithm does not find it when the $k$ agents are not colored properly. The probability of a proper coloring (i.e., the $k$ agents in the assumed assignment are colored distinctly) is at least $k!/k^k > 1/e^k$, and this is only exponentially small in $k$. Therefore one can repeat this randomized algorithm $e^k\ln n$ times to increase the probability of success to $1-1/n$ (with overall runtime $2^{O(k)}(np \log n)$).

	To de-randomize this algorithm, one can use a $k$-perfect family of hash functions from $N$ to $[k]$.
	Specifically, if we have a list of colorings of agents $N$ such that for every subset $N' \subseteq N$ of size $|N'| = k$ there exists a coloring in the list that gives each agent in $N'$ a distinct color, then we can simply enumerate over this list of colorings. This is precisely what a $k$-perfect family of hash functions from $N$ to $[k]$ is, and it is known that the size of the family can be specified using $2^{O(k)}\log n$ bits (for details please see~\cite{ColorCoding}). Therefore we can obtain a deterministic algorithm whose runtime is bounded by $2^{O(k)}(np \log n)$, and conclude that $k$-IR-GASP is in FPT. 
\end{proof}

Individual rationality is the weakest solution concept among the four we consider, and naturally $k$-IR-GASP is the least complex problem. On the other hand, we will show that other problems are $W[1]$- or $W[2]$-hard unless additional restrictions are assumed. Note that Theorem~\ref{GASP:thm:k_IR_GASP_FPT} proves an easiness result, and therefore it is implied that $k$-IR-GASP is in FPT under any of the three restrictions on inputs we consider (see the $k$-IR-GASP column in Table~\ref{GASP:tbl:summary}).


%%% =============================================================== Stable
\subsection{$k$-Stable-GASP: Finding Stable Assignments}

Recall that $k$-Stable-GASP is the problem of finding a stable assignment of size $k$. 
Stability is a stronger solution concept than individual rationality, and the problem of finding a stable assignment is harder than that of finding an IR assignment. This relationship is not apparent under the classic complexity hierarchy as both problems are NP-complete. However, under parameterization, $k$-IR-GASP is FPT whereas $k$-Stable-GASP is $W[1]$-hard.

\begin{theorem}
$k$-Stable-GASP is $W[1]$-hard.
The problem remains to be $W[1]$-hard even if each agent approves at most one size per activity (i.e., $|S_i(a_j)| \leq 1$ for all $i\in N$ and $a_j\in A^*$).
\end{theorem}
\begin{proof} 
	The $k$-Clique problem is known to be $W[1]$-hard, and we reduce the $k$-clique problem to $k$-Stable-GASP. 
	
	\paragraph{Construction of \GASPs instance.}
	Consider an instance of the $k$-Clique problem, $G = (V, E)$ and a parameter $k$ where $V = \{v_1, v_2, \dots, v_n\}$.
	Let us create an instance of \GASPs as follows: Let $N = V \cup \{w_{i, x} : (1 \leq i \leq n) \land (1 \leq x \leq k-1)\}$; that is, we create $n$ node-agents $v_i$'s (by abusing notation) and $(k-1)$ copies of neighbor-agents ($w_{i,x}$'s) for each $v_i$. The neighbor-agents will be used to ``select'' the $k-1$ edges incident to each node if the node is to be included in a clique we are seeking.
	Let $A^* = \{a_1, \dots, a_k \} \cup \{ e_{i,j} : 1 \leq i < j \leq n \}$; we create $k$ clique-activities (which are used to determine membership of a node in a clique) and $\binom{n}{2}$ edge-activities $e_{i,j}$ (where $i<j$).
 	For each node-agent $v_i$, we set its approval set $S_{v_i} = \{(a_j, 1) : 1 \leq j \leq k\} \cup \{(e_{i,j}, 3) : i \neq j\}$. 
 	For each neighbor-agent $w_{i, x}$, we set its approval set $S_{w_{i,x}} = \{(e_{i,j}, 2) : (v_i, v_j) \in E\}$.
 	Finally	we set the parameter $k'$ of \GASPs (to distinguish from $k$ in the Clique problem) to $k'=k+2\binom{k}{2}$.
	This is a valid FPT-reduction as $k'$ depends only on $k$ but not on $n$, and the size of our instance of \GASPs is bounded by $O(nk^3)$ as there are $O(nk)$ agents and $O(k^2)$ activities in the instance.

	Let us first describe how a clique in the original instance and a stable assignment in the \GASPs instance we created are related. A node-agent is assigned to a clique-activity if and only if its corresponding node belongs to a (corresponding) clique. For each node-agent, there exists $k-1$ neighbor-agents, and these neighbor-agents must be assigned properly to edge-activities in order to ensure that the resulting set of nodes is indeed a clique. 		 
	We claim that there exists a clique of size $k$ in the original instance if and only if there exists a stable assignment of size $k'$ in the \GASPs instance we constructed. 
 	
	\paragraph{Proof of equivalence between instances.}	
	First suppose there exists a clique $C$ of size $k$ in $G$, and without loss of generality assume $C = \{v_1, v_2, \dots, v_k\}$. Consider the following assignment $\pi$:
	\begin{equation*}
	\pi(v_i) = \begin{cases} a_i & i \leq k \\ \void & i > k \end{cases} 
	\mbox{~~~~and~~~~~} 
	\pi(w_{i,x}) = \begin{cases} e_{i, x+1} & i \leq k \land i \leq x \\ e_{x, i} & i \leq k \land i > x \\  \void & i > k \end{cases}.
	\end{equation*}

	That is, node-agents are assigned to the clique-activities and their associated neighbor-agents are assigned to the edge-activities; all other agents are assigned to the void activity. Clearly $\pi$ assigns exactly $k + 2\binom{k}{2} = k'$ agents to non-avoid activities. While the details are omitted, it is easy to verify that $\pi$ is indeed a stable assignment.
	
	Conversely, suppose there is a stable assignment $\pi$ of size $k'=k+2\binom{k}{2}$, and we want to show that a clique of size $k$ exists in $G$. 
	First notice that for each edge-activity $e_{i,j}$ there are precisely two agents who approve the outcome $(e_{i,j}, 3)$ -- namely, $v_i$ and $v_j$. Therefore if $\pi$ is stable, it cannot assign any node-agents (of the form $v_i$) to any edge-activity (of the form $e_{i,j}$). In other words, for each $v_i$, $\pi(v_i) \in \{\void\}\cup\{a_1, \dots, a_k\}$. Let $C = \{v_i : \pi(v_i) \neq \void \}$; since there are $k$ clique-activities, $|C| \leq k$. We claim that $|C| = k$ if $\pi$ is stable; if $|C| < k$, then there exists some $a_j$ such that no agent is assigned to it; since $k \leq n$, there must be some $v_i$ such that $\pi(v_i) = \void$. This implies that $\pi$ is not stable because $(a_j, 1) \in S_{v_i}$ while $\pi(v_i) = \void$; hence $|C| = k$ must hold. Without loss of generality, we now assume that $\pi(v_i) = a_i$ if $i \leq k$ and $\pi(v_i) = \void$ if $i > k$ (by re-labeling), and we claim that $C = \{v_1, v_2, \dots, v_k\}$ is a $k$-clique in the original instance.
	
	We argued earlier that $\pi$ never assigns node-agents to any edge-activities if it is stable. This implies that, if $\pi$ assigns any agent to an edge-activity, it must be the case that $\pi$ assigns exactly two neighbor-agents (of the form $w_{i,x}$) to it (due to the construction of $S_{w_{i,x}}$'s). If $\pi$ is stable, then $\pi$ must assign no neighbor-agents to $e_{i,j}$ if $i>k$ or $j>k$ and exactly two neighbor-agents to $e_{i,j}$ if $i\leq k$ and $j\leq k$. To prove the first claim, suppose that $\pi$ assigns two neighbor-agents to $e_{i,j}$ where $i>k$ (and recall that $\pi(v_i) = \void$ when $i>k$). Then $\pi$ is not stable because $(e_{i,j}, 3) \in S_{v_i}$, and thus $v_i$ wishes to join $e_{i,j}$, and this is a contradiction. Similarly one can prove the claim in the case where $j>k$. To prove the second part, recall that $|\pi| = k' = k + 2\binom{k}{2}$. Since $\pi$ assigns exactly $k$ node-agents to non-void activities, it must assign $k(k-1) = 2\binom{k}{2}$ neighbor-agents to $\binom{k}{2}$ edge-activities from $\{e_{i,j}: i, j \leq k\}$. By the Pigeon Hole principle, $\pi$ must assign two agents to each of the edge-activities in $\{e_{i,j} : i,j \leq k\}$. This implies that there is an edge between $v_i$ and $v_j$ in the original instance if $i,j \leq k$. Otherwise, if $(v_i,v_j) \not\in E$ where $i,j \leq k$, then there is no neighbor-agents who can be assigned to $e_{i,j}$, which contradicts the assumption that $\pi$ is of size $k'$. This completes the proof of the claim that if a stable assignment of size $k'$ exists, then a clique of size $k$ exists, and one can construct a clique by choosing the corresponding nodes to the node-agents that are assigned to one of the clique-activities. 
	
	Note that in our reduction each agent approves at most one size per activity, proving the second statement in the theorem.
\end{proof}

Because our reduction increases the parameter quadratically (i.e., $k' = k + 2\binom{k}{2} = k^2$), the following corollary follows immediately. 
\begin{corollary}
Unless the Exponential Time Hypothesis (ETH) fails, $k$-Stable-GASP cannot be solved in time $(np)^{o(\sqrt{k})}$.
\end{corollary}

We now consider $k$-Stable-GASP with the restriction that all agents have increasing preferences for all activities. 
\begin{theorem}
$k$-Stable-GASP is $W[1]$-complete when all agents have increasing preferences for all activities.
\end{theorem}
\begin{proof} 
	We show $W[1]$-hardness of the problem by reducing from the $k$-Clique problem, and show $W[1]$-completeness by reducing $k$-Stable-GASP to the $k$-Clique problem.

	\paragraph{Construction of \GASPs instance.}
	Let $G = (V, E)$ be a graph instance of the $k$-Clique problem.
	For each vertex $v_i \in V$, we create $k^2$ copies of $v_i$ as agents (call them copies of $v_i$) and create an activity $a_i$; this creates $k^2|V|$ agents and $|V|$ activities.
	For each edge $e_{i,j} = (v_i, v_j) \in E$, we create two copies of $e_{i,j}$ as agents (call them copies of $e_{i,j}$) and create an activity $w_{i,j}$; this creates $2|E|$ agents and $|E|$ activities. 
	Let $k' = k^3 + k^2 - k$, and we create $k'+1$ copies of dummy agents (call them copies of $z$).
	For each of the $k^2$ copies of $v_i$ agents, we set its approval set such that $l_{v_i}(a_i) = k^2$ (i.e., approves any outcome with $a_i$ and size $k^2$ or larger) and $l_{v_i}(w_{i,j}) = 3$ if $(v_i, v_j) \in E$ and $l_{v_i}(\cdot) = n+1$ for all other activities (where $n = k^2|V| + 2|E| + k' + 1$ is the total number of agents we create).
	For each of the two copies of $e_{i,j}$ agents, we set its approval set such that $l_{e_{i,j}}(w_{i,j}) = 2$.
	For each of the $k'+1$ copies of $z$ agents, we set its approval set such that $l_{z}(w_{i,j}) = 4$ for all $(i,j)$ where $(v_i, v_j) \in E$.
	We claim that a clique of size $k$ exists in $G$ if and only if a stable assignment of size $k'$ exists in the \GASPs instance we created. 

	\paragraph{Proof of equivalence between instances.}	
	First, suppose that $C = \{v_1, v_2, \dots, v_k\}$ is a clique of size $k$ in $G$. We can construct a stable assignment of size $k'$ as follows: (a) For $k^2$ copies of $v_i$, we assign them to $a_i$ if $v_i\in C$ and to $\void$ otherwise, (b) for two copies of $e_{i,j}$, we assign them to $w_{i,j}$ if $v_i\in C$ and $v_j\in C$ and to $\void$ otherwise, and (c) copies of $z$ are assigned to $\void$. Note that this assignment assigns exactly $k^3 + 2\binom{k}{2} = k^3 + k(k-1) = k'$ agents to non-void activities. It is easy to verify that $\pi$ is IR and stable, proof of which is created to due space. 
	
	Conversely, now suppose that $\pi$ is a stable assignment of size $k'$, and we show that there exists a clique of size $k$ in $G$. 
	If $\pi$ assigns three or more agents to any $w_{i,j}$, then $\pi$ must assign all copies of $z$ to some activity (possibly $w_{i,j}$) or $\pi$ would not be stable; yet we know that $\pi$ is of size $k'$ and there are $k'+1$ copies of $z$, and therefore $\pi$ can only assign two or fewer agents to each $w_{i,j}$. 
	If $\pi$ assigns two agents to some $w_{i,j}$, then those two agents must be the two copies of $e_{i,j}$ because no other agent approves the outcome $(w_{i,j}, 2)$. 
	Furthermore, if $\pi$ assigns the two copies of $e_{i,j}$ to $w_{i,j}$, then $\pi$ must assign all $k^2$ copies of $v_i$ to $a_i$ and all $k^2$ copies of $v_j$ to $a_j$ -- otherwise, $\pi$ would not be stable. Let $W$ be the set of activities of the form $w_{i,j}$ such that $\pi$ assigns exactly two agents to $w_{i,j}$; if $|W| > \binom{k}{2}$, then there must be at least $k+1$ indices that appear in elements of $W$, which implies that $\pi$ must assign agents to at least $k+1$ activities of the form $a_i$. 
	This is a contradiction because $\pi$ is of size $k'$ but $(k+1)k^2 > k'$. 
	Therefore, $|W| \leq \binom{k}{2}$. 
	Now suppose $|W| < \binom{k}{2}$ instead. 
	As argued earlier, $\pi$ can assign to at most $k$ activities of the form $a_i$, but $k^3 + 2|W| < k'$, which implies that $\pi$ cannot be of size $k'$ if $|W| < \binom{k}{2}$. 
	Lastly, suppose $|W| = \binom{k}{2}$ (and from previous arguments, it is clear that the number of the indices that appear in the elements of $W$ must be exactly $k$); without loss of generality, assume $W = \{w_{i,j} : 1 \leq i < j \leq k\}$ (by re-labeling) -- this implies that $\pi$ assigns $k^2$ copies of $v_l$ to $a_l$ if $1 \leq l \leq k$, but more importantly, it implies that $(v_i, v_j) \in E$ because we create $w_{i,j}$ if and only if there is an edge between $v_i$ and $v_j$. 
	That is, $C = \{v_1, v_2, \dots, v_k\}$ is a clique in the original instance. 
	This completes the proof of W[1]-hardness of the problem.

	\paragraph{Proof of completeness.} %TODO
	Let us reduce $k$-Stable-GASP with increasing preferences to the $k$-Clique problem, which shows that $k$-Stable-GASP is in $W[1]$. (Details to be added.)

\end{proof}

Unlike the case of increasing preferences, if all agents have decreasing preferences the problem admits an FPT algorithm. 

\begin{theorem}
$k$-Stable-GASP is in FPT when all agents have decreasing preferences for all activities.
\end{theorem}
\begin{proof}
We use Color Coding to reduce the $k$-Stable-GASP with decreasing preferences to a variant of the Vertex Cover problem.
With probability which is exponentially small only in $k$, we color agents and activities ``properly'', and given a proper coloring we can find a stable assignment of size $k$ in polynomial time in $n,p$ yet exponential only in $k$. 

\paragraph{Preliminaries.}
Suppose that a stable assignment of size $k$ exists, and without loss of generality we know that it assigns $k$ agents to $l$ distinct activities (where $l \in [1, k]$), which can be done by checking every value in $[1,k]$. 
We first color agents and activities using $l$ colors $1$ through $l$, uniformly and independently at random (let $c(i)$ denote the color of agent $i$ and $c(a_j)$ the color of activity $a_j$), and then fix the value of $k_d$ for each $d\in[1, l]$ such that $\sum_{d\in[1,l]} k_d = k$. We say that the coloring $c$ (together with $l$ and $k_d$'s) is compatible with a stable assignment $\pi$ of size $k$ (using $l$ activities) if $\pi$ assigns exactly $k_d$ agents to an activity of color $d$ for every $d\in [1, l]$. Given some coloring $c$, our algorithm will find a stable assignment compatible with $c$ or determine that no such stable assignment exists. 
It is clear that any stable assignment (of size $k$) has at least one compatible coloring. 
With probability at least $(1/l)^{l+k}$ (which is exponentially small only in $k$), our randomized coloring is a compatible coloring of a stable assignment of size $k$ (provided that it exists); the algorithm can be  easily be de-randomized using a family of $k$-perfect hash functions as shown in the work~\cite{ColorCoding}. 

\paragraph{FPT Algorithm.}
We now proceed with fixed values of $l$ and $k_d$'s as well as some coloring $c$ as described earlier. 
We will use the special color $l+1$ to mark the agents and activities that cannot be assigned/used in any stable assignment that is compatible with the given coloring $c$. 
Define $N_d = \{i\in N: c(i) = d\}$ and $A^*_d = \{a_j \in A^* : c(a_j) = d\}$ where $d\in [1, l+1]$; these subsets naturally partition $N$ and $A^*$ into $l+1$ subsets by their colors (at first $N_{l+1}$ and $A^*_{l+1}$ are empty, but we may re-color some agents and activities during the course of the algorithm).
Let $N(a_j) = \{i \in N_{c(a_j)} : u_i(a_j) \geq k_{c(a_j)}\}$, which is the set of agents who have the same color as $a_j$ and approve the size $k_{c(a_j)}$ for activity $a_j$ (recall that agents have decreasing preferences, so we only need to check their upper-bound $u_i(a_j)$ for a given activity $a_j$).
If $|N(a_j)| > k_{c(a_j)}$, then we label the activity $a_j$ as ``popular'' because any compatible assignment must assign $k_{c(a_j)}$ agents of the same color to $a_j$, but more than $k_{c(a_j)}$ agents approve $a_j$ for size $k_{c(a_j)}$.
If any color $d\in [1, l]$ contains two or more popular activities, we reject the coloring because there is no stable assignment compatible with this coloring. To see why, if no agents are assigned to a popular activity of some color $d$, then due to compatibility there must exist at least one agent of the same color who is assigned to the void activity but approves the popular activity for size $1$. Therefore, any stable, compatible assignment must assign $k_d$ agents to a popular activity for color $d$ (if any), but if there exist multiple popular activities of the same color, then no compatible assignment is stable. 
Without loss of generality (by re-coloring) let us assume that colors $[1,q]$ contain exactly one popular activity and colors $[q+1, l]$ contain non-popular activities (it is possible that $q = 0$ or $q = l$). To emphasize, we shall refer to colors in $[1, q]$ as ``popular'' colors and in $[q+1, l]$ as ``unpopular'' colors. 

Let us now examine each color to decide whether we should reject the coloring or whether we can exclude some agents and/or activities from consideration (by re-coloring them as the special color, $l+1$).
First, for each popular color $d\in [1, q]$ with a popular activity $a_{j_d}$ (recall that there is exactly one popular activity for each popular color), we re-color all agents in $(N_d \setminus N(a_{j_d}))$ and all activities in $(A^*_d \setminus \{a_{j_d}\})$ as the special color $(l+1)$ because they cannot be assigned/used in any stable assignment compatible with $c$ as we argued earlier. 
Next, for each unpopular color $d\in [q+1, l]$, if there exist two distinct activities $a_j, a_{j'} \in A^*(d)$ such that $N(a_j) \neq N(a_{j'})$, then we reject the coloring; any compatible assignment must assign no agents to at least one of these two activities (assume that $a_j$ is such activity), but at least one agent in $N(a_j)$ approves $(a_j, 1)$ (due to decreasing preferences) while she must be assigned to the void activity, which implies instability of the assignment. If the coloring is not rejected after these conditions are checked, then we have $N(a_j) = N(a_{j'})$ for all $a_j, a_{j'}\in A^*(d)$ where $d\in [q+1, l]$. Let us re-color all agents in $N(d) \setminus N(a_j)$ as $l+1$ where $a_j$ is any activity in $A^*(d)$ for all $d\in [q+1, l]$.
We then check for another condition for each unpopular color $d\in [q+1,l]$. Let $A'(d) = \{a_j \in A^*(d) : \exists i \in N_{l+1}, u_i(a_j) \geq 1\}$. If $A'(d)$ contains two or more activities, it is clear that the coloring must be rejected because agent $i$ (who cannot be assigned to any activity under the given coloring) approves size $1$ for the activities in $A'(d)$ but the assignment can only choose one activity from $A^*(d)$. Therefore, if $|A'(d)| \geq 2$ then we reject the coloring; otherwise, if $|A'(d)| = 1$, then we re-color all activities in $A^*(d) \setminus A'(d)$ as $l+1$ (as the only one in $A'(d)$ must be used for color $d$). If $|A'(d)| = 0$, this step has no effect for this color. 
Lastly, we now consider the agents of color $l+1$ (who must be assigned to the void activity by any stable assignment compatible with the given coloring).
Let us define $k_{l+1} = 0$ for convenience (i.e., we do not assign any agents of color $l+1$ to any activities).
For each color $d \in [1, l+1]$, if there exists some activity $a_j\in A^*(d)$ and some agent $i\in N(l+1)$ with $u_i(a_j) \geq k_{d} + 1$, then we reject the coloring because agent $i$ is to be assigned to the void-activity, but she approves the outcome $(a_j, k_d+1)$ as well as $(a_j, 1)$ (due to decreasing preferences), which means that regardless of whether $a_j$ is used or not, no assignment would not be stable and compatible at the same time due to agent $i$. 
If the coloring has not been rejected, then we can now safely ignore all agents in $N(l+1)$ (as if they are non-existent) because stability constraint would not be violated by those agents. 

We now proceed with the assumption that the coloring has not been rejected by our algorithm. 
Recall that we need to choose $k_d$ agents among $N_d$ where $d\in [1, q]$ to be assigned to the popular activity $a_{j_d}$ while we know exactly which $k_{d'}$ agents must be assigned to one of the activities in $A^*_{d'}$ where $d'\in [q+1, l]$.
For each popular color $d\in [1, q]$ define $N'_d = 
\{i \in N_d : \exists d'\in [1, l+1], u_i(a_j) \geq k_{d'}+1 \text{~where~} a_j\in A_{d'} \} \cup
\{i \in N_d : \exists d' \in [q+1, l], |\{a_j \in A_{d'} : u_i(a_j) \geq 1\}  | \geq 2\}$.
Any stable assignment compatible with $c$ must assign all agents in $N'_d$ to the popular activity $a_{j_d}$.
Otherwise, if some agent $i$ in $N'_d$ is assigned to the void-activity instead, then the resulting assignment cannot be stable;
if $i$ is contained in the first set (on the right-hand-side of definition of $N'_d$) above, then $i$ approves sizes of both $k_{d'}+1$ and $1$ for some activity $a_j$, which implies that regardless of whether $a_j$ is used or not, $i$ would wish to join $a_j$ instead of $\void$, while if $i$ is contained in the second set (on the right-hand-side of definition of $N'_d$), then $i$ approves size $1$ for at least two non-popular activities of the same color which implies that $i$ would wish to join one of them that is not used. Therefore, if $|N'_d| > k_d$ for some $d \in [1, q]$ we must reject the coloring, and otherwise we must assign all agents in $N'_d$ to $a_{j_d}$.
Without loss of generality we can assume that $N'_d = \emptyset$ for all $d\in [1, q]$ (provided that the coloring is not rejected by this point) by assigning all such agents to the appropriate popular activity and then decreasing $k_d$ by $|N'_d|$ before we proceed to the next step. 

Now suppose that for some color $d\in [1,q]$ and agent $i\in N_d$ and some color $d'\in [q+1, l]$ and some activity $a_j\in A^*_{d'}$, we have $u_i(a_j) \geq 1$. If $i$ is assigned to $\void$ (instead of $a_{d_j}$) and $a_j$ is not used (no agents is assigned to it), then the assignment cannot be stable as $i$ approves $(a_j, 1)$. That is, any compatible, stable assignment must assign $i$ to $a_{d_j}$ and/or use activity $a_j$. If we consider agents in $X = \cup_{d\in [1,q]} N_d$ and activities in $Y = \cup_{d'\in [q+1,l]} A^*_{d'}$ as vertices and there is an edge between $(i, a_j)$ if and only if $u_i(a_j) \geq 1$ where $i\in X$ and $a_j \in Y$ (as a bipartite graph), finding a compatible assignment is equivalent to finding a vertex cover such that it chooses exactly $k_d$ vertices from each $N_d$ with $d\in [1, q]$ and exactly $1$ vertex from each $A^*_{d'}$ with $d'\in [q+1, l]$. Because the total number of vertices to be selected is bounded above by $k+l$, one can use a bounded search tree to determine whether a vertex cover of a small size exists or not in FPT time (i.e., exponential only in $k$ but polynomial in $n,p$). If vertex $i$ from $X$ is chosen then we assign $i$ to the popular activity of the same color and if vertex $a_j$ from $Y$ is chosen then we assign the agents of the same color to it. It is easy to verify that a compatible, stable assignment exists if and only if a vertex cover (with the aforementioned constraints) exists in this bipartite graph. 

While we omit the details, it is straightforward to verify that our algorithm would not reject any coloring $c$ which is compatible with at least one stable assignment.
\end{proof}

Lastly, we consider another special case of \GASPs when all activities are (pairwise) equivalent.
In this case, the problem of finding a stable assignment becomes the problem of partitioning agents into groups where only group sizes matter (as all of the activities are identical to every agent). This restriction allows the problem for an FPT algorithm
\begin{theorem}
$k$-Stable-GASP is in FPT if all (non-void) activities are equivalent.
%
%only one $p$-copyable activity in $A^*$ (where $p = |A^*|$).
\end{theorem}
\begin{proof}
Let $p = |A^*|$ be the number of non-void activities which we assume are all equivalent.
We use Color Coding to design a randomized FPT algorithm, which can easily be de-randomized using a family of $k$-perfect hash functions as shown in the work~\cite{ColorCoding}. 

We can assume that $p \leq k+1$ because $k$ agents can be assigned to at most $k$ copies and having more than one extra copy to which no agents is assigned does not change the problem (this is because we are seeking a solution of size exactly $k$). For brevity, we only prove the claim when $p = k+1$ in this work, but it can be easily extended to the cases when $p < k+1$. 

\paragraph{Preliminaries.}
Let $N = \{1, 2, \dots, n\}$ be the set of $n$ agents and $A^* = \{a_1, a_2, \dots, a_p\}$ be the set of $p$ copies of the only activity when $p = k+1$. Recall that by definition of equivalent activities, every agent $i$ has $S_i(a_j) = S_i(a_{j'})$ for all $j,j'$.
We first fix $l$ (the number of copies of the activity to be used by a stable assignment) which must be between $1$ and $k$, and $k_1, k_2, \dots, k_l$ which is the number of agents assigned to each of the $l$ copies; for convenience we define $k_{l+1} = 0$ as there is at least one extra copy that would not be used by the assignment.
The total number of possible values for $l$ and $k_d$'s are bounded above by $O(k^k)$, which is exponential only in $k$.
After we fix $l$, we color all agents uniformly and independently at random using colors $1$ through $l$; let $c(\cdot)$ be this coloring scheme and $c(i)$ denote the color of agent $i$. 
We say that coloring $c$ (together with $l$ and $k_d$'s) and a stable assignment $\pi$ of size $k$ are {\em compatible} if $\pi$ assigns exactly $k_d$ agents of color $d$ to activity $a_d$ for $d\in [1, l]$.

\paragraph{FPT Algorithm.}
Our algorithm will find a stable assignment compatible with $c$ or determine that no such stable assignment exists. 
Note that any stable assignment of size $k$ has at least one compatible coloring, and the probability that a random coloring we choose is compatible with some fixed stable assignment of size $k$ is at least $(1/l)^k$ (as we must color those $k$ agents correctly) which is exponentially small only in $k$. 

Our algorithm first partitions agents of each color into several subsets, and check several necessary conditions for the coloring to be compatible with at least one stable assignment; if any of the conditions is not met, the coloring will be rejected by the algorithm (as there is no stable assignment compatible with the given coloring). 
For each color $d\in [1, l]$, the algorithm computes three subsets: $N_d = \{i \in N : c(i) = d\}$,
 $N^{\text{IR}}_d = \{i \in N_d : (a_d, k_d) \in S_i \}$, and
 $N^{\text{IN}}_d = \{i \in N_d : \exists d'\in[1, l+1] \text{~s.t.~} (a_{d'}, k_{d'}+1) \in S_i\}$ (recall $k_{l+1} = 0$).
If $|N^{\text{IR}}_d| < k_d$ for any $d\in [1, l]$, no stable assignment is compatible with $c$ because assigning $k_d$ agents to $a_d$ would not be individually rational (i.e., not enough agents approve the outcome), so the coloring should be rejected in this case.
If for some $d \in [1, l]$ the set $N^{\text{IN}}_d - N^{\text{IR}}_d$ is not empty but contains some agent $i$, then no stable assignment is compatible with $c$ because a stable assignment cannot assign $i$ to $a_d$ (because $i\not\in N^{\text{IR}}_d$) but $i$ would wish to join $a_{d'}$ for some $d'\in [1, l+1]$ which would make the assignment not stable; hence the coloring should be rejected in this case. 
If $|N^{\text{IN}}_d| > k_d$, then at least one agent $i$ in $N^{\text{IN}}_d$ should be assigned to the void activity, but $i$ would wish to join $a_{d'}$ for some $d'\in[1, l+1]$ which would make the assignment not stable; hence the coloring should be rejected.
If the coloring is not rejected by any of the cases mentioned earlier, then we have the following three conditions for every color $d\in [1, l]$: (a) $|N^{\text{IR}}_d| \geq k_d$, (b) $|N^{\text{IN}}_d| \leq k_d$, and (c) $N^{\text{IR}}_d \subseteq N^{\text{IN}}_d$.
Let us define $X_d$ for each $d\in [1, l]$ as follows: $X_d$ contains an arbitrary set of $k_d$ agents from $N^{\text{IR}}_d$ such that every agent in $N^{\text{IN}}_d$ is contained in $X_d$. Note that this is always possible due to the three conditions mentioned above. 
We claim that an assignment $\pi$ which assigns agents in $X_d$ to $a_d$ and all other agents to $\void$ is a stable assignment compatible with $c$. To prove compatibility, all agents in $X_d$ are by definition of color $d$ and $|X_d| = k_d$ for all $d\in [1, l]$.
To prove stability, first consider any agent $i$ who is assigned to the void activity and suppose $d = c(i)$. Since $i\not\in X_d$, we know that $i\not\in N^{\text{IN}}_d$ by definition, and therefore there is no $d'\in[1,l+1]$ such that $(a_{d'}, k_{d'}+1)\in S_i$. Now consider any agent $i$ who is assigned to $a_d$ by $\pi$ (thus $c(i) = d$). By definition $i\in X_d$ and thus $i\in N^{\text{IR}}_d$, which implies that $(a_d, k_d)\in S_i$. Therefore $\pi$ is a stable assignment of size $k$, compatible with $c$. 

Let us now prove that if there is at least one stable assignment that is compatible with $c$, then the algorithm does not reject the coloring. 
Let $\pi$ be one such assignment and let $X_d$ be the set of agents assigned to $a_d$ by $\pi$. Due to compatibility we have $|X_d| = k_d$ and $c(i) = d$ for all $i \in X_d$ for all $d\in [1, l]$; in particular, by definition $X_d \subseteq N^{\text{IR}_d}$ and thus the first condition (a) above holds for all $d\in [1, l]$. Due to stability of $\pi$, every agent $i$ with $\pi(i) = \void$ satisfies that $\not\exists d'\in [1, l+1]$ such that $(a_{d'}, k_{d'}+1) \in S_i$. Therefore the conditions (b) and (c) above hold for all $d\in [1, l]$, which proves that the coloring $c$ would not be rejected by the algorithm.

We have shown that if we begin with a coloring $c$ (together with $l$ and $k_d$'s) that is compatible with at least one stable assignment, then our algorithm would find a stable assignment compatible with the coloring and that if no such assignment exists the coloring would be rejected. This is a Monte Carlo algorithm with probability of success at least $(1/k)^k$ and runtime bounded by $O((k^k)nk)$ (as our algorithm must enumerate all possible values of $l$ and $k_d$'s), which is polynomial in $n$ but exponential only in $k$. 
\end{proof}


%%% =============================================================== Envy-free
\subsection{$k$-EF-GASP: Finding Envy-free Assignments}
Recall that $k$-EF-GASP is the problem of finding an envy-free assignment of size $k$. 
Similarly to how we showed that finding a stable assignment is computationally harder than finding an IR assignment under parameterization, we can show that finding an envy-free (EF) assignment is also computationally harder than finding an IR assignment under parameterization; again, this relationship is not apparent under the classic complexity classes (i.e., NP-hardness).

\begin{theorem} \label{GASP:thm:ef_gasp_w1}
$k$-EF-GASP is $W[1]$-complete. 
The problem remains to be $W[1]$-complete even if each agent approves at most one size per activity.
\end{theorem}
\begin{proof} % k-EF-GASP is W[1]-hard.
	We reduce from the $k$-Clique problem. Let $G = (V, E)$ be an undirected graph and $k$ be a parameter of the $k$-Clique instance.

	\paragraph{Construction of \GASPs instance.}
	Let $V = \{v_1, v_2, \dots, v_n\}$, and for each vertex $v_i\in V$, we create activity $a_i$ and $k^2$ copies of $v_i$ as agents.
	For each edge $(v_i, v_j)\in E$, we create an activity $e_{i,j}$ and an agent $w_{i,j}$.
	This creates $|V| + |E|$ activities and $k^2|V| + |E|$ agents overall.
	For each copy of $v_i$, we set its approval set as $S_{v_i} = \{(a_i, k^2)\} \cup \{(e_{i,j}, 1) : (v_i, v_j)\in E\}$ and for each agent $w_{i,j}$ we set its approval set as $S_{w_{i,j}} = (e_{i,j}, 1)$.
	Let $k' = k^3 + \binom{k}{2} = k^3 + k(k-1)/2$.
	We claim that a clique of size $k$ exists in $G$ if and only if an envy-free assignment of size $k'$ exists in the \GASPs instance we created.
	
	\paragraph{Proof of equivalence between instances.}	
	Suppose that $\pi$ is an envy-free assignment of size $k'$. If $\pi$ assigns any copy of $v_i$ to some activity $e_{i,j}$, then $\pi$ cannot be envy-free because $w_{i,j}$ wishes to be assigned to $e_{i,j}$ in place of the copy of $v_i$; furthermore, $\pi$ cannot assign more than one agent to any $e_{i,j}$ as no other agent approves the activity with any other size than $1$. If $\pi$ assigns any copy of $v_i$ to $a_i$, then it must assign all $k^2$ copies of $v_i$ to $a_i$ as those agents only approve $a_i$ with size $k^2$. Because $\pi$ is of size $k'$, it is clear that $\pi$ can only assign agents to at most $k$ different activities of the form $a_i$. 
	Now suppose $\pi$ assigns some $w_{i,j}$ to $e_{i,j}$; due to envy-freeness, all $k^2$ copies of $v_i$ must be assigned to $a_i$ and all $k^2$ copies of $v_j$ must be assigned to $a_j$; this implies that $\pi$ can assign at most $\binom{k}{2}$ agents of the form $w_{i,j}$ to activities of the form $e_{i,j}$ (otherwise, $\pi$ cannot be of size $k'$ because $k^2(k+1) > k'$). Therefore, we conclude that $\pi$ assigns $k^3$ agents of the form $v_i$ to $k$ activities of the form $a_i$ (without loss of generality, assume those activities are $\{a_1, a_2, \dots, a_k\}$) and that $\pi$ assigns $w_{i,j}$ to $e_{i,j}$ if and only if $1 \leq i, j \leq k$ (all other agents are assigned to the void activity). This implies that the original instance contains a clique $C = \{v_1, v_2, \dots, v_k\}$ as there is an edge $(v_i,v_j)$ if $1 \leq i,j \leq k$. 
	
	To prove the converse, suppose that $C = \{v_1, v_2, \dots, v_k\}$ is a clique in the original instance. 
	Let $\pi$ be an assignment such that $\pi$ assigns $k^2$ copies of $v_i$ to $a_i$ if $i \leq k$ and $w_{i,j}$ to $e_{i,j}$ if $1 \leq i,j \leq k$ and assigns all other agents to the void activity. Clearly $\pi$ is individually rational by the construction of approval sets; $\pi$ is also envy-free because no copy of $v_i$ with $i > k$ or $w_{i,j}$ with $i>k$ or $j>k$ wishes to replace any other agent who is assigned to a non-void activity. This completes the proof of $W[1]$-hardness.
	
	Note that in our reduction each agent approves at most one size per activity, proving the second statement in the theorem.

	\paragraph{Proof of completeness.} %TODO
	We now show completeness by reducing $k$-EF-GASP to the colored $k$-clique problem. (details to be added.)
\end{proof}

Although $k$-EF-GASP is $W[1]$-complete, the problem may admit FPT algorithms if we assume that preferences of agents are restricted to increasing preferences or decreasing preferences. 
Unlike the case of stable assignments, we show that $k$-EF-GASP remains to be $W[1]$-complete even if all agents have increasing preferences or all agents have decreasing preferences. 

\begin{theorem}
$k$-EF-GASP is $W[1]$-complete even if all agents have increasing preferences.
\end{theorem}
\begin{proof}[sketch]
	A slight modification to the reduction we used in proof of Theorem~\ref{GASP:thm:ef_gasp_w1} shows that $k$-EF-GASP is $W[1]$-hard even if all agents have increasing preferences. 
	We construct the same instance, but we change the approval set of each agent such that if agent $i$ approves an outcome $(a, x)$ for some activity $a$ and size $x$, then we let the agent approve all outcomes $(a, x')$ with $x < x' \leq |N|$, which ensures that all agents have increasing preferences.
	In addition we create $k'+1$ copies of a dummy agent $z$ such that $z$ approves all outcomes $(e_{i,j}, x)$ with $2 \leq x \leq |N|$ for all activities $e_{i,j}$ we create. 
	
	It is easy to see that if $\pi$ is an envy-free assignment of size $k'$, then $\pi$ cannot assign any copy of $z$ to any activity because if $\pi$ assigns any copy of $z$ to some $e_{i,j}$, then $\pi$ must assign all copies of $z$ to some activity (to avoid envy-freeness with respect to those copies) which results in the size of $\pi$ being at least $k'+1$. 
	This in turn implies that $\pi$ cannot assign more than one agent to any $e_{i,j}$; if two or more agents are assigned to $e_{i,j}$, then all copies of $z$ must be assigned to some activity (or they would be envious) due to increasing preferences. Lastly, this implies that $\pi$ cannot assign any $v_i$ to any $e_{i,j}$ because such assignment implies $w_{i,j}$ must also be assigned to $e_{i,j}$ (due to envy-freeness and increasing preferences), resulting in more than one agent being assigned to $e_{i,j}$. With this observation, the rest of the proof follows in a straightforward manner, and we omit details. 
	
	Note that completeness follows from the fact that $k$-EF-GASP is in $W[1]$.
\end{proof}


\begin{theorem}
$k$-EF-GASP is $W[1]$-complete even if all agents have decreasing preferences.
\end{theorem}
\begin{proof}[sketch]
	A slight modification to the reduction we used in proof of Theorem~\ref{GASP:thm:ef_gasp_w1} shows that $k$-EF-GASP is $W[1]$-hard even if all agents have decreasing preferences. We construct the same instance, but we change the approval set of each agent such that if agent $i$ approves an outcome $(a, x)$ for some activity $a$ and size $x$, then we let the agent approve all outcomes $(a, x')$ with $1 \leq x' < x$, which ensures that all agents have decreasing preferences. The rest of the proof follows in a straightforward manner, and we omit details.
	
	Note that completeness follows from the fact that $k$-EF-GASP is in $W[1]$.
\end{proof}

\begin{theorem}
$k$-Stable-GASP is in FPT if all (non-void) activities are equivalent.
%
%there is only one $p$-copyable activity in $A^*$ (where $p = |A^*|$).
\end{theorem}
\begin{proof}
Let $p=|A^*|$ and let all non-void activities be equivalent.
We use Color Coding to design a randomized FPT algorithm, which can easily be de-randomized using a family of $k$-perfect hash functions as shown in the work~\cite{ColorCoding}. 
Because there are only $n$ agents we can assume that $p \leq k$ because $k$ agents can be assigned to at most $k$ copies (unlike the case of stability, extra copies have no effect in envy-freeness as agents are only envious of others who participate in some activity).
For simplicity we only prove the claim when $p = k$ but it can be easily extended to the cases when $p < k$. 

\paragraph{Preliminaries.}
Let $N = \{1, 2, \dots, n\}$ be the set of $n$ agents and $A^* = \{a_1, a_2, \dots, a_p\}$ be the set of $p$ copies of the only activity when $p = k$. Recall that by definition of equivalent activities, every agent $i$ has $S_i(a_j) = S_i(a_{j'})$ for all $j,j'$.
We first fix $l$ (the number of copies of the activity to be used by an EF assignment) which must be between $1$ and $k$, and $k_1, k_2, \dots, k_l$ which is the number of agents assigned to each of the $l$ copies.
The total number of possible values for $l$ and $k_d$'s are bounded above by $O(k^k)$, which is exponential only in $k$.
After we fix $l$, we color all agents uniformly and independently at random using colors $1$ through $l$; let $c()$ be this coloring scheme and $c(i)$ denote the color of agent $i$. 
We say that coloring $c$ (together with $l$ and $k_d$'s) and an EF assignment $\pi$ of size $k$ are {\em compatible} if $\pi$ assigns exactly $k_d$ agents of color $d$ to activity $a_d$ for $d\in [1, l]$.

Our algorithm will find an EF assignment compatible with $c$ or determine that no such EF assignment exists. 
Note that any EF assignment of size $k$ has at least one compatible coloring, and the probability that a random coloring we choose is compatible with some fixed EF assignment of size $k$ is at least $(1/l)^k$ (as we must color those $k$ agents correctly) which is exponentially small only in $k$. 

\paragraph{FPT algorithm.}
Our algorithm first partitions agents of each color into several subsets, and check several necessary conditions for the coloring to be compatible with at least one EF assignment; if any of the conditions is not met, the coloring will be rejected by the algorithm (as there is no EF assignment compatible with the given coloring). 
For each color $d\in [1, l]$, the algorithm computes three subsets: $N_d = \{i \in N : c(i) = d\}$,
 $N^{\text{IR}}_d = \{i \in N_d : (a_d, k_d) \in S_i \}$, and
 $N^{\text{IN}}_d = \{i \in N_d : \exists d'\in[1, l] \text{~s.t.~} (a_{d'}, k_{d'}) \in S_i\}$.
If $|N^{\text{IR}}_d| < k_d$ for any $d\in [1, l]$, no EF assignment is compatible with $c$ because assigning $k_d$ agents to $a_d$ would not be individually rational (i.e., not enough agents approve the outcome), so the coloring should be rejected in this case.
If for some $d \in [1, l]$ the set $N^{\text{IN}}_d - N^{\text{IR}}_d$ is not empty but contains some agent $i$, then no EF assignment is compatible with $c$ because an EF assignment cannot assign $i$ to $a_d$ (because $i\not\in N^{\text{IR}}_d$) but $i$ would wish to join $a_{d'}$ for some $d'\in [1, l]$ which would make the assignment not envy-free; hence the coloring should be rejected in this case. 
If $|N^{\text{IN}}_d| > k_d$, then at least one agent $i$ in $N^{\text{IN}}_d$ should be assigned to the void activity, but $i$ would wish to join $a_{d'}$ for some $d'\in[1, l]$ which would make the assignment not envy-free; hence the coloring should be rejected.
If the coloring is not rejected by any of the cases mentioned earlier, then we have the following three conditions for every color $d\in [1, l]$: (a) $|N^{\text{IR}}_d| \geq k_d$, (b) $|N^{\text{IN}}_d| \leq k_d$, and (c) $N^{\text{IR}}_d \subseteq N^{\text{IN}}_d$.
Let us define $X_d$ for each $d\in [1, l]$ as follows: $X_d$ contains an arbitrary set of $k_d$ agents from $N^{\text{IR}}_d$ such that every agent in $N^{\text{IN}}_d$ is contained in $X_d$. Note that this is always possible due to the three conditions mentioned above. 
We claim that an assignment $\pi$ which assigns agents in $X_d$ to $a_d$ and all other agents to $\void$ is an EF assignment compatible with $c$. To prove compatibility, all agents in $X_d$ are by definition of color $d$ and $|X_d| = k_d$ for all $d\in [1, l]$.
To prove stability, first consider any agent $i$ who is assigned to the void activity and suppose $d = c(i)$. Since $i\not\in X_d$, we know that $i\not\in N^{\text{IN}}_d$ by definition, and therefore there is no $d'\in[1,l]$ such that $(a_{d'}, k_{d'})\in S_i$. Now consider any agent $i$ who is assigned to $a_d$ by $\pi$ (thus $c(i) = d$). By definition $i\in X_d$ and thus $i\in N^{\text{IR}}_d$, which implies that $(a_d, k_d)\in S_i$. Therefore $\pi$ is an EF assignment of size $k$, compatible with $c$. 

Let us now prove that if there is at least one EF assignment that is compatible with $c$, then the algorithm does not reject the coloring. Let $\pi$ be one such assignment and let $X_d$ be the set of agents assigned to $a_d$ by $\pi$. Due to compatibility we have $|X_d| = k_d$ and $c(i) = d$ for all $i \in X_d$ for all $d\in [1, l]$; in particular, by definition $X_d \subseteq N^{\text{IR}_d}$ and thus the first condition (a) above holds for all $d\in [1, l]$. Due to stability of $\pi$, every agent $i$ with $\pi(i) = \void$ satisfies that $\not\exists d'\in [1, l]$ such that $(a_{d'}, k_{d'}) \in S_i$. Therefore the conditions (b) and (c) above hold for all $d\in [1, l]$, which proves that the coloring $c$ would not be rejected by the algorithm.

We have shown that if we begin with a coloring $c$ (together with $l$ and $k_d$'s) that is compatible with at least one EF assignment, then our algorithm would find an EF assignment compatible with the coloring and that if no such assignment exists the coloring would be rejected. This is a Monte Carlo algorithm with probability of success at least $(1/k)^k$ and runtime bounded by $O((k^k)nk)$ (as our algorithm must enumerate all possible values of $l$ and $k_d$'s), which is polynomial in $n$ but exponential only in $k$. 
\end{proof}

%%% =============================================================== Perfect
\subsection{$k$-Perfect-GASP: Finding Perfect Assignments}
Recall that $k$-Perfect-GASP is the problem of finding a perfect assignment that uses $k$ activities out of $p$ activities. 
Under this parameterization, \GASPs is W[2]-hard as the following theorem shows. 

\begin{theorem} \label{GASP:thm:perfect_gasp_w2_hard}
	$k$-Perfect-GASP is $W[2]$-hard. The problem remains to be $W[2]$-hard even if all agents have increasing preferences or all agents have decreasing preferences. 
\end{theorem}
\begin{proof}
	Consider an instance of the parameterized Dominating Set problem which consists of a graph $G = (V, E)$ and a parameter $k$, and asks whether there exists a dominating set $D$ of size $k$ ($D \subseteq V$ is a dominating set if for every node $v\in V$ either $v \in D$ or $v$ has a neighbor in $D$). This problem is known to be $W[2]$-complete. 
	
	Let us create an instance of \GASPs as follows: Let $N = \{1, 2, \dots, n\}$ and $A^* = \{a_1, a_2, \dots, a_n\}$ where $n = |V|$. For each agent $i$, define $S_i = \{(a_j, x) : ((v_i, v_j) \in E) \land (1 \leq x \leq n) \} \cup \{(a_i, x) : 1 \leq x \leq n\}$. Note that in this instance agents do not care about the number of participants, but the activities only. Finally we set the parameter of \GASPs to be equal to the parameter of the Dominating Set instance. Note that the size of the \GASPs instance we create is polynomial in $n$ and $k$. 

	Let $D$ be a dominating set of size $k$ in the original instance. 
	Let us construct a perfect assignment $\pi$ as follows: 
	For each agent $i$, if $v_i \in D$, then let $\pi(i) = a_i$; if $v_i \not\in D$, then there exists some $v_j \in D$ such that $(v_i, v_j) \in E$ because $D$ is a dominating set, and let $\pi(i) = a_j$. Clearly $\pi$ is a perfect assignment that uses only $k$ activities.
	Conversely, suppose that a perfect assignment $\pi$ exists which uses exactly $k$ activities. Let $A'$ be the set of $k$ activities to which at least one agent is assigned under $\pi$ (note that $|A'| = k$). Let $D = \{v_i : a_i \in A'\}$, and we claim that $D$ is a dominating set in $G$. For any node $v_i \not\in D$, we know that $\pi$ assigns agent $i$ to some activity $a_j \in A'$ where $a_j \neq a_i$, and thus $v_j \in D$. Since $\pi$ is individually rational, it implies that $(v_i, v_j)\in E$, and therefore $D$ is a dominating set. 
	
	Note that in our reduction all agents have increasing preferences as well as decreasing preferences, proving the second statement of the theorem. 
\end{proof}

It is not surprising that $k$-Perfect-GASP is the most complex problem being considered, as it pertains to the strongest solution concept. We do not know the exact complexity of $k$-Perfect-GASP when all activities are equivalent, and it remains to be an open problem (yet the problem is known to be NP-complete by Darmann et al.~\cite{GASP12WINE}).


\section{Discussion} \label{GASP:sec:discussion}
In this work we investigated the parameterized complexity of the Group Activity Selection Problem (\GASP) when parameterized by the size of the solution, for four different solution concepts and under various restrictions on inputs. 
Despite the fact that all problems being considered are NP-hard, we showed that some special cases of the problem admit efficient FPT algorithms.
Our results indicate that the computational complexity of \GASPs varies when its input is restricted (imposed by special preferences of agents or uniformity of activities) or the solution concept changes, which is not distinguishable under the classic complexity.
Our work leaves a few interesting open problems for future work. 
First, we do not know the exact complexity of $k$-Perfect-GASP with equivalent activities besides its NP-completeness. It would be intriguing if the problem is FPT.
Second, the focus in this chapter is to exhibit any FPT algorithm; we have not tried hard to optimize the dependence on $k$. It would be interesting to do this and to show conditional lower bounds on how the runtime should depend on $k$, especially in the case of $k$-IR-GASP. 
Lastly, one can consider a different setting where agents have a strict ordering over the set of outcomes, instead of having approved outcomes that are equally preferred. Darmann recently proved several easiness and hardness results under this setting~\cite{DARMANN15ADT}, but no parameterized complexity results are known yet.





\chapter{Stable Invitations Problem}
%TODO: If accepted at AAAI'17, must reflect suggestions from reviewers, which will affect this chapter as well. For instance, friend graph and enemy graph should be defined early on, instead of being in each proof.

\label{SIP:chapter}

In Chapter~\ref{GASP:chapter} we considered the setting where an event organizer is to assign agents to multiple activities that take place concurrently. In this chapter, we consider a slightly different setting in which there is only one activity -- and thus the organizer is to choose a subset of agents to be ``invited'' to participate in the activity, while agents may exhibit friend or enemy relationships. 
On the one hand, this is a special case of the Group Activity Selection Problem in that there is only one activity. On the other hand, it is a generalization of this specical case in that agents now have non-anonymous preferences -- they do not only care about the number of participants in the activity, but they also care about who those participants are.

Let us consider a concrete example to understand the setting. Suppose that  an event organizer trying to convene an event -- for example, a fundraiser. We assume that the time and venue for the event are fixed,
and that the only remaining decision for the organizer to make is whom
to invite among a set of agents.
An \emph{invitation} is simply defined to be a subset of agents.
The goal of the organizer is to maximize attendance (for example, 
in order to maximize donations), but the potential invitees have 
their own preferences over how many attendees there should be at the event 
and possibly also who the potential attendees should be.
For example, a given donor may not want to attend if too few attendees 
show up, but she may not want the event to be overly crowded. 
Another donor may want to attend the event only
if her friends attend and her business competitor does not.

Clearly, this can be viewed as an instance of the Group Activity Selection Problem (\GASP) with the exception that agents have non-anonymous preferences. 
We call this the Stable Invitations Problem (\SIP), in which each agent 
can specify a set of friends and a set of enemies (in addition to her preference over sizes as in \GASP).
An agent is willing to attend only if all of her friends attend, none of her enemies attends, and  the number of attendees is acceptable to her. 

Not surprisingly, complexity of \SIPs depends highly on the cardinality of friend-sets and enemy-sets, as friends-and-enemies relationship introduces combinatoric complexity in the problem of finding a good solution. 
In this chapter we provide a complete analysis of complexity results on \SIP; we consider individual rationality (IR) and Nash stability as we did for \GASP, and we also consider both asymmetric and symmetric friends-and-enemmies relation. 


\section{Definitions and Notation} \label{SIP:sec:SIP:prelim}

Let us first define the Stable Invitations Problem and related solution concepts. 

\begin{definition} \label{SIP:def:instance}
An instance of the Stable Invitations Problem (\SIP) is given by a set of agents $N = \{1, 2, \dots, n\}$, and an {\em approval set} $S_i \subseteq [1,n]$, a {\em friend set} $F_i \subseteq N$, and an {\em enemy set} $E_i \subseteq N$ for each agent $i\in N$.
We interpret that agent $i$ is willing to attend if all friends in $F_i$ attend, no one in $E_i$ attends, and the number of attendees (including $i$) is acceptable (i.e., it is contained in $S_i$).
\end{definition}

\begin{definition} 
An {\em invitation} $I$ in \SIPs is a subset of agents.

	We say that an invitation $I$ is {\em individually rational} (IR) if for every agent $i\in I$, $|I| \in S_i$, $F_i \subseteq I$, and $R_i \cap I = \emptyset$.
	
	We say that an invitation $I$ is {\em (Nash) stable} if it is individually rational, and if for every agent $j \not\in I$, $|I_j'| \not\in S_j$, $F_j \not\subseteq I_j'$, or $R_j \cap I_j' \neq \emptyset$ where $I_j' = I \cup \{j\}$.
\end{definition}

Individual rationality (IR) requires every invited agent be willing to attend.
Stability further requires those who are not invited be unwilling to participate (without permission of others)
because not all of her friends are attending, some of her enemies are attending, or the number of attendees would be unacceptable. 
We consider the following two problems of finding invitations of size $k$:
\begin{itemize}
	\item $k$-IR-Invitation: $\exists$ IR invitation of size $k$?
	\item $k$-Stable-Invitation: $\exists$ stable invitation of size $k$?
\end{itemize}

As we shall see, both problems are NP-hard even under some restrictions on inputs of the problems. Yet, as we mentioned in Chapter~\ref{GASP:chapter}, NP-hardness is not the end of hardness, and we focus on parameterized complexity of these problems in this chapter. 

There are two kinds of restrictions on inputs we will consider in this work.
First, we consider inputs in which the sizes of largest friend sets and enemy sets is bounded above by constants; intuitively, the smaller the constants are, the easier the problems become. Second, we consider the cases where friend relationship and enemy relationship are symmetric (i.e., A is a friend of B if and only if B is a friend of A and X is an enemy of Y if and only if Y is an enemy of X). 

We first consider restrictions on inputs by limiting the size of largest friend-sets and enemy-sets, respectively. 
For integer constants $\alpha$ and $\beta$, if an instance of \SIPs satisfies $|F_i| \leq \alpha$ and $|E_i| \leq \beta$ for all $i\in N$, we call it an $(\alpha,\beta)$-instance of \SIP.
We show that that \SIPs can be solved in polytime only if $\alpha$ and $\beta$ are small enough, but the problems are NP-hard in general (this is summarized in Table~\ref{SIP:tbl:prelim}).
We then investigate parameterized complexity of the problem under the same restrictions when the size of a solution is parameterzed. 
Lastly, we will consider a special case of \SIPs where agents have symmetric social relationships.
\begin{definition} \label{SIP:def:symmetric_social}
	Given an instance of \SIP, we say that agents have {\em symmetric social relationships} if $j\in F_i$ if and only if $i\in F_j$ and $l \in E_i$ if and only if $i \in E_l$ for every $i$. 
\end{definition}

Theorem~\ref{SIP:thm:nphard} summarizes the most relevant results of Darmann et al.~\cite{GASP12WINE} and Lee and Shoham~\cite{LEE15AAAI} on complexity of \SIPs.\footnote{
Darmann et al.~\cite{GASP12WINE} showed easiness when $\alpha=\beta=0$, while Lee and Shoham~\cite{LEE15AAAI} proved easiness and hardness in all other cases.}
\begin{theorem} \label{SIP:thm:nphard} [Lee and Shoham~\cite{LEE15AAAI} and Darmann et al.~\cite{GASP12WINE}]
	$k$-IR-Invitation and $k$-Stable-Invitation can be solved in polynomial time if $(\max_{i \in N} |F_i|) + (\max_{i \in N} |E_i|) \leq 1$ (i.e., $\alpha + \beta \leq 1$). In other cases, both problems are NP-hard.
\end{theorem}
Note that $k$-IR-Invitation and $k$-Stable-Invitation are of the same classical complexity, even though stability is a stronger solution concept. Under parameterization, however, these two problems are contained in different complexity classes in the W-hierarchy (see Table~\ref{SIP:tbl:summary}).
In what follows, we show that the parameterized complexity of these problems varies with different solution concepts and under different restrictions on inputs to \SIP.


\section{Classical Complexity} \label{SIP:sec:prelim}

 \begin{table*}[t!]
	 \small
 	\centering
 \begin{tabular}{|l|*{3}{c|}|*{3}{c|}}\hline
 \multirow{2}*{} & \multicolumn{3}{c||}{$k$-IR-Invitations} & \multicolumn{3}{c|}{$k$-Stable-Invitations} \\ \cline{2-7}
  & $\beta = 0$ & $\beta = 1$ & $2 \leq \beta$ & $\beta = 0$ & $\beta = 1$ & $2 \leq \beta$ \\ \hline
 $\alpha = 0$ & P & P & NP-C  &  P & P & NP-C  \\ \hline
 $\alpha = 1$ & P & NP-C  & NP-C & P & NP-C & NP-C \\ \hline
 $\alpha \geq 2$ & NP-C & NP-C & NP-C & NP-C & NP-C & NP-C  \\ \hline
 \end{tabular}
 \caption{\small Complexity of $k$-IR-Invitation and $k$-Stable-Invitation.   
Entries with ``NP-C'' indicate their NP-completeness. 
 }
 \label{SIP:tbl:prelim}
 \end{table*}

Earlier we mentioned that both $k$-IR-Invitation and $k$-Stable-Invitation are NP-hard in general. In particular, we show that these two problems are NP-hard if $\alpha + \beta > 1$ and they are poly-time solvable if $\alpha + \beta \leq 1$. In particular, when $\alpha = \beta = 0$, the two problems coincide with the Group Activity Selection Problem (with the respective solution concept), and they are known to be solvable in polynomial time (Darmann et al.~\cite{GASP12WINE}). The classical complexity of these problems is summarized in Table~\ref{SIP:tbl:prelim}. 

\subsection{$k$-IR-Invitation}

Let us present easiness results when $\alpha + \beta \leq 1$, followed by hardness results when $\alpha + \beta > 1$. 

\begin{theorem} \label{SIP:thm:IR_invitation_P}
	$k$-IR-Invitation can be solved in polynomial time when $\alpha + \beta \leq 1$.
\end{theorem} 
\begin{proof}
We know from the work of Darmann et al.~\cite{GASP12WINE} that $k$-IR-Invitation is solvable in polynomial time when $\alpha = \beta = 0$.
	
	Let us first consider $(1, 0)$-instances of $k$-IR-Invitation in which all agents have at most one friend and all agents have no enemies. Without loss of generality, we can assume that every agent approves size $k$. Otherwise, such agents cannot be included in the solution without violating IR conditions. If we remove an agent who does not approve size $k$, then all agents who consider this agent as friends must also be removed as they cannot be included in the solution. This removal process can be done in polynomial time (in fact, in time $O(n)$), and we assume all agents approve size $k$ from here on. 
	
	Let us construct a (directed) ``friend graph'' ($G_f$) in which nodes represent agents and edges represent (directed) friend relationship such that there is an edge from node $i$ to node $j$ if and only if $j \in F_i$. Clearly, there is a one-to-one correspondence between an invitation and a subset of nodes -- we shall interchangeably refer to a subset of agents or a subset of nodes as an IR invitation. Because individual rationaliy requires that all friends of every invited agent are also invited, it is clear that every strongly connected component in $G_f$ must all or none be included in an IR invitation. Hence, we can construct a directed acyclic (weighted) graph $G'$ in which each node corresponds to a strongly connected component of $G_f$ and the weight of the node is equal to the number of ndoes in the component. The problem of finding an IR invitation of size $k$ is now equivalent to finding a subset of weighted nodes in $G'$ whose sum of weights is equal to $k$ and there is no outgoing edge from the subset. 

TODO: To be added.
\end{proof}




\begin{theorem} \label{SIP:thm:IR_invitation_NPC}
	$k$-IR-Invitation is NP-complete when $\alpha + \beta > 1$.
\end{theorem} 
\begin{proof} %TODO
	To be added.
	
	The proofs are available from AAAI'15 submission (supplemental section).
	
	May need to change notation.
	
Lorem ipsum dolor sit amet, consectetur adipiscing elit. Integer purus nunc, mollis vitae vestibulum et, sagittis a mauris. Donec aliquam commodo condimentum. Pellentesque ac odio ornare, sagittis turpis eu, finibus tellus. Cras maximus est id condimentum aliquet. Proin rhoncus eros vitae aliquet dictum. Quisque eget rutrum nisi. Curabitur rutrum ante in vestibulum placerat. Ut consectetur libero risus, quis bibendum ligula fermentum a. Nullam nec arcu non sapien molestie mattis in non orci. Nulla venenatis hendrerit luctus. Donec dictum non augue id congue. Vestibulum arcu purus, aliquam eu libero vitae, ornare ornare elit. Quisque viverra venenatis ex, quis fermentum justo pretium porttitor. Curabitur condimentum porttitor massa. Maecenas faucibus augue ut blandit posuere. Integer sed dui mollis, suscipit justo eu, blandit arcu.

Cras a accumsan tortor, ut rutrum lectus. In hac habitasse platea dictumst. Sed erat velit, varius at consectetur non, placerat non enim. Mauris viverra consectetur libero sed feugiat. Aliquam porttitor facilisis turpis non pharetra. Nam hendrerit, nulla vel fermentum tincidunt, odio dolor posuere mauris, quis euismod ex nisi aliquam tortor. Vivamus ipsum quam, ornare quis efficitur eget, scelerisque in metus. Nullam in metus vitae lacus ornare tempor in eget tortor. Aliquam id quam rutrum, posuere odio et, condimentum diam. Phasellus non lacus sit amet arcu rhoncus euismod eget at ex. Nulla sed ante rutrum, condimentum orci et, suscipit mauris.

Pellentesque porttitor efficitur leo. Nam sed risus urna. Integer mollis interdum est, quis ultrices ligula iaculis elementum. Cras elementum sem non ex condimentum, in dignissim lacus tincidunt. In a odio a lorem faucibus laoreet. Fusce auctor pellentesque blandit. Suspendisse ornare quam interdum, luctus lacus vel, tristique nulla. Curabitur tincidunt est sed velit euismod sodales. Mauris euismod, sem sed condimentum aliquet, nunc nibh fermentum ex, nec scelerisque justo lorem at risus. Integer vitae risus accumsan odio tincidunt ullamcorper. Vestibulum feugiat ultricies nisl, a tincidunt felis ultrices ac. Proin lobortis in tortor quis dignissim. Curabitur nec ipsum posuere purus porttitor tristique. Maecenas non tortor sollicitudin, finibus erat tincidunt, auctor urna. Vivamus ornare sapien a bibendum maximus. Lorem ipsum dolor sit amet, consectetur adipiscing elit.

Integer vitae vulputate mauris. Nulla sit amet felis vel quam dignissim vestibulum vitae sed nibh. Mauris malesuada, odio at rhoncus interdum, neque nunc pretium tellus, et feugiat magna leo vel libero. Maecenas ut odio eros. Fusce eu arcu augue. Nulla facilisi. Proin iaculis eu justo dapibus lacinia. Maecenas efficitur felis non convallis tempor. Curabitur fermentum, elit in feugiat auctor, quam erat pharetra nisl, at efficitur libero lorem in urna. Maecenas nec luctus risus, sodales mattis diam. Aliquam rutrum nulla sed suscipit viverra. Class aptent taciti sociosqu ad litora torquent per conubia nostra, per inceptos himenaeos. Quisque vestibulum quis purus sit amet pharetra. Fusce vitae porta quam. Cras ut vehicula nulla. Maecenas at lobortis massa.

Donec feugiat vel tortor non mollis. Fusce in nisl mauris. Maecenas mi velit, scelerisque sed velit quis, accumsan fermentum augue. Nulla scelerisque arcu a ipsum suscipit, ac convallis turpis gravida. Nulla mattis aliquet nunc in venenatis. Sed eget posuere nibh, nec facilisis velit. Nulla in lorem scelerisque, imperdiet ipsum nec, placerat neque. Fusce commodo venenatis leo non tincidunt. Integer ante urna, sagittis placerat diam quis, lacinia consectetur odio.	
\end{proof}


\subsection{$k$-IR-Stable}

Let us present easiness results when $\alpha + \beta \leq 1$, followed by hardness results when $\alpha + \beta > 1$. 

\begin{theorem} \label{SIP:thm:stable_invitation_P}
	$k$-Stable-Invitation can be solved in polynomial time when $\alpha + \beta \leq 1$.
\end{theorem} 
\begin{proof} %TODO
	To be added.
\end{proof}

\begin{theorem} \label{SIP:thm:stable_invitation_NPC}
	$k$-Stable-Invitation is NP-complete when $\alpha + \beta > 1$.
\end{theorem} 
\begin{proof} %TODO
	To be added.
	
	The proofs are available from AAAI'15 submission (supplemental section).
	
	May need to change notation.

Lorem ipsum dolor sit amet, consectetur adipiscing elit. Integer purus nunc, mollis vitae vestibulum et, sagittis a mauris. Donec aliquam commodo condimentum. Pellentesque ac odio ornare, sagittis turpis eu, finibus tellus. Cras maximus est id condimentum aliquet. Proin rhoncus eros vitae aliquet dictum. Quisque eget rutrum nisi. Curabitur rutrum ante in vestibulum placerat. Ut consectetur libero risus, quis bibendum ligula fermentum a. Nullam nec arcu non sapien molestie mattis in non orci. Nulla venenatis hendrerit luctus. Donec dictum non augue id congue. Vestibulum arcu purus, aliquam eu libero vitae, ornare ornare elit. Quisque viverra venenatis ex, quis fermentum justo pretium porttitor. Curabitur condimentum porttitor massa. Maecenas faucibus augue ut blandit posuere. Integer sed dui mollis, suscipit justo eu, blandit arcu.

Cras a accumsan tortor, ut rutrum lectus. In hac habitasse platea dictumst. Sed erat velit, varius at consectetur non, placerat non enim. Mauris viverra consectetur libero sed feugiat. Aliquam porttitor facilisis turpis non pharetra. Nam hendrerit, nulla vel fermentum tincidunt, odio dolor posuere mauris, quis euismod ex nisi aliquam tortor. Vivamus ipsum quam, ornare quis efficitur eget, scelerisque in metus. Nullam in metus vitae lacus ornare tempor in eget tortor. Aliquam id quam rutrum, posuere odio et, condimentum diam. Phasellus non lacus sit amet arcu rhoncus euismod eget at ex. Nulla sed ante rutrum, condimentum orci et, suscipit mauris.

Pellentesque porttitor efficitur leo. Nam sed risus urna. Integer mollis interdum est, quis ultrices ligula iaculis elementum. Cras elementum sem non ex condimentum, in dignissim lacus tincidunt. In a odio a lorem faucibus laoreet. Fusce auctor pellentesque blandit. Suspendisse ornare quam interdum, luctus lacus vel, tristique nulla. Curabitur tincidunt est sed velit euismod sodales. Mauris euismod, sem sed condimentum aliquet, nunc nibh fermentum ex, nec scelerisque justo lorem at risus. Integer vitae risus accumsan odio tincidunt ullamcorper. Vestibulum feugiat ultricies nisl, a tincidunt felis ultrices ac. Proin lobortis in tortor quis dignissim. Curabitur nec ipsum posuere purus porttitor tristique. Maecenas non tortor sollicitudin, finibus erat tincidunt, auctor urna. Vivamus ornare sapien a bibendum maximus. Lorem ipsum dolor sit amet, consectetur adipiscing elit.

Integer vitae vulputate mauris. Nulla sit amet felis vel quam dignissim vestibulum vitae sed nibh. Mauris malesuada, odio at rhoncus interdum, neque nunc pretium tellus, et feugiat magna leo vel libero. Maecenas ut odio eros. Fusce eu arcu augue. Nulla facilisi. Proin iaculis eu justo dapibus lacinia. Maecenas efficitur felis non convallis tempor. Curabitur fermentum, elit in feugiat auctor, quam erat pharetra nisl, at efficitur libero lorem in urna. Maecenas nec luctus risus, sodales mattis diam. Aliquam rutrum nulla sed suscipit viverra. Class aptent taciti sociosqu ad litora torquent per conubia nostra, per inceptos himenaeos. Quisque vestibulum quis purus sit amet pharetra. Fusce vitae porta quam. Cras ut vehicula nulla. Maecenas at lobortis massa.

Donec feugiat vel tortor non mollis. Fusce in nisl mauris. Maecenas mi velit, scelerisque sed velit quis, accumsan fermentum augue. Nulla scelerisque arcu a ipsum suscipit, ac convallis turpis gravida. Nulla mattis aliquet nunc in venenatis. Sed eget posuere nibh, nec facilisis velit. Nulla in lorem scelerisque, imperdiet ipsum nec, placerat neque. Fusce commodo venenatis leo non tincidunt. Integer ante urna, sagittis placerat diam quis, lacinia consectetur odio.
\end{proof}

It is interesting that, even though individual rationality is a much weaker solution concept than stability, classical complexity results are identical under all cases we considered in this work (summarized in Table~\ref{SIP:tbl:prelim}). However, we shall see in Section~\ref{SIP:sec:results} that the two solution concepts differ in many cases when we consider parameterized complexity. 




\section{Parameterized Complexity} \label{SIP:sec:results}

 \begin{table*}[t!] 
	 \small
 	\centering
 \begin{tabular}{|l|*{4}{c|}|*{4}{c|}}\hline
 \multirow{2}*{} & \multicolumn{4}{c||}{$k$-IR-Invitations} & \multicolumn{4}{c|}{$k$-Stable-Invitations} \\ \cline{2-9}
  & $\beta = 0$ & $\beta = 1$ & $2 \leq \beta \leq f(k)$ & $\beta<n$ & $\beta = 0$ & $\beta = 1$ & $2 \leq \beta \leq f(k)$ & $\beta<n$ \\ \hline
 $\alpha = 0$ & P & P & FPT  & W[1]-C & P & P & FPT & W[2]-C \\ \hline
 $\alpha = 1$ & P & FPT  & FPT & W[1]-C & P & W[1]-C & W[1]-C & W[2]-C\\ \hline
 $\alpha \geq 2$ & W[1]-C & W[1]-C & W[1]-C & W[1]-C & W[1]-C & W[1]-C & W[1]-C & W[2]-C \\ \hline
 \end{tabular}
 \caption{\small Complexity of $k$-IR-Invitation and $k$-Stable-Invitation. $f(k)$ can be an arbitrary function of $k$ that only depends on $k$.
 All entries other than ``P'' imply NP-completeness.
  ``W[1]-C'' and ``W[2]-C'' mean W[1]-completeness and W[2]-completeness, respectively. 
  Note that P and NP-completess results were presented in Section~\ref{SIP:sec:prelim}.
  }
 \label{SIP:tbl:summary}
 \end{table*}


In this section, we study parameterized complexity of $k$-IR-Invitation and $k$-Stable-Invitation.
Our main contributions are summarized in Table~\ref{SIP:tbl:summary}. For instance, finding an IR invitation of size $k$ is in FPT when $\alpha = 1$ and $\beta$ is a positive constant (bounded above by some function of $k$), but finding a stable invitation in the same cases is W[1]-complete. 
 

\subsection{$k$-IR-Invitation}

Recall that $k$-IR-Invitation is the problem of finding an IR invitation of size $k$.
When $\alpha + \beta > 1$, the problem is known to be NP-hard (Theorem~\ref{SIP:thm:nphard}). 
We first present easiness results: $k$-IR-Invitation is in W[1] in general, and it is in FPT if $\alpha \leq 1$ and $\beta$ is bounded by some function $f(k)$ of $k$.
We then present hardness results by showing that $k$-IR-Invitation is W[1]-hard when $\alpha \geq 2$ and/or $\beta$ is unbounded.


\begin{theorem} \label{SIP:thm:IR_invitation_W1}
	$k$-IR-Invitation is in W[1].
\end{theorem} 
\begin{proof}
	We reduce $k$-IR-Invitation to the weighted circuit SAT (WCSAT) of constant depth and of weft at most 1. 
	
	Details of the proof can be found in proof of Theorem~\ref{SIP:thm:stable_W2} which appears later in this chapter.
\end{proof}


\begin{theorem} \label{SIP:thm:IR_invitation_FPT}
	$k$-IR-Invitation is in FPT if $\alpha \leq 1$ and $\beta \leq f(k)$ where $f(k)$ can be an arbitrary function of $k$. 
\end{theorem}
\begin{proof}
	Without loss of generality, assume that $k\in S_i$ for all $i \in N$.
	Otherwise, we can remove $i$ from the input instance as no IR invitation of size $k$ can contain $i$. If $i$ is removed, and there is some $j$ with $i \in F_j$, we remove $j$ as well for the same reason. We repeat this removal process until no such agent remains (this can be done in linear time).

	Let $\mathcal{A}$ be some polytime algorithm that solves $k$-IR-Invitation if $\alpha \leq 1$ and $\beta = 0$ (it exists due to Theorem~\ref{SIP:thm:nphard}). We will use $\mathcal{A}$ as a sub-routine in our FPT algorithm. Consider any coloring $c$ which colors agents using two colors $\{0,1\}$; let $c(i) \in \{0,1\}$ be the color of agent $i$.
	We say that coloring $c$ and IR invitation $I$ of size $k$ are {\em compatible} if the following holds: For every agent $i\in I$, $c(i) = 1$ and for every agent $j \in \cup_{i: i\in I} E_i$, $c(j) = 0$. 
	Note that coloring $c$ may be compatible with any number of IR invitations of size $k$ (possibly none), and any IR invitation of size $k$ may be compatible with many colorings (but it is compatible with at least one coloring).
	
	Given some arbitrary coloring $c$, we can find an IR invitation of size $k$ that is compatible with $c$ or determine that no compatible IR invitation exists in FPT time as follows.
	First, we re-color every agent $i$ with $c(i)=1$ to color $0$ such that $\exists j\in F_i$ with $c(j)=0$ or $\exists l \in E_i$ with $c(l) = 1$ (order in which we re-color agents does not matter).
	Notice that this process does not re-color any agent $i\in I$ if $I$ is compatible with $c$. After the re-coloring step, let $N_1 = \{i\in N: c(i) = 1\}$, and we run the algorithm $\mathcal{A}$ on $N_1$ as input. Suppose that $\mathcal{A}$ finds an IR invitation $I$ of size $k$ given $N_1$. $I$ is individually rational because its friend constraints are satisfied (due to correctness of $\mathcal{A}$) and its enemy constraints are satisfied because no agent with color $0$ is included in $N_1$ (enforced by coloring). Now suppose that $\mathcal{A}$ reports that no IR invitation $I$ of size $k$ exists among the agents in $N_1$. Then there is no IR invitation of size $k$ that is compatible with $c$; if such invitation $I' \subseteq N_1$ exists, then $I'$ satisfies the friend constraints (because it is IR) and therefore $\mathcal{A}$ should find it, which is a contradiction. Hence if our algorithm begins with coloring that is compatible with some IR invitation(s), it will find one.

	If we color agents uniformly and independently at random, then the probability of success of our algorithm is at least $1/2^{(k+1)\beta}$ (because, with respect to some fixed IR invitation $I^*$, we must color all agents in $I^*$ as 1 and the union enemies of agents in $I^*$ as 0, to start with compatible coloring). If we run this algorithm $2^{(k+1)\beta}\ln n$ times, the probability of success is at least $1 - 1/n$. Our FPT algorithm's runtime depends on the runtime of $\mathcal{A}$. The algorithm can be de-randomized using a family of $k$-perfect hash functions as shown in the work by Alon et al.~\cite{ColorCoding}.
\end{proof}


\begin{theorem} \label{SIP:thm:IR_invitation_large_beta}
	$k$-IR-Invitation is W[1]-complete if $\beta$ is not bounded above by any function $f(k)$. 
\end{theorem}
\begin{proof}
	We reduce from the $k$-Independent-Set problem which is known to be W[1]-complete. 
	Given an arbitrary graph $G = (V, E)$ and a parameter $k$, we create agents $N = V = \{v_1, v_2, \dots, v_n\}$. 
	For each $v_i$, define $S_{v_i} = \{k\}$, $F_{v_i} = \emptyset$, and $E_{v_i} = \{v_j : (v_i, v_j)\in E\}$ (hence $\beta$ is equal to the max-degree of nodes in $G$). 
	If $I \subset V$ is an independent set of size $k$, then $I$ is an IR invitation in the instance we created: For all $v_i \in I$, we have $|I| = k \in S_{v_i}$, $F_{v_i} = \emptyset \subset I$, and $E_{v_i} \cap I = \emptyset$ because $I$ is an independent set in the original graph.
	Conversely, suppose $I$ is an IR invitation of size $k$ in the instance we created. Then $I$ is an independent set because no two agents in $I$ are enemies of each other, and thus their corresponding nodes in the graph are not neighbors of each others. This reduction proves W[1]-hardness, and W[1]-completenes follows from Theorem~\ref{SIP:thm:IR_invitation_W1}.
\end{proof}

\begin{theorem} \label{SIP:thm:IR_invitation_alpha2}
	$k$-IR-Invitation is W[1]-complete if $\alpha \geq 2$.
\end{theorem}
\begin{proof}	We reduce from the $k$-Clique problem.
	Given an arbitrary graph $G = (V, E)$ and a parameter $k$, we create a set of agents $N$ as follows. 
	For each node $v_i\in V$, we create $k^2$ node-agents that are labeled as $w_{i,x}$ where $x \in [k^2]$. 
	For each node-agent $w_{i,x}$ we define $F_{w_{i,x}} = \{w_{i,x+1}\}$ (where $w_{i,k^2+1}$ is understood as $w_{i,1}$) and $E_{w_{i,x}} = \emptyset$. 
	Note that an IR invitation must include all or none of the $w_{i,x}$'s for each $i$ because of their friend sets.
	Next, for each edge $(v_i, v_j) \in E$, we create an edge-agent $e_{i,j}$ with $F_{e_{i,j}} = \{w_{i,1}, w_{j,1}\}$ and $E_{e_{i,j}} = \emptyset$. 
	Note that if an IR invitation includes $e_{i,j}$, then it must also include all $2k^2$ node-agents of the form $w_{i,x}$ and $w_{j,x}$ with $x\in[k^2]$ (due to friend sets).
	
	Finally, define $k' = k^3 + \binom{k}{2}$ to be the parameter for the $k$-IR-Invitations we created, and define approval sets of all agents to contain $k'$. 
	Clearly the instance we created satisfies $\alpha = 2$ and $\beta = 0$. 
	The number of agents we created is $k^2|V| + |E|$, polynomial in the size of the original instance.
	It remains to show that a clique of size $k$ exists if and only if an IR invitation of size $k' = k^3 + \binom{k}{2}$ exists.

	If $C$ is a clique of size $k$ in the original graph, define $I = \{w_{i, x}: v_i\in C, x\in[k^2]\} \cup \{e_{i,j} : v_i, v_j \in C\}$. Clearly, $|I| = k^3 + \binom{k}{2}$ because $|C| = k$. To see that $I$ is individually rational, every agent in $I$ approves size $k^3 + \binom{k}{2}$, every agent included in $I$ has their friends included in $I$.  

	Conversely, suppose that $I$ is an IR invitation of size $k^3 + \binom{k}{2}$ in the instance we created.
	Let us define $G_i = \{w_{i,x} : x\in [k^2]\}$ to be the set of node-agents corresponding to $v_i$ in the original instance (informally, $G_i$ is a cycle of length $k^2$ representing node $v_i$). Recall that $I$ must contain all node-agents or none in $G_i$ for each $i$ if $I$ is individually rational. If $I$ contains more than $k$ such sets, then $|I| \geq (k+1)k^2 > k^3 + \binom{k}{2}$, and therefore $I$ can contain at most $k$ such sets, which means that $I$ must contain at least $\binom{k}{2}$ edge-agents (because $|I| = k^3 + \binom{k}{2}$). 
	Now suppose that $I$ contains more than $\binom{k}{2}$ edge-agents. Then $I$ must include at least $k+1$ of $G_i$'s because each edge-agent $e_{i,j}$ in $I$ requires both $G_i$ and $G_j$ to be included in $I$, and containing more than $\binom{k}{2}$ edge-agents in $I$ implies that at least $k+1$ sets of node-agents are included in $I$ which we argued leads to a contradiction. That is, $I$ can contain at most $\binom{k}{2}$ edge-agents. Together with the previous argument, this means that $I$ contains exactly $\binom{k}{2}$ edge-agents and exactly $k$ sets of node-agents. Because $e_{i,j} \in I$ implies that there is an edge $(v_i, v_j)$ in the original instance, this implies that a clique of size $k$ exists in the original instance. 

	
	Note that W[1]-completenes follows from Theorem~\ref{SIP:thm:IR_invitation_W1}.
\end{proof}








\subsection{$k$-Stable-Invitation}
 
$k$-IR-Invitation and $k$-Stable-Invitation have the same classical complexity for all values of $\alpha$ and $\beta$, but parameterization indicates that $k$-Stable-Invitation is a more difficult problem than $k$-IR-Invitation.
This is not surprising because a stable invitation requires that everyone (whether invited or not) be satisfied with the invitation.

% thm:stable_W2
\begin{theorem} \label{SIP:thm:stable_W2}
	$k$-Stable-Invitation is in W[2]. When $\beta$ is bounded above by some function $f(k)$, it is in W[1].
\end{theorem}
\begin{proof}
	Assume unbounded $\alpha$ and $\beta$, and let us reduce the $k$-Stable-Invitation problem to the weighted circuit SAT (WCSAT) of constant depth and of weft at most 2. This shows that the problem is in W[2]; along the way, we prove the other statements as well.

	Given an instance of the $k$-Stable-Invitation problem with $n$ agents ($N = \{1, 2, \dots, n\}$), we create $n$ input nodes which represent each agent being invited or not; let $x_i$ be the input node for agent $i$ ($x_i = 1$ means that agent $i$ is included in an invitation). For convenience we also create a node $y_i$ for each $x_i$ where $y_i$ is the NOT gate attached to $x_i$. We create the output AND gate denoted by $x_o$. 
	
	For each agent $i$, if $k\not\in S_i$ and/or $|F_i| \geq k$, then agent $i$ cannot be included in a stable invitation of size $k$; for all such $i$, we create a path from the node $x_i$ to $y_i$ (recall that $y_i$ simply negates $x_i$) to the output node $x_o$.
	For each agent $i$ with $k \in S_i$ and $|F_i| < k$, we wish to ensure that if agent $i$ is included, then everyone in $F_i$ is included and no one in $R_i$ is included. For each agent $j\in F_i$, we create a new node $f_{i,j}$ with the OR gate whose input is $y_i$ and $x_j$, and the output node $x_o$ takes $f_{i,j}$ as input. Note that a solution to the WCSAT instance must have $f_{i,j} = 1$ for all $i,j$ (if the node $f_{i,j}$ is created), which holds if and only if $(x_i, x_j) \in \{(0,0), (0,1), (1,1)\}$. This ensures that if $i$ is selected, all agents in $F_i$ are also selected. Note that each path containing $f_{i,j}$ to the output node has a constant depth and a weft of $1$ (due to the output node). Similarly, for each $j\in R_i$ with $k\in S_i$ and $|F_i| < k$, we create a new node $r_{i,j}$ with the OR gate whose input is $x_i$ and $y_j$, and the output node $x_o$ takes $r_{i,j}$ as input. 
	So far we have implemented a circuit such that any solution to the WCSAT instance corresponds to an IR invitation of size $k$. Every path from any input node to the output node has constant depth and weft 1 (due to the output node), and this shows that $k$-IR-Invitation is in W[1]. 
	
	To ensure stability in addition to individual rationality, we need to add more gates and paths. 	
	For each agent $i$ with $k+1 \in S_i$, if agent $i$ is not included in an invitation, then the invitation is stable only if at least one agent in $F_i$ is not included and/or at least one agent in $R_i$ is included.
	Note that if $|F_i| > k$, then every solution to the WCSAT leaves out at least one agent from $F_i$, and therefore agent $i$ (even if excluded) would not be willing to join the invitation. Therefore, we only need to ensure stability for those agents with $k+1\in S_i$ and $|F_i| \leq k$. 
	For each such agent $i$, let us create a new node $s_i$ with the OR gate whose input consists of all nodes $y_j$ where $j\in F_i$, all nodes $x_j$ where $j \in R_i$, and $x_i$ itself. Note that $s_i = 1$ if $x_i = 1$ (when agent $i$ is selected) or if $x_i = 0$ and at least one friend (enemy) of agent $i$ is not included (included). Notice that any path containing $s_i$ has a constant depth but a weft of size 2 due to $s_i$ and the output node. This shows that $k$-Stable-Invitation is in W[2].
	
	When $\beta$ is bounded by $f(k)$ (any constant not depending on $n$), we can create a chain of nodes to implement $s_i$. Specifically, instead of having $s_i$ to take $(|F_i| + |R_i| + 1)$ inputs, we create a chain of nodes with the OR gate, each of which takes only two inputs. Although this increases the depth of a path containing these nodes to $(|F_i| + |R_i| + 1)$, the depth is bounded by $(k + f(k) + 1)$ because $|F_i| \leq k$. Therefore, when $\beta$ is bounded, the $k$-Stable-Invitation is in W[1].	
\end{proof}

\begin{theorem} \label{SIP:thm:stable_FPT}
	$k$-Stable-Invitation is in FPT when $\alpha = 0$ and $\beta \leq f(k)$ where $f(k)$ can be an arbitrary function of $k$.
\end{theorem}
\begin{proof}
	The main idea is similar to that of our proof of Theorem~\ref{SIP:thm:IR_invitation_FPT}. 

	Consider any coloring $c$ which colors agents using two colors $\{0, 1\}$; let $c(i)\in \{0,1\}$ be the color of agent $i$. Let $I$ be a stable invitation of size $k$. We say that $c$ and $I$ are {\em compatible} if the following holds: For every agent $i\in I$, $c(i) = 1$ and for every agent $j\in \cup_{i: i\in I} R_i$, $c(j) = 0$; other agents can be of any color. 
	Given an arbitrary coloring $c$, we can find a stable invitation of size $k$ that is compatible with $c$ or determine that no compatible invitation exists in FPT time. 

	Given $c$, we first re-color some of the agents if necessary as follows. If there is some agent $i$ with $c(i) = 1$ and $k\not\in S_i$, then we re-color agent $i$ as $c(i) = 0$ as no stable invitation can include $i$. If there is some agent $i$ with $c(i) = 1$ and $\exists j\in R_i$ such that $c(j) = 1$, then we re-color agent $i$ as $c(i) = 0$ as no compatible invitation $I$ can include $i$ in it. We repeat these re-coloring steps until no such agents exist (which happens at most $O(|N|)$ times). 
	
	After all re-coloring steps are completed, we partition the set of agents into four subsets as follows. Define $X_d = \{i\in N : c(i) = d \land k+1\in S_i\}$ and $Y_d = \{i\in N : c(i) = d \land k+1\not\in S_i\}$ where $d\in \{1, 2\}$.
	For any $I$ compatible with $c$, it must hold that $X_1 \subseteq I$ (because $I$ is stable) and that $I \subseteq X_1 \cup Y_1$ (because $c$ and $I$ are compatible). 
	Therefore if $|X_1| > k$ or $|X_1| + |Y_1| < k$, then we conclude that no stable invitation of size $k$ is compatible with $c$ and terminate. Because $X_1 \subseteq I$ for all $I\in \mathcal{I}_c$, let us assume that $X_1 = \emptyset$ without loss of generality.
	
	Then, $I \subseteq Y_1$ for every stable invitation $I$ that is compatible with $c$. We cannot do a naive exhaustive search on $Y_1$ to find compatible $I$ because the search space can be as large as $O(n^k)$. Yet we can reduce the search space to $O(\beta^k)$. Let $I'$ be any subset of $Y_1$ of size $k$; $I'$ is individually rational due to the proper coloring constraint. Therefore its stability only depends on the existence of some agent $i \in X_0$ such that $R_i \cap I' = \emptyset$. Let $J = \emptyset$, and let us construct an invitation of size $k$. If there is some agent $i\in X_0$ such that $R_i \cap J = \emptyset$, then we try adding one agent from $R_i \cap Y_1$ to $J$, thereby increasing the size of $J$ by one. We repeat this process in an exhaustive manner until $J$ contains $k$ agents or there is no such agent $i \in X_0$ such that $R_i \cap J = \emptyset$. If the former occurs, we check whether $J$ is a stable invitation of size $k$ in poly-time; if $J$ is stable, then we output $J$, otherwise we continue exhaustive search. If the latter occurs, we can choose an arbitrary invitation $I$ with $J\subseteq I \subseteq Y_1$ because any such $I$ is guaranteed to be stable. Note that the search space is $O(\beta^k)$ because we choose up to $k$ agents (as $|J| \leq k$ at all times) and at each step we have at most $\beta$ agnets to choose from (i.e., $|R_i \cap J| \leq |R_i| \leq \beta$). 

	Therefore, this algorithm finds a stable invitation of size $k$ if it is given a proper coloring of any stable invitation of size $k$ in time $O(\beta^k n^O(1))$. We can de-randomize this algorithm by de-randomizing the coloring step (for instance, one can use $k$-perfect family of hash functions as in the work by \cite{ColorCoding}).
\end{proof}




\begin{theorem} \label{SIP:thm:stable_W1hard_alpha1_beta1}
	$k$-Stable-Invitation is W[1]-complete if $\alpha,\beta \geq 1$ and $\beta$ is bounded above by some function $f(k)$.
\end{theorem}
\begin{proof}
	We reduce from the $k$-Clique problem to show W[1]-hardness. 
	Let $G = (V, E)$ be an arbitrary graph for the $k$-Clique problem with parameter $k$. 
	Let us define $k' = 2(k^3 + \binom{k}{2})$ which is the parameter for $k$-Stable-Invitation. 
	For each node $v_i \in V$, we first create a group of $2k^2$ node-agents (call them $G_i$) such that $G_i=\{w_{i,x}: x \in [2k^2]\}$, and define $F_{w_{i,x}} = \{w_{i,x+1}\}$ (where $w_{i,2k^2+1}$ is understood as $w_{i,1}$) and $S_{w_{i,x}} = \{k'\}$.
	 For each edge $(v_i, v_j) \in E$, we create four edge-agents $e_{i,j}, e'_{i,j}, f_{i,j}$, and $f'_{i,j}$.
	 Define $F_{e_{i,j}} = \{w_{i,1}\}$, $F_{e'_{i,j}} = \{w_{j,1}\}$, and $S_{e_{i,j}} = S_{e'_{i,j}} = \{k'\}$. 
	 Define $F_{f_{i,j}} = \{e_{i,j}\}$, $E_{f_{i,j}} = \{e'_{i,j}\}$, and $S_{f_{i,j}} = \{k'+1\}$.
	 Define $F_{f'_{i,j}} = \{e'_{i,j}\}$, $E_{f'_{i,j}} = \{e_{i,j}\}$, and $S_{f'_{i,j}} = \{k'+1\}$.
	 We have created $2k^2n$ node-agents and $4|E|$ edge-agents, whose size is polynomial in $n,k$, and each agent we created has at most one friend and at most one enemy (thereby satisfying $\alpha=\beta=1$). 

	 Suppose $C$ is a clique of size $k$. Without loss of generality, suppose $C = \{v_1, v_2, \dots, v_k\}$, and define $I = (\cup_{i=1}^{k} G_i)\cup \{e_{i,j}, e'_{i,j}: 1 \leq i < j \leq k\}$. Note that $|I| = 2k^3 + 2 \binom{k}{2} = k'$, and we claim that $I$ is a stable invitation. It is clear that $I$ is individually rational, as each agent in $I$ approves the size $k'$ and her friends are all included in $I$. To see why $I$ is stable, first notice that any node-agent $v_j \not\in I$ and any edge-agents $e_{i,j}, e'_{i,j}\not\in I$ does not approve the size $k+1$. Now consider any edge-agent $f_{i,j}$. Since $I$ contains both or neither of $e_{i,j}$ and $e'_{i,j}$, $f_{i,j}$ does not wish to join $I$ because $f_{i,j}$ has one of them in friend-set and the other in enemy-set. Similarly, no edge-agent $f'_{i,j}$ wishes to join $I$. Therefore, $I$ is a stable invitation of size $k'$. 
	 
	 Conversely, suppose that $I$ is a stable invitation of size $k'$. Because $I$ must contain all or none of $G_i$ for each $i$, $I$ can contain at most $k$ such groups of node-agents (otherwise, if it contains more than $k$ such groups, the size of the invitation would exceed $k'$). Because $I$ is stable, it cannot contain any of the edge-agents $f_{i,j}$ or $f'_{i,j}$ as they do not approve the size $k'$. Finally, for every pair of edge-agents $e_{i,j}$ and $e'_{i,j}$, $I$ must contain both or neither of them; if only one of them is included in $I$, then either $f_{i,j}$ or $f'_{i,j}$ would wish to join $I$ which contradicts the stability of $I$. Therefore, because we earlier argued that $I$ can contain at most $k$ groups of node-agents, we conclude that $I$ must contain at least $2\binom{k}{2}$ edge-agents of the form $e_{i,j}$ and $e'_{i,j}$. Furthermore, because $e_{i,j},e'_{i,j} \in I$ implies that $G_i\subset I$ and $G_j\subset I$, if $I$ contains more than $2\binom{k}{2}$ edge-agents, it must also contain at least $k+1$ groups of node-agents (which we argued earlier leads to a contradiction). In summary, if $I$ is a stable invitation of size $k'$, it must contain exactly $k$ groups of node-agents (without loss of generality, assume that $I$ contains $G_1, G_2, \dots, G_k$) and exactly $2\binom{k}{2}$ edge-agents that are precisely $\{e_{i,j}, e'_{i,j}: 1\leq i<j\leq k\}$. Then, $C = \{v_1, v_2, \dots, v_k\}$ is a clique in the original instance because there exists an edge between $v_i$ and $v_j$ for every pair $v_i,v_j\in C$ (the fact that edge-agents $e_{i,j}$ exist implies this). 
	
	This shows W[1]-hardness, and W[1]-completenes follows from Theorem~\ref{SIP:thm:stable_W2}. 

\end{proof}


\begin{theorem} \label{SIP:thm:stable_W1hard_alpha2_beta0}
	$k$-Stable-Invitation is W[1]-complete if $\alpha \geq 2$ and $\beta$ is bounded above by some function $f(k)$.
\end{theorem}
\begin{proof}
	We reduce from the $k$-Independent-Set problem to show W[1]-hardness. 
	Let $G = (V, E)$ be an arbitrary instance of the $k$-Independent-Set problem with parameter $k$. 
	For each node $v\in V$ we create a node-agent $v$ with approval set $S_v = \{k\}$ and friend set $F_v = \emptyset$.
	For each edge $(v, w) \in E$ we create an edge-agent $e_{v,w}$ with friend set $F_{e_{v,w}} = \{v, w\}$ and approval set $S_{e_{v,w}} = \{k+1\}$.
	
	Suppose $S \subseteq V$ is an independent set of size $k$ and let $I = S$. $I$ is individually rational as for every node-agent $v\in I$, $|I| = k \in S_v$ and $F_v \subseteq I$. To see why $I$ is stable, for every node-agent $w \not\in I$, $k+1 \not\in S_w$. For every edge-agent $e_{v,w}$, we know that $v\not\in I$ or $w\not\in I$ because if $v,w\in I$ then $v,w\in S$ and $(v,w)\in E$, which contradicts the fact that $S$ is an independent set (note that the edge-agent $e_{v,w}$ is created only if $(v,w)\in E$). Therefore $I$ is a stable invitation of size $k$.
	
	Conversely, suppose that $I$ is a stable invitation of size $k$. Then $I$ cannot contain any edge-agent because every edge-agent only approves the size $k+1$. Let $S = I$ and we claim that $S$ is an independent set. If there is a pair of nodes $v,w \in S$ with $(v, w)\in E$, then there must exist an edge-agent $e_{v,w}$. However, $F_{e_{v,w}} = \{v, w\}$ and $e_{v,w}\not\in I$, which implies that $I$ is not stable. This is a contradiction, and therefore $S$ is an independent set of size $k$. 

	This shows W[1]-hardness, and W[1]-completenes follows from Theorem~\ref{SIP:thm:stable_W2}.
\end{proof}


\begin{theorem} \label{SIP:thm:stable_W2hard_beta}
	$k$-Stable-Invitation is W[2]-complete if $\beta$ is not bounded above by any function of $k$.
\end{theorem}
\begin{proof}
	We reduce from the $k$-Dominating-Set problem which is known to be W[2]-hard.
	Given an arbitrary graph $G = (V, E)$ and a parameter $k$, we create $2n$ node-agents by creating $x_i$ and $y_i$ for each $v_i\in V$. We define their approval sets and enemy sets as follows: $S_{x_i} = \{k\}$ and $E_{x_i} = \emptyset$ for all $x_i$ while $S_{y_i} = \{k+1\}$ and $E_{y_i} = \{x_i\} \cup \{x_j : (v_i, v_j)\in E\}$ for all $y_i$. Note that a stable invitation cannot contain any of $y_i$'s because of their approval sets.
	
	Let $D$ be a dominating set of size $k$ in the original instance, and let $I = \{x_i : v_i \in D\}$. 
	We claim that $I$ is a stable invitation of size $k$. 
	For every node-agent $x_i\in I$, $|I| = k \in S_{x_i}$ and $E_{v_i} \cap I = \emptyset$, and thus $I$ is individually rational. 
	For every node-agent $x_j \not\in I$, $k+1 \not\in S_{x_j}$, and thus $x_j$ has no incentive to join $I$. 
	For every node-agent $y_i$, either $x_i\in I$ or there is some $x_j\in I$ such that $x_j \in E_{y_i}$ because $D$ is a dominating set; that is, if $v_i \not\in D$, then there exists some $v_j\in D$ where $(v_i, v_j) \in E$, which implies that $\exists x_j \in E_{y_i} \cap I$ if $y_i\not\in I$. 
	
	Conversely, suppose that there exists a stable invitation $I$ of size $k$. 
	Clearly $I$ can only contains node-agents of the form $x_i$'s (due to approval sets of $y_i$'s), and we claim that $D = \{v_i : x_i \in I\}$ is a dominating set. 
	Pick any node-agent $x_i\not \in I$. Since $I$ is stable and $x_i\not\in I$, we know that there exists some node-agent $x_j\in I$ such that $x_j\in E_{y_i}$. This implies that $(v_i, v_j) \in E$, and therefore $D$ is a dominating set. 
	
	This shows W[2]-hardness, and W[2]-completenes follows from Theorem~\ref{SIP:thm:stable_W2}.
\end{proof}







\section{Symmetric Social Relationship} \label{SIP:sec:symm}

 \begin{table*}[t!] 
	 \small
 	\centering
 \begin{tabular}{|l|*{5}{c|}|*{5}{c|}}\hline
 \multirow{2}*{} & \multicolumn{5}{c||}{Symmetric $k$-IR-Invitations } & \multicolumn{5}{c|}{Symmetric $k$-Stable-Invitations } \\ \cline{2-11}
  & $\!\!\beta = 0\!\!$ & $\!\!\beta = 1\!\!$& $\!\!\beta=2\!\!$ & $\!\!3 \le \beta \le f(k)\!\!$& $\!\!\beta < n\!\!$ & $\!\!\beta = 0\!\!$ & $\!\!\beta = 1\!\!$ & $\!\!\beta=2\!\!$ & $\!\!3 \leq \beta \leq f(k)\!\!$ & $\!\!\beta < n\!\!$ \\ \hline
 $\alpha = 0$ & P & P & P & FPT & \!\!W[1]-C\!\! & P & P & P & FPT & \!\!W[2]-C\!\! \\ \hline
 $\alpha = 1$ & P & P & FPT & FPT & \!\!W[1]-C\!\! & P & P & FPT & FPT & \!\!W[2]-C\!\! \\ \hline
 $\alpha \geq 2$ & P & FPT & FPT & FPT & \!\!W[1]-C\!\! & P & FPT & FPT & FPT & \!\!W[2]-C\!\! \\ \hline
 \end{tabular}
 \caption{\small Complexity of \SIPs with symmetric social relationships. $f(k)$ can be an arbitrary function of $k$ that only depends on $k$.
All entries other than ``P'' imply NP-completeness.
 ``W[1]-C'' and ``W[2]-C'' mean W[1]-completeness and W[2]-completeness, respectively. 
 All results are original (including classical complexity results). }
 \label{SIP:tbl:summary_symmetric}
 \end{table*}

Recall the definition of ``symmetric social relationships'' from Definition~\ref{SIP:def:symmetric_social}.
Under symmetric social relationships, both $k$-IR-Invitation and $k$-Stable-Invitation admit efficient FPT algorithms when $\beta$ is bounded, as shown in Table~\ref{SIP:tbl:summary_symmetric}, although their complexity does not change when $\beta$ is unbounded (W[1]-complete and W[2]-complete, respectively).

When we compare results in Table~\ref{SIP:tbl:summary} and Table~\ref{SIP:tbl:summary_symmetric}, two interesting observations can be made.
First, $k$-IR-Invitations and $k$-Stable-Invitations have the same classical complexity even under symmetric social relationships. Second, both problems now admit efficient FPT algorithms for broader domains of inputs -- as long as $\beta$ is bounded.
Note that our classical complexity results are original (i.e., not implied by Theorem~\ref{SIP:thm:nphard}) because Lee and Shoham~\cite{LEE15AAAI} did not consider the special case of symmetric social relationships. 

\subsection{Symmetric $k$-IR-Invitations}

We first present classical complexity results for $k$-IR-Invitations under symmetric social relationships, followed by parameterized complexity results. 


\begin{theorem} \label{SIP:thm:symmetric_IR_p_npc}
	When agents have symmetric social relationships, 
	$k$-IR-Invitations can be solved in polynomial time if (i) $\beta = 0$, (ii) $\beta = 1$ and $\alpha \leq 1$, or (iii) $\beta = 2$ and $\alpha = 0$. Otherwise, the problem is NP-hard. 
\end{theorem}
\begin{proof}
	Let us consider case (ii) in the statement. 
	As before, without loss of generality assume that all agents accept the size $k$ (i.e., $k\in S_i$ for all $i\in N$). 
	
	We first construct an {\em enemy graph} in which nodes represent agents, and we create an edge between two nodes if their corresponding agents are enemies of each other. For every pair of friends, we merge their nodes in this graph into a meta-node of weight $2$ (if they are also enemies of each other, then we simply remove them from the graph); let us call the resulting graph a {\em friend graph}. 
	Now finding an IR invitation of size $k$ is equivalent to finding an independent set of total weight $k$ in the friend graph. Although finding an independent set (of an arbitrary size) is NP-hard, all nodes in the friend graph have at most two edges, and thus each connected component in the friend graph is either a path or a cycle. 
	A dynamic programming algorithm can solve this problem.
	For each connected component (which is either a path or a cycle), we can check for every integer $x$ between $0$ and $k$, inclusive, whether it is possible to choose exactly $k$ nodes (agents) from the component, while satisfying individual rationality (provided that we will choose $k-x$ agents from other components, thereby making the invitation's size equal to $k$). This check can be done trivially: If a path contains $y$ nodes in total, we can choose up to $\lceil y/2 \rceil$ agents from the component (which should be an independent set in the enemy graph) and if a cycle contains $z \geq 3$ nodes in total, we can choose up to $\lfloor z/2 \rfloor$ nodes. After this check for each component and each integer, we can solve the main problem using a knapsack-like algorithm to find an IR invitation of size $k$.

 	Case (i) can be solved in polytime as follows (again, assume $k \in S_i$ for every agent $i$). Because $\beta = 0$, agents only have friends. Again we construct a friendship graph in which nodes are agents and there is an edge between two nodes if they are friends of each other in \SIP. Now we find all connected components in this graph; everyone in the same component must all be invited or all be uninvited. This reduces to a knapsack problem where we seek a subset of components whose total weight (the number of nodes in them) is equal to $k$. 

	Lastly, Case (iii) can be solved in polytime as follows (again, assume $k \in S_i$ for every agent $i$).
	Since $\alpha = 0$, no agent has friends. If we construct an enemy graph in which nodes are agents and there is an edge between two nodes if the corresponding agents are enemies, then finding an IR invitation of size $k$ is equivalent to finding an independent set of size $k$ in the enemy graph we constructed. Furthermore, each node in this graph has at most two edges (because $\beta = 2$), and therefore every connected component in this graph is either a path or a cycle. Again, we can use a dynamic programming algorithm (similar to the one we used for Case (i) above) to solve this problem in polytime.

		
	Let us now prove that the problem is NP-hard if none of the three conditions in the statement holds. 
	It is known that the Independent Set problem is NP-hard even if every node has degree at most 3~\cite{Garey_Max_Is_Cubic}. 
	Given an instance of this problem, we can create an instance of $k$-IR-Invitations as follows. 
	For each node, we create an agent $i$ with $S_i =\{k\}$ (agent only approves size $k$). 
	If there is an edge between two nodes, we make their corresponding agents enemies of each other. 
	The resulting instance is a valid instance (with symmetric social relationships) of $k$-IR-Invitations with $\alpha = 0$ and $\beta = 3$ (because each node in the original instance as at most three neighbors). 
	This shows NP-hardness of $k$-IR-Invitations with symmetric social relationships when $\alpha = 0$ and $\beta \geq 3$.	
	
	We can modify our reduction to show that the problem is NP-hard when $\alpha = 2$ and $\beta = 1$.
	First, given an instance of the Independent Set problem with max-degree 3, we create three agents for each vertex and make the three agents friends with one another; these three agents together represent a vertex in the original instance.
	For each edge in the original instance, pick one agent from each component representing the either end-point of the edge, and make them enemies of each other; because each node has at most three edges in the original instance, we can manage to keep each agent's enemy set to contain at most one agent. Lastly, we set approval sets of agents such that everyone approves of size $k$. The resulting instance is a valid instance of \SIPs with symmetric social relationships with $\alpha = 2$ and $\beta = 1$. 
	
	Lastly, we need a similar modification to our reduction for the case when $\alpha = 1$ and $\beta = 2$. 
	First, given an instance of the Independent Set problem with max-degree 3, we create two agents for each vertex and make the two agents friends with each other; these two agents together represent a vertex in the original instance.
	For each edge in the original instance, pick one agent from each component representing the either end-point of the edge, and make them enemies of each other; because each node has at most three edges in the original instance, we can manage to keep each agent's enemy set to contain at most two agents (and each component can have at most four edges). Lastly, we set approval sets of agents such that everyone approves of size $k$. The resulting instance is a valid instance of \SIPs with symmetric social relationships with $\alpha = 1$ and $\beta = 2$. 
 	
	Note that NP-hardness for $\alpha = a$ and $\beta = b$ implies NP-hardness for $\alpha \geq a$ and $\beta \geq b$.
	Hence, our proofs provide a complete analysis as stated in the theorem.
	
\end{proof}


\begin{theorem} \label{SIP:thm:symmetric_IR_FPT}
	When agents have symmetric social relationships, 
	$k$-IR-Invitations is in FPT if $\beta \leq f(k)$ where $f(k)$ can be an arbitrary function of $k$. 
\end{theorem}
\begin{proof}
	As before, without loss of generality, assume $k\in S_i$ for all $i\in N$. 
	We first create a {\em friend graph} in which nodes represent agents, and we create an edge between two nodes if their corresponding agents are friends. Clearly, subsets of nodes in this graph and invitations have one-to-one correspondence.
	In the friend graph, it is clear that all or none agents in each component should be chosen to form an IR invitation. 
	Thus, if any connected component contains two nodes whose corresponding agents are enemies of each other, then we can safely remove the component from the graph (as it cannot be included in any IR invitation).
	Likewise, if any component contains more than $k$ agents, we can remove the component as well. 
	
	We then create an {\em enemy graph} in which nodes represent connected components in the friend graph. Each node in the enemy graph has a weight that is equal to the size of the component it represents, and we create an edge between two nodes if their corresponding components contain a pair of enemies (one agent in each component). Because each agent has at most $\beta$ enemies, each node in the enemy graph has at most $k\cdot \beta$ edges. 
	Notice that an independent set in the enemy graph represents an IR invitation in the original instance. 
		
	Similarly to the FPT algorithm given in proof of Theorem~\ref{SIP:thm:IR_invitation_FPT}, we use Color Coding to color each node in the enemy graph as $\{0,1\}$ with equal probability.
	If there is any edge in the enemy graph whose both end-points (components) are of color 1, then we re-color both of them as 0. We repeat this process until no such pair exists (which can be done in linear time by scanning through the edges). 
	After this step, it is clear that all nodes of color $1$ form an independent set; we can easily determine if a subset of nodes whose weight is $k$ exists, using a knapsack-like algorithm. 
	
	Provided that an IR invitation of size $k$ exists, this algorithm's probability of success is at least $(1/2^k) \cdot (1/2^{k\beta}) \geq 1/2^{k(1+f(k))}$. For any fixed IR invitation $I^*$ of size $k$, 
	we color all agents in $I^*$ as color 1 with probability $1/2^k$, and with probability at least $1/2^{k \beta}$ we color the union of enemies of all agents in $I^*$ as color $0$. Regardless of coloring of all other agents, this coloring will ensure that all agents in $I^*$ remain to be of color $1$ in the enemy graph, and thus our algorithm can find $I^*$ (or some other solution). 
	
	The overall runtime of our algorithm is $O(f(k) n)$ as all sub-routines can be implemented in linear time in size of each graph and each graph contains at most $O(n)$ nodes and $O(f(k)n)$ edges.
	We can repeat this randomized algorithm $2^{k(1+f(k))}\ln n$ times to increase the probability of success to $1-1/n$ (with overall runtime $2^{k(1+f(k))}(f(k) n \ln n)$).
	This algorithm can also be de-randomized using a family of $k$-perfect hash functions~\cite{ColorCoding}. 
\end{proof}

Lastly we show that $k$-IR-Invitations remains to be W[1]-complete, even under symmetric social relationships, when $\beta$ is not bounded. Proof of W[1]-hardness is similar to that of Theorem~\ref{SIP:thm:IR_invitation_large_beta}, and completeness follows from Theorem~\ref{SIP:thm:IR_invitation_W1}. We omit proof of Theorem~\ref{SIP:thm:symmetric_IR_W1C}.
\begin{theorem} \label{SIP:thm:symmetric_IR_W1C}
	When agents have symmetric social relationships, 
	$k$-IR-Invitations is W[1]-complete if $\beta$ is not bounded above by any function of $k$. 
\end{theorem}



\subsection{Symmetric $k$-Stable-Invitations}
Interestingly, complexity of $k$-Stable-Invitations and that of $k$-IR-Invitations are identical, except when $\beta$ is unbounded, if we assume symmetric social relationships.
This implies that the combinatoric complexity due to social relationships plays an important role in \SIP, and restrictions on social relationships (such as symmetry) can substantially reduce the complexity. 
Yet we emphasize that both polytime and FPT algorithms for $k$-Stable-Invitations are much more complicated than those for $k$-IR-Invitations, and much of its complexity is due to the additional requirement that uninvited agents must not be willing to attend.

We first present classical complexity results for $k$-Stable-Invitations under symmetric social relationships, followed by parameterized complexity results. 


\begin{theorem} \label{SIP:thm:symmetric_stable_p_npc}
	When agents have symmetric social relationships, 	
	$k$-Stable-Invitations can be solved in polynomial time if (i) $\beta = 0$, (ii) $\beta = 1$ and $\alpha \leq 1$, or (iii) $\beta = 2$ and $\alpha = 0$. Otherwise, the problem is NP-hard. 
\end{theorem}
\begin{proof}
	Let us first consider the case (i) when $\beta = 0$.
	We construct a friend graph as before, and find connected components in this graph. 
	Any stable invitation must contain all or none of nodes in each connected component.
	For each connected component, we check two things: Whether a stable invitation can contain all of nodes in the component and whether it can contain none of nodes in it. 
	To check the first, we simply check if the component is of size $k$ or less and if everyone in the component approves size $k$. 
	To check the second, we check if the component contains two or more nodes (then we can leave them out) or if it is a singleton component but the only agent in it does not approve size $k$ (then we can leave the agent out). All of these checks can be done in linear time.
	Now we can use a dynamic programming algorithm to determine whether a subset of connected components that contains $k$ nodes over all such that every component (whether selected or not) does not violate the stability conditions (which can be easily checked by the two conditions we processed earlier).
	
	Let us now consider case (ii) where $\alpha \leq 1$ and $\beta = 1$.
	We first construct a friend graph, and merge any connected component into a meta-node with weight 2.
	The resulting graph would only have nodes but no edges, and each node is either of weight 1 or 2.
	Now we construct an enemy graph on top of this by creating an edge between nodes if their corresponding agents are enemies of each other; if we create self-loops, then it means two agents are both friends and enemies at the same time, and we just remove them from our graph (and input instance) with no harm.
	Now, after all edges are created, each node of weight 1 has at most one edge and each node of weight 2 has at most two edges. This implies that all connected components in this graph are either a path or a cycle.
	We can use the same algorithm for case (iii) below to solve this problem. 

	Finally let us consider case (iii) where $\alpha = 0$ and $\beta = 2$.
	Let us construct an enemy graph by creating nodes for agents and an edge between nodes if the agents are enemies of each other. Each node can have at most two edges because $\beta = 2$.
	Therefore, every connected component in the enemy graph is either a path or a cycle.
	For each connected component and every integer $x$ between $0$ and $k$, we check if it is possible to choose exactly $x$ nodes from the component such that inviting their corresponding agents and not inviting other agents (in the same component) do not violate stability conditions. This is similar to finding an independent set on a path or a cycle, but with extra conditions: the nodes not being chosen must still satisfy the stability requirements (by not approving size $k+1$ or by having one of its neighbors included in the set). We omit the details, but mention that this can be solved by a dynamic programming algorithm in polytime. After this step, we can solve the main problem using a knapsack algorithm in a straightforward manner. 
	
	Our reductions for $k$-IR-Invitations in proof of Theorem~\ref{SIP:thm:symmetric_IR_p_npc} show NP-hardness for $k$-Stable-Invitations as well, because agents only approve invitations of size $k$ in our reduction; it ensures that any uninvited agent would be unwilling to attend due to the size of an invitation.
	
\end{proof}



\begin{theorem} \label{SIP:thm:symmetric_stable_FPT}
	When agents have symmetric social relationships, $k$-Stable-Invitations is in FPT if $\beta \leq f(k)$ where $f(k)$ can be an arbitrary function of $k$. 
\end{theorem}
\begin{proof}
	The main is similar to that of our proof of Theorem~\ref{SIP:thm:symmetric_IR_FPT}, but details entail more complicated processes due to stability conditions on uninvited agents.

	We first create a {\em friend graph} in which nodes represent agents, and we create an edge between two nodes if they are friends of each other. 
	Any stable invitation must contain all or none of agents in the same connected component; if a connected component contains more than $k$ agents or it contains enemies within itself, then we can safely remove the components (or agents in them) from input instance. 
	Also, if a connected component of size at least $2$ contains an agent who does not approve size $k$, then we remove the entire connected component; if it is just one agent who does not approve size $k$, then we remove it only if it does not approve size $k+1$ as well (this is because of stability conditions). 
	After the removal process, we create an {\em enemy graph} by treating each connected component as a weighted node whose weight is equal to the size of the component, and creating an edge between nodes if they contain enemies (one in each).
	
	Again, we use color coding approach by coloring nodes using two colors $\{0,1\}$ uniformly and independently at random. 
	If two adjacent nodes in the enemy graph are both colored as 1, then we re-color both of them to 0. 
	We repeat the re-coloring step until no such nodes remain; at this point, all nodes of color 1 form an independent set, and any subset of them form an IR invitation.
	We will try to find a subset of nodes of color $1$ whose weight is equal to $k$ that forms a stable invitation. We can disregard any node of color 0 and weight greater than 1, because the agents contained in those nodes would not block stability (because we only consider single-agent deviation).
	On the other hand, singleton nodes of color $0$ may block an invitation when excluded, if those node approve size $k+1$ and no enemy of them are invited. 
	To avoid this, we are going to choose at least one enemy (of color $1$) of each singleton node of color $0$ who approves size $k+1$ in brute-force manner; the search space is bounded because we can only choose up to $k$ agents and each agent has at most $f(k)$ enemies (i.e., the search space is $O((f(k))^k)$). After we pre-select a subset of agents of color $1$ in this manner, we no longer need to worry about stability condition being violated by any agent of color $0$. Next, we check if there is any singleton node of color $1$ who approves size $k+1$ and is not pre-selected in the previous step; those nodes must also be included so as not to violate stability condition. Suppose that we have so far selected $x$ agents in this manner; if $x > k$, then our algorithm will (possibly incorrectly) report that no stable invitation of size $k$ exists. Otherwise, among the remaining nodes of color $1$, we will check if an independent set of weight $k - x$ exists (via a knapsack algorithm), to solve the main problem.

		Overall, this randomized algorithm runs in FPT time ($O((f(k))^k \cdot (f(k)n))$), and its probability of success is at least $(1/2^k)\cdot (1/2^{k\beta}) \geq 1/2^{k(1+f(k))}$ (similar to the algorithm shown in Theorem~\ref{SIP:thm:symmetric_IR_FPT}).
		
\end{proof}



\section{Discussion and Future Work} \label{SIP:sec:discussion}
In this chapter, we thoroughly investigated the complexity of the Stable Invitations Problem (\SIP) for two different solution concepts -- individual rationality and (Nash) stability. We considered restrictions on inputs by limiting the number of friends and enemies each agent can have, and also studied the special case in which all agents have symmetric social relationships. In all cases we studied, classical complexity of two solution concepts did not differ at all. However, when we analyzed parameterized comlexity, their complexity results showed that finding an IR solution is much less complex a problem than finding a stable solution in all cases, whether agents have symmetric relationships or not.

Our work leaves a few interesting open problems for future work. 
Lee and Shoham~\cite{LEE15AAAI} considered another solution concept in which agents who are not invited are not envious of those who are invited (motivated by `envy-freeness'). It would be interesting to analyze the parameterized complexity of finding an envy-free invitation of size $k$, and compare the results with what we have in this work. In addition, analyzing the parameterized complexity of the Group Activity Selection Problem~\cite{GASP12WINE} is another interesting direction for future work. 




\chapter{Strategic Agents and Mechanisms}
\label{GT:chapter}


We have so far assumed that the event organizer knows of or can query preferences and constraints of agents, but in some settings this may not be a realistic assumption. For instance, if agents are strategic, the organizer can no longer assume that agents will report their preferences and constraints truthfull because lying may lead to a better outcome than telling the truth would.

To understand why strategic behavior of agents can be troublesome, let us consider a small example of the (anonymous) Stable Invitations Problem.

\begin{example} \label{GT:eg:strategicAgents}
		Consider two agents with identical preferences on the number of participants with no friends or enemies. 
	\begin{equation*}
		\begin{aligned}
				S_1 = \{1\}, ~~& S_2 = \{1\}
		\end{aligned}
	\end{equation*}

Note that both agents have decreasing preferences in this example. 
There are two stable invitations, namely $I_1 \{a_1\}$ and $I_2 = \{a_2\}$. 
The empty invitation ($\emptyset$) is not stable while the full invitation ($\{a_1,a_2\}$) is not individually rational. 

If the organizer were to choose $I_1$ given preferences of agents, then $a_2$ would have an incentive to lie -- if $a_2$ had reported her preference as $\hat{S}_2 \{1, 2\}$ instead of $S_2$, then $I_2$ would have been the only stable invitation given $(S_1, \hat{S}_2)$. Clearly, $a_2$ prefers $I_2$ over $I_1$, and therefore $a_2$'s best action in hindsight would have been lying about her preference. By symmetric, the same holds for $a_1$, if the organizer were to choose $I_2$ given $(S_1, S_2)$.
\end{example}

In this chapter we investigate how strategic behavior of agents can affect the outcome in the Stable Invitations Problem and the Group Activity Selection Problem. We begin by considering the least complex problem, which is the intersection of these two problems -- agents have anonymous preferences (i.e., no friends and enemies) and there is only one activity. This is a fairly restrictive assumption, but it reveals interesting interactions between strategic behavior of agents and the objectives in these problems. In fact, as we shall see, it is in general impossible to design a mechanism under which truth-telling is a dominant strategy. In a more restrictive setting, however, where all agents have increasing preferences (i.e., all agents prefer more participants than fewer), we can design a mechanism under which truth-telling is a dominant strategy in addition to many other desirable properties. 



\section{Definitions and Notation} \label{GT:sec:prelim}



While we focused on designing algorithms or proving computational hardness results in previous chapters, the focus of this chapter is on mechanism design and impossibility results (much like computational hardness results). In mechanism design theory, the objective is to design a mechanism which is given an action profile of all agents and chooses an outcome (which can be probabilistic) such that the mechanism satisfies certain properties. One of the most important properties is incentive compatibility -- agents must be incentivized to act in a certain way (such as truth-telling) because no agent can be better off by acting in a different way. In many settings, utilites of agents may be transferable (such as money), which provides ways to compensate for agents who end up with less desirable outcomes. However, in social settings like the ßgroup scheduling and assignment problems, it is more reasonable to assume that utilites of agents are assumed to be non-transferable, and therefore it is not possible to compenstate agents for any outcome. 



\section{Impossibility Results and Strategy-proof Mechanisms} \label{GT:sec:Mechanism}
In the strategic case of \ASIP, we assume that agents may act strategically in reporting their preferences to the event organizer. 
Example~\ref{GT:eg:strategicAgents} shows how agents may have an incentive to act strategically. 


% \begin{example}[An agent may act strategically.] \label{GT:eg:strategicAgents}
%
% 	Let us consider an example with two agents with identical preferences:
% 	\begin{equation*}
% 		\begin{aligned}
% 				P_1: 1 \succ 0 \succ 2,~~~& P_2: 1 \succ 0 \succ 2
% 		\end{aligned}
% 	\end{equation*}
% 	Note that both agents have \DEC-preferences (with $h_1 = h_2 = 1$).
% 	The two stable invitations are $S_1 = \{a_1\}$ and $S_2 = \{a_2\}$.
% 	The empty invitation ($\emptyset$) is not EF while the full invitation ($N$) is not IR.
%
% 	Hence $S_1 = \{a_1\}$ and $S_2 =\{a_2\}$ are the only two stable invitations provided that both agents are truthful.
% 	If $S_1$ were to be chosen by the organizer, $a_2$ would have an incentive to act strategically, by reporting $\hat{P}_2 (1 \succ 2 \succ 0)$ instead of $P_2$.
% 	Given $(P_1, \hat{P}_2)$, the only stable invitation is $S_2$ ($S_1$ is no longer EF due to $a_2$).
% 	Notice that $a_2$ strictly prefers $S_2$ over $S_1$ (because $1 \succ_2 0$) and has an incentive to misreport in this example. By symmetry, $a_1$ may act strategically if $S_2$ were to be chosen.
% \end{example}

We first provide a formal definition of a mechanism in the context of \ASIP\ with strategic agents. 
We then state several impossibility results for general cases of \ASIP, and also provide a strategy-proof mechanism for special cases of \ASIP.  Although we only consider deterministic mechanisms here, we discuss how one can generalize to randomized mechanisms at the end of this section.

\begin{definition} \label{GT:def:mechanism}
Given an instance $(N, P)$ of \ASIP, we define $V_i$ (the set of available actions to $a_i$) to be the set of all preferences over $X$ where $X = \{0, 1, 2, \dots, n\}$ is the set of outcomes. 
A (deterministic) \emph{mechanism} is a pair $(V, Z)$ where $V = (V_1 \times \cdots \times V_n)$ is the set of action profiles of all agents (i.e., $V_i$ is a subset of total preorder on $X$) and $Z: V \mapsto U$ is a mapping from each action profile to an invitation in $U$ where $U = 2^{N}$. 
Let $V_{-i} = (V_1 \times \cdots \times V_{i-1} \times V_{i+1} \times \cdots \times V_{n})$ be the set of action profiles available to all agents but agent $a_i$. A mechanism $(V, Z)$ is said to be \emph{strategy-proof} if for all $a_i\in N$ it holds that $Z(P_i, v_{-i}) \succeq_i Z(v_i, v_{-i})$ for all $v_i \in V_i$ and $v_{-i} \in V_{-i}$.
\end{definition}
%%%% ========== Impossibility Results
We now formally state our first impossibility result.
\begin{theorem} \label{GT:thm:impossibility}
No strategy-proof mechanism can find a stable invitation, even if it exists, for arbitrary instances of \ASIP.
\end{theorem}
\begin{proof}
Example~\ref{GT:eg:strategicAgents} can serve as a proof; no strategy-proof mechanism can find a stable invitation for this instance.  
%\qed
\end{proof}
Intuitively this result is due to the conflicting interests of the organizer and agents -- the organizer is trying to maximize attendance while the agents (with \DEC-preferences) to minimize. 
Since no strategy-proof mechanisms can find a stable invitation, 
one can instead seek to design a strategy-proof mechanism that can find a non-empty IR invitation 
by dropping the requirement of envy-freeness.
We show that this is also impossible as Theorem~\ref{GT:thm:impossibility_IR} states.

\begin{theorem} \label{GT:thm:impossibility_IR}
	No strategy-proof mechanism can find a non-empty individually rational (IR) invitation, even if it exists, for arbitrary instances of \ASIP. 
\end{theorem}
\begin{proof}
Consider three agents with preferences as follows:
\begin{equation*}
	\begin{aligned}
			P_1: 3\succ 0 \succ 2 \sim 1,~ P_2: 2 \succ 3 \succ 2 \succ 1, ~ P_3: 3 \succ 2 \succ 0 \succ 1
	\end{aligned}
\end{equation*}
There are two non-empty IR invitations: $S_1 = \{a_1, a_2, a_3\}$ and $S_2 = \{a_2, a_3\}$. 
If a mechanism chooses $S_1$ given $(P_1, P_2, P_3)$, $a_2$ can report $(2 \succ 0 \succ 3 \sim 1)$ and make $S_2$ the only non-empty IR invitation. Similarly, if a mechanism chooses $S_2$ given $(P_1, P_2, P_3)$, $a_3$ can report $(3 \succ 0 \succ 2 \sim 1)$ so as to make $S_3$ the only non-empty IR invitation. The rest of the proof is similar to that of Theorem~\ref{GT:thm:impossibility}.
%\qed
\end{proof}

% TODO: proof sketch, at least.
% To strengthen our impossibility results even further, we show that manipulation is computationally easy provided that preferences of other agents are known to a manipulator. We provide an informal lemma and omit its proof.
% \begin{lemma}[Informal] \label{GT:lemma:gsip_manipulation}
% There exists a polynomial time algorithm, given an instance $(N, P)$ of \GSIP\ and a mechanism $(V, Z)$, that decides in polynomial time whether there exists a certain preference ordering $v_i \in V_i$ of $a_i$ such that $a_i$ (strictly) prefers $Z(v_i, v_{-i})$ to $Z(P_i, v_{-i})$ where $v_{-i} \in V_{-i}$.
% \end{lemma}  

Earlier we emphasized that the conflict between agent(s) and the organizer is the main factor that leads to an impossibility result. 
Indeed, in the example we used in the proof of Theorem~\ref{GT:thm:impossibility}, both agents have \DEC-preferences while the organizer's goal is to maximize attendance.

We now consider a special case of \ASIP\ in which all agents have \INC-preferences.
We obtain a positive result in this case as Theorem~\ref{GT:thm:mechanism} states. 

\begin{theorem} \label{GT:thm:mechanism}
	There is a strategy-proof mechanism for \INC-instances of \ASIP, which can also find a maximum stable invitation in linear time (after sorting). 
% A mechanism described in Algorithm~\ref{GT:alg:mechanism} is strategy-proof, and finds a maximum stable invitation in linear time (after sorting), given \INC-instances of \ASIP. 
\end{theorem}
\begin{proof}[Proof sketch]
For simplicity let us assume that each agent reports her threshold value (i.e., $l_i$ as defined in Definition (to be added)) as $L_i$ ($L_i$ may differ from $l_i$). 
Our mechanism then chooses the largest $k$ such that $L_k \leq k$ holds and chooses the set of $k$ agents with largest threshold values (if no such $k$ exists, mechanism chooses the empty invitation). 
\end{proof}
	Although our mechanism is simple, proof of Theorem~\ref{GT:thm:mechanism} is not trivial. First, even though all agents have \INC-preferences, a full invitation is not necessarily stable (if at least one agent is unwilling to attend at all). Second, an agent may have an incentive to under-report that results in having the organizer to invite more agents than when the agent had truthfully reported. It is indeed possible for an agent to under-report ($L_i < l_i$) and lead to a larger invitation, but we show that the size of the larger invitation would still be less than $l_i$ (see Appendix for a proof). 


\subsection{Extensions}\label{GT:sec:asip_extension}
Although we have so far only discussed deterministic mechanisms, our impossibility results can be extended to randomized mechanisms. First we define $Z$ to be a mapping from $V$ to $\Pi(U)$ where $\Pi(U)$ denotes the set of all probability distributions over $U$. The definition of a strategy-proof mechanism must change accordingly -- we do this by adopting the axioms in the von Neumann-Morgenstern utility theorem~\cite{von1947theory}. We introduce lotteries over invitations and define preferences of agents over lotteries. Given a probability distribution over invitations, one can compute the expected cardinal utility of lotteries. We then define a strategy-proof mechanism analogously to Definition~\ref{GT:def:mechanism}: for each $a_i$, it must hold that the expected utility of $Z(P_i, v_{-i})$ is no less than the expected utility of $Z(v_i, v_{-i})$ for all $v_i\in V_i$ and for all $v_{-i} \in V_{-i}$. The impossibility result given by Theorem~\ref{GT:thm:impossibility} still holds: If $(V, Z)$ is a strategy-proof mechanism, then $Z(P_1, P_2)$ must assign zero probability to both $\{a_1\}$ and $\{a_2\}$, yet these are the only two stable invitations.  All other impossibility results we mention in this work can be extended in this manner.

We can extend our results in a different direction by considering multiple time-alternatives for the event. In such settings, agents may have preferences over time-alternatives for the event, in addition to size of invitations.   
In the non-strategic case, our easiness result is still applicable: One can run the algorithm used in Theorem~(to be added  GT:thm:asip:algo) iteratively for each time-alternative, and choose the maximum stable invitation among all.
In the strategic case, our impossibility results immediately imply the same negative results. 
For \INC-instances of \ASIP, we obtain a similar impossibility result even if there is only two time-alternatives. The intuition is that over-reporting ($L_i > l_i$) can give the veto power to an agent, which prevents us from designing a strategy-proof mechanism even for \INC-instances of \ASIP. Note that all of our impossibility results can be naturally extended to the Group Activity Selection Problem by Darmann et al.~\cite{GASP12WINE} since \ASIPs is a sub-class of \GASPs with a single activity.


\chapter{Discussion and Open Problems}{}
\label{discussion:chapter}

In Chapter~\ref{bdoodle:chapter}, we modeled group scheduling problems as an optimization problem by considering time and inconvenience as two important meausres of efficiency of the scheduling processes.  

\nocite{*}

\bibliographystyle{plain}

\bibliography{thesis}

\end{document}


