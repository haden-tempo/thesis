
My journey of studying computer science started nearly two decaeds ago when I first used a personal computer to play games. Just like anyone else around me, I wanted to create a fun game of my own, and decided to learn how to write a program. I first learned QBasic, and it was fun to write a program that prints out a bunch of characters on an old CRT monitor. At the time playing games was more fun to me than writing programs -- partly because I was not as motivated -- and I was more interested in studying for math olympiads than programming.

Then, in year 2000, I randomly stumbled upon a programming contest by a recommendation from my middle school teacher -- not because I knew how to write a program, but because I was good at using some of the PC tools for homework. I thought it would be fun to participate, so I did. It turned out that I was taking a written exam on basic programming questions, such as ``what will be the output of the following program?'' I was given thirty multiple-choice questions, and I was not able to answer any of them with confidence  -- after all, I knew nothing about programming.

For some reason, this bothered me to a great extent so that I asked my parents to let me study computer programming more seriously at a private institute. The institute happened to be the most successful institute in South Korea at the time in the sense that every year they would send three to four kids to the International Olympiad in Informatics (out of four kids per country). Without knowing this, I took an entrance exam for the institute, and failed; I was asked to write a program in any language of my choice which could output the following:

\begin{verbatim}
 1  2  3  4  5
16 17 18 19  6
15 24 25 20  7
14 23 22 21  8
13 12 11 10  9
\end{verbatim}

To my surprise, I did not know how to do this even though I had written hundreds of for-loops by then, and I was ashamed; to their surprise, I knew nothing, and I wanted to attend this institute. As my father recalls, he saw something in my eyes that day, and he was sure I wanted to, and needed to, try it anyway. So




After all, I think I was lucky enough to find the right resources at the right time.  I have enjoyed studying computer science, learning how research is done at top universities, and solving problems that are exciting to me.


