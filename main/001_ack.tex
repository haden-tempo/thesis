It has been quite a journey to pursue a doctoral degree in computer science, and finally put an end to it. My journey began two decades ago when I first used a personal computer to play games. Computers and programs were a fascinating thing, and problem solving became my hobby and specialty as I participated in various programming contests -- mostly, algorithm design and problem solving. 

Studying computer science as my major in undergrad was a no-brainer for me -- it was the only thing I knew how to do and the only thing I wanted to do better. Just a couple of months before my graduation, Johannes asked me what I planned to do after graduation; at the time, I accepted a full-time job offer from a trading company in Chicago at which I interned before. I told him, and he suggested whether I would give a try to work as a research assistant at his database lab, and possibly apply for graduate school a year later. I thought about it, and I took the offer. 


% My journey of studying computer science started nearly two decaeds ago when I first used a personal computer to play games. Just like anyone else around me, I wanted to create a fun game of my own, and decided to learn how to write a program. I first learned QBasic, and it was fun to write a program that prints out a bunch of characters on an old CRT monitor. At the time playing games was more fun to me than writing programs -- partly because I was not as motivated -- and I was more interested in studying for math olympiads than programming.
%
% Then, in year 2000, I randomly stumbled upon a programming contest by a recommendation from my middle school teacher -- not because I knew how to write a program, but because I was good at using some of the PC tools for homework. I thought it would be fun to participate, so I did. It turned out that I was taking a written exam on basic programming questions, such as ``what will be the output of the following program?'' I was given thirty multiple-choice questions, and I was not able to answer any of them with confidence  -- after all, I knew nothing about programming.
%
% For some reason, this bothered me to a great extent so that I asked my parents to let me study computer programming more seriously at a private institute. The institute happened to be the most successful institute in South Korea at the time in the sense that every year they would send three to four kids to the International Olympiad in Informatics (out of four kids per country). Without knowing this, I took an entrance exam for the institute, and failed; I was asked to write a program in any language of my choice which could output the following:
%
% \begin{verbatim}
%  1  2  3  4  5
% 16 17 18 19  6
% 15 24 25 20  7
% 14 23 22 21  8
% 13 12 11 10  9
% \end{verbatim}
%
% To my surprise, I did not know how to do this even though I had written hundreds of for-loops by then, and I was ashamed; to their surprise, I knew nothing, and I wanted to attend this institute. After all, the institute was to train kids who already know how to write programs but need to practice more on problem solving skills for the contests. As my father recalls, he saw something special in my eyes that day, and he was sure I wanted to, and needed to, try it anyway. So, I did enroll for the most basic level class where I would start from the beginning -- by printing out ``hello, world.''
%
% A year later, I participated in the same contest again, and I barely passed the preliminary contest as the very last perosn to move on to the next round, a minucipal contest. Again, I barely passed the minicipal contest to move on to a national contest as the second-to-last person. In the national contest, I barely made it to a sliver medalist, which meant that I would be able to participate in a training camp for the International Olympiad in Informatics. Three years later, I represented South Korea to participate in the Internatinoaly Olympiad in Informatics. It was the moment when I first felt that I had accomplished something great that I had been longing for.

% After all, I think I was lucky enough to find the right resources at the right time.  I have enjoyed studying computer science, learning how research is done at top universities, and solving problems that are exciting to me.
%
%
