Scheduling an event for a group of agents is a challenging problem. 
It tends to be tedious and time-consuming for everyone involved.
In practice, procrastination and strategic behavior of agents are often a problem, but even if we assume that agents are truthful, prompt, and indifferent among the possible outcomes, the group scheduling problem exhibits an interesting problem. 
We formally model this as an optimization problem, and provides an intuitive, efficient algorithm that solves the problem. We shall define the underlying problem as abstract and general as possible, and discuss how it can be applied to other real world settings than group scheduling problems.

A different setting that is relevant to group scheduling is a group assignment problem. Consider a setting where the organizer wishes to assign agents to social activities. Agents may have preferences over activities as well as the number of participants in the activity they are assigned to (but it is assumed that agents have anonymous preferences in that they do not care who participates and who does not). For instance, agents may prefer more participants in social networking receptions, but wish to have not too many participants in a table tennis tournament. Naturally the organizer wishes to assign as many agents to activities as possible, but at the same time he wishes to ensure that every individual is satisfied with the assignment. We study the solution concept of individual rationality and stability which require that there is no single-agent deviation from an assignment. This problem is known to be NP-hard in general even if certain restrictions of preferences of agents are assumed. In this work we first show that the problem is still computationally difficult even if we seek a small, fixed-size solution in general. However, we also show that for a relaxed version of the stability requirement, the problem admits an efficient algorithm for finding a fixed-size, small solution.

Next we consider a special case of the assignment problem when there is only one activity, which is known to be in P. We introduce friends-and-enemies relationship among agents, and we show if the number of friends or enemies of an agent is large (namely, two or more), then the problem of finding a stable solution is NP-hard. Furthermore, finding a fixed-size, small solution also depends on the cardinality of the largest friend-set or enemy-set, which naturally categorizes the underlying problem into different complexity classes. We also show that the problem becomes computationally easier if the friends and enemies relationship is symmetric.
Lastly we consider strategic agents in this problem where agents may report false information to the event organizer. We show that in general finding a stable solution and strategy-proofness are incompatible in the activity selection problem, but we also provide a socially optimal, computationally efficient, and strategy-proof mechanism in the special case where there is only one activity and the preferences of agents align with the goal of the designer (so as to maximize the number of participants).

