Scheduling an event for a group of agents is a challenging problem. 
It tends to be tedious and time-consuming for everyone involved. 
In practice, procrastination and strategic behavior of agents are often a problem, but even if we assume that agents are truthful, prompt, and indifferent among the possible outcomes, the group scheduling problem exhibits an interesting algorithmic problem. Assuming truthful and prompt behavior of agents, we consider settings where the event organizer has probabilistic estimates on availability of agents and is allowed to query agents for their actual availability. Naturally, it is desirable to minimize the (expected) number of queries so as to optimize the time and inconvenience caused by the scheduling process. We consider two models that are motivated by an existing tool that is widely used in practice today, and offer intuitive and computationally efficient algorithms that are applicable to group scheduling settings. Furthermore, we also discuss how our algorithms can be used in different domains than group scheduling. 

Another setting that is relevant to group scheduling comes from group assignment problems. Imagine that an event organizer wishes to assign agents to social activities. Agents may have preferences over activities as well as the number of participants in the activity they are assigned to. The organizer would like to assign as many agents to activities as possible, but not at the cost of upsetting some agents by disregarding their preferences. Finding a Nash stable solution is an interesting algorithmic question of its own in this setting, and it has been studied in the literature that many variants of the problem are computationally difficult (i.e., NP-hard). In this thesis, we take an extra step to investigate some of these problems by analyzing their parameterized complexity when the size of a solution is parameterized. Parameterized complexity offers a finer scale than classical complexity, and we classify many variants of the group assignment problem into different complexity classes in the W-hierarchy. Motivated by this problem, we also consider a 


For instance, agents may prefer more participants in social networking receptions, but wish to have not too many participants in a table tennis tournament. Naturally the organizer wishes to assign as many agents to activities as possible, but at the same time he wishes to ensure that every individual is satisfied with the assignment. We study the solution concept of individual rationality and stability which require that there is no single-agent deviation from an assignment. This problem is known to be NP-hard in general even if certain restrictions of preferences of agents are assumed. In this work we first show that the problem is still computationally difficult even if we seek a small, fixed-size solution in general. However, we also show that for a relaxed version of the stability requirement, the problem admits an efficient algorithm for finding a fixed-size, small solution. 

We then futher generalize the group assignment problem by considering the cases where agents have friend and/or enemy relationships which introduce additional constraints or preferences. Because most of the technical results we obtain for the anonymous case are already hardness results, we focus on a special case where there is only one activity, which is known to be easy (i.e., polynomial-time solvable) when agents have anonymous preferences. We show if the number of friends or enemies of an agent is large (namely, two or more), then the problem of finding a stable solution is NP-hard. Furthermore, finding a fixed-size, small solution also depends on the cardinality of the largest friend-set or enemy-set, which naturally categorizes the underlying problem into different complexity classes. We also show that the problem becomes computationally easier if the friends and enemies relationship is symmetric.

Lastly we consider strategic agents in this setting where agents may report false information to the event organizer. We show that in general finding a stable solution and strategy-proofness are incompatible in the activity selection problem, but we also provide a socially optimal, computationally efficient, and strategy-proof mechanism in the special case where there is only one activity and the preferences of agents align with the goal of the designer (so as to maximize the number of participants).
