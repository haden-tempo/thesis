\section{Introduction and Related Work}  \label{sec:SIP:intro}
Imagine an event organizer trying to convene an event -- for example,
a fundraiser. We assume that the time and venue for the event are fixed,
and that the only remaining decision for the organizer to make is whom
to invite among a set of agents.
An \emph{invitation} is simply defined to be a subset of agents.
The goal of the organizer is to maximize attendance (for example, 
in order to maximize donations), but the potential invitees have 
their own preferences over how many attendees there should be at the event 
and possibly also who the potential attendees should be.
For example, a given donor may not want to attend if too few attendees 
show up, but she may not want the event to be overly crowded. 
Another donor may want to attend the event only
if her friends attend and her business competitor does not.

To model this setting, we turn our attention to the restricted case of the Group Activity Selection Problem (\GASP) with just one activity, but we generalize preferences of agents. 
Specifically, in the Stable Invitations Problem (\SIP), each agent 
can specify a set of friends and a set of enemies (in addition to her preference over sizes).
An agent is willing to attend only if all of her friends attend, none of her enemies attends, and 
the number of attendees is acceptable to her. 
Note that the Stable Invitations Problems is a generalization of the Group Activity Selection Problem with the restriction of one activity present. 

Not surprisingly, complexity of \SIPs depends highly on the cardinality of friend-sets and enemy-sets, as friends-and-enemies relationship introduces combinatoric complexity in the problem of finding a good solution. 
In this chapter we provide a complete analysis of complexity results on \SIPs; we consider individual rationality (IR) and Nash stability as we did for \GASP, and we also consider both asymmetric and symmetric friends-and-enemmies relation. 

\paragraph{Related Work.} %TODO: To relate this to previous chapter.


The rest of this chapter is organized as follows.
%TODO

\section{Definitions and Known Results} \label{sec:SIP:prelim}

To make this work self-contained, we begin by introducing the formal definitions proposed by 
Lee and Shoham~\shortcite{LEE15AAAI}, yet we make slight modifications to notation for readability and consistency in this paper.

\begin{definition}
An instance of the Stable Invitations Problem (\SIP) is given by a set of agents $N = \{a_1, a_2, \dots, a_n\}$, and an {\em approval set} $S_i \subseteq [1,n]$, a {\em friend set} $F_i \subseteq N$, and an {\em enemy set} $E_i \subseteq N$ for each agent $a_i\in N$.
It is interpreted that agent $a_i$ is willing to attend if all friends in $F_i$ attend, no one in $E_i$ attends, and the number of attendees (including $a_i$) is acceptable (i.e., the number is contained in $S_i$).
\end{definition}

\begin{definition} 
An invitation $I$ in \SIPs is a subset of agents.

	We say that an invitation $I$ is {\em individually rational} (IR) if for every agent $a_i\in I$, $|I| \in S_i$, $F_i \subseteq I$, and $R_i \cap I = \emptyset$.
	
	We say that an invitation $I$ is {\em (Nash) stable} if it is individually rational, and if for every agent $a_j \not\in I$, $|I_j'| \not\in S_j$, $F_j \not\subseteq I_j'$, or $R_j \cap I_j' \neq \emptyset$ where $I_j' = I \cup \{a_j\}$.
\end{definition}

Individual rationality (IR) requires that every invited agent is willing to attend.
Stability further requires that those who are not invited are not willing to participate (without permission of others)
because not all of her friends are attending, some of her enemies are attending, or the number of attendees would be unacceptable. 
We consider the following two problems of finding invitations of size $k$:
\begin{itemize}
	\item $k$-IR-Invitation: $\exists$ IR invitation of size $k$?
	\item $k$-Stable-Invitation: $\exists$ stable invitation of size $k$?
\end{itemize}

We first consider restrictions on inputs by limiting the size of largest friend-sets and enemy-sets, respectively. 
For integer constants $\alpha$ and $\beta$, if an instance of \SIPs satisfies $|F_i| \leq \alpha$ and $|E_i| \leq \beta$ for all $a_i\in N$, we call it an $(\alpha,\beta)$-instance of \SIP. 
Lee and Shoham~\shortcite{LEE15AAAI} showed that \SIPs can be solved in polytime only if $\alpha$ and $\beta$ are small enough, but the problems are NP-hard in general. We will consider the same restrictions in this work, and provide our complete analysis of parameterized complexity of \SIP.

In addition to these restrictions, we consider the special case where agents have symmetric social relationships.
\begin{definition} \label{def:symmetric_social}
	Given an instance of \SIP, we say that agents have {\em symmetric social relationships} if $a_j\in F_i$ if and only if $a_i\in F_j$ and $a_l \in E_i$ if and only if $a_i \in E_l$ for every $a_i$. 
\end{definition}

Theorem~\ref{thm:nphard} summarizes the most relevant results of Darmann et al.~\shortcite{GASP12WINE} and Lee and Shoham~\shortcite{LEE15AAAI} on complexity of \SIPs.\footnote{
Darmann et al.~\shortcite{GASP12WINE} showed easiness when $\alpha=\beta=0$, while Lee and Shoham~\shortcite{LEE15AAAI} proved easiness and hardness in all other cases.}
\begin{theorem} \label{thm:nphard} [\cite{LEE15AAAI,GASP12WINE}]
	$k$-IR-Invitation and $k$-Stable-Invitation can be solved in polynomial time if $(\max_{a_i \in N} |F_i|) + (\max_{a_i \in N} |E_i|) \leq 1$ (i.e., $\alpha + \beta \leq 1$). In other cases, both problems are NP-hard.
\end{theorem}
Note that $k$-IR-Invitation and $k$-Stable-Invitation are of the same classical complexity, even though stability is a stronger solution concept. Under parameterization, however, these two problems are contained in different complexity classes in the W-hierarchy (see Table~\ref{tbl:summary}).
In what follows, we show that the parameterized complexity of these problems varies with different solution concepts and under different restrictions on inputs to \SIP.

